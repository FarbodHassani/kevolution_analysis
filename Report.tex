\documentclass[a4paper,12pt]{article}
%% My standard included packages
%\pdfoutput=1 % if your are submitting a pdflatex (i.e. if you have
%             % images in pdf, png or jpg format)
%\usepackage{jcappub} % for details on the use of the package, please
%                     % see the JCAP-author-manual
%\usepackage[T1]{fontenc} % if needed

\usepackage{setspace}           % Allows easy changes to line spacing 
\usepackage{graphicx}           % Allows including of graphics files
\usepackage{amsmath}            % Additional math capabilities
\usepackage{marginnote}         % Used with todonotes package
\usepackage{datetime}           % Allows formatting of date and time
\newcommand {\be}{\begin{equation}}
\newcommand {\ee}{\end{equation}}

\usepackage{empheq}
\usepackage{cancel}
\usepackage{etoolbox}


\usepackage{enumitem} 
\usepackage{color}
%Mathematica colors
\definecolor{identifiercolor}{rgb}{.4,.6,.56}
\definecolor{stringcolor}{gray}{0.5}
\definecolor{inactivecolor}{rgb}{0.15,0.15,0.5}
\usepackage{listings}
%Mathematica
\usepackage{listings}
\lstset{basicstyle={\footnotesize\def\fvm@Scale{.85}\fontfamily{fvm}\selectfont},
  breaklines=true,
  escapeinside={\%*}{*)},
  keywordstyle={\bfseries\color{inactivecolor}},
  stringstyle={\bfseries\color{stringcolor}},
  identifierstyle={\bfseries\color{identifiercolor}},
  language=Mathematica,
  otherkeywords={DiscretizeRegion},
  showstringspaces=false}
\renewcommand{\lstlistingname}{Listing}




\usepackage{amsmath}
\usepackage{graphicx}% Use pdf, png, jpg, or eps� with pdflatex; use eps in DVI mode
\usepackage{caption}
\usepackage{subcaption}
          % List formatting commands
\setlist{noitemsep}             % Remove space between list items 
%\usepackage{subfigure}          % Create numbered and captioned subfigures
\usepackage{rotating}           % Create landscape tables and figures
\usepackage[dvipsnames]{xcolor} % Refer to colors by name
\usepackage[colorlinks=true,urlcolor=blue,linkcolor=Blue,citecolor=RedViolet]{hyperref}           % URLS and hyperlinks
%\usepackage{hyperref}           % URLS and hyperlinks
\usepackage{float}              % Activate [H] option to place figure HERE
\usepackage[numbers]{natbib}
\usepackage{versionPO}          % Include text conditionally
\usepackage{caption}
%\usepackage[utf8]{inputenc}
%\usepackage[nottoc]{tocbibind}
\lstset{basicstyle=\ttfamily,
  showstringspaces=false,
  commentstyle=\color{red},
  keywordstyle=\color{blue}
}
% These next lines allow including or excluding different versions of text
% using versionPO.sty
\includeversion{notes}		% Include notes?
%\excludeversion{notes}
\excludeversion{comment}
\includeversion{links}          % Turn hyperlinks on?
\excludeversion{submit}		% Format for conference submission?
\includeversion{toc}		% Include table of contents?
%\graphicspath{{./Results1-Perihelionadvance}}

% Turn off hyperlinking if links is excluded
\iflinks{}{\hypersetup{draft=true}}

% Notes options
\ifnotes{%
\usepackage[margin=1in,paperwidth=10in,right=2.5in]{geometry}%
\usepackage[textwidth=1.4in,shadow,colorinlistoftodos]{todonotes}%
}{%
\usepackage[margin=1in]{geometry}%
\usepackage[disable]{todonotes}%
}

% Allow todonotes inside footnotes without blowing up LaTeX
% Next command works but now notes can overlap. Instead, we'll define 
% a special footnote note command that performs this redefinition.
%\renewcommand{\marginpar}{\marginnote}%

% Save original definition of \marginpar
\let\oldmarginpar\marginpar
% Workaround for todonotes problem with natbib (To Do list title comes out wrong)
\makeatletter\let\chapter\@undefined\makeatother % Undefine \chapter for todonotes
% Packages included specifically for this document.
\usepackage{texintro}           % Document-specific definitions
\usepackage{tocvsec2}           % More flexible formatting of table of contents
\usepackage{bibentry}           % Print full citation in text
\nobibliography*                                % Allow use of \bibentry command
\usepackage{tikz}             % Already included by todonotes
\usetikzlibrary{matrix}
\usepackage[retainorgcmds]{IEEEtrantools}  % Equation formatting. Option needed to
                                           % allow enumitem to work.

% Workaround for todonotes problem with natbib (To Do list title comes out wrong)
% If you're including tocvsec2, do so before this command.
\makeatletter\let\chapter\@undefined\makeatother % Undefine \chapter for todonotes.

% Number paragraphs and subparagraphs and include them in TOC
\setcounter{tocdepth}{2}

\usepackage[affil-it]{authblk} 
\usepackage{etoolbox}



\def\be{\begin{equation}}
\def\ee{\end{equation}}
\def\bea{\begin{eqnarray}}
\def\eea{\end{eqnarray}}
\def\bean{\begin{eqnarray*}}
\def\eean{\end{eqnarray*}}
\def\cd{\cdot}
\def\vp{\varphi}
\def\l {\langle}
\def\re {\rangle}
\def \dd {\partial}
\def \ra {\rightarrow}
\def \la {\lambda}
\def \La {\Lambda}
\def \De {\Delta}
\def \DH {\Delta_{\rm HI}}
\newcommand{\de}{\delta}
\def \b {\beta}
\def \al {\alpha}
\def \ka {\kappa}
\def \Ga {\Gamma}
\def \ga {\gamma}
\def \si {\sigma}
\def \Si {\Sigma}
\def \ep {\epsilon}
\def \om {\omega}
\def \Om {\Omega}
\def \lap {\triangle}
\def \ep {\epsilon}


%%%%%%%%%%%%%%%%%%%%%%%%%%%%%%%%%%%
%Special definitions for this paper
%%%%%%%%%%%%%%%%%%%%%%%%%%%%%%%%%%%

\newcommand{\MyRed}{\color [rgb]{0.8,0,0}}
\newcommand{\MyGreen}{\color [rgb]{0,0.7,0}}
\newcommand{\MyBlue}{\color [rgb]{0,0,0.8}}
\newcommand{\MyBrown}{\color [rgb]{0.8,0.4,0.1}}
\newcommand{\MyPurple}{\color [rgb]{0.6,0.0,0.6}}
\def\GV#1{{\MyRed [GV: #1]}}
\def\RD#1{{\MyGreen [RD:  {\tt #1}]}} 
\def\RDt#1{{\MyGreen #1}}   
\def\GM#1{{\MyBlue [GM: #1]}}  
\def\GF#1{{\MyPurple [GF: #1]}}    



\newcommand{\ie}{\emph{i. e.}}
\newcommand{\cf}{\emph{cf.}}
\newcommand{\etal}{\emph{et al.}\xspace}
\newcommand{\eg}{\emph{e. g.}}

\newcommand{\Scal}{\mathcal S}
\newcommand{\DD}{\mathcal D}
\newcommand{\EE}{\mathcal E}
\newcommand{\MM}{\mathcal M}
\newcommand{\HH}{\mathcal H}

\newcommand{\Real}{\mathbb{R}}
\newcommand{\bn}{\boldsymbol{n}}
\newcommand{\bv}{\boldsymbol{v}}
\newcommand{\bx}{\boldsymbol{x}}
\newcommand{\bnabla}{\boldsymbol{\nabla}}
\newcommand{\bell}{\boldsymbol{\ell}}
\newcommand{\bal}{\boldsymbol{\alpha}}





%\usepackage{lmodern}
%\renewcommand\Authfont{\fontsize{12}{14.4}\selectfont}
%\renewcommand\Affilfont{\fontsize{9}{10.8}\itshape}
%\renewcommand\Authfont{\fontsize{12}{15}\selectfont}
%\renewcommand\Affilfont{\fontsize{9}{11}\itshape}
\definecolor{astral}{RGB}{46,116,181}
%\subsectionfont{\color{astral}}
%\sectionfont{\color{astral}}
%\usdate{17 May}                         % Use usual LaTeX date layout

%\title{\color{BlueViolet}\Huge{On the accuracy of approximated geodesic equations and different potentials with different numerical methods } }
\title{\color{BlueViolet}\Huge{Reports- EFT implementation project}}
%\vskip 2em
\author{Farbod Hassani}
%\thanks{Email:\href{mailto:farbod.hassani@unige.ch}{{farbod.hassani@unige.ch}}}  \thanks{Homepage: \href{http://www.farbod-hassani.com}{farbod-hassani.com}}}
%\affil{D\'epartement de Physique Th\'eorique and Center for Astroparticle Physics, Universit\'e de Gen\'eve,
%24 quai Ansermet, CH-1211 Gen\'eve 4, Switzerland}

%{farbod-hassani.com}} }
%\newcommand*{\TitleFont}{%     \usefont{\encodingdefault}{\rmdefault}{b}'%     \fontsize{18}{16}%    \selectfont}
%\title{\TitleFont Halo finder}
%\author[1]{{Farbod Hassani} \thanks{ \url{farbod.hassani@gmail.com}
%}
%\thanks{farbod-hassani.com}}
%\author[2]{Author E\thanks{E.E@university.edu}}
%\affil[1]{D\'epartement de Physique Th\'eorique and Center for Astroparticle Physics, Universit\'e de Gen\'eve,
%24 quai Ansermet, CH-1211 Gen\'eve 4, Switzerland}
%\emailAdd{farbod.hassani@gmail.com}
%\affil[2]{Department of Mechanical Engineering, \LaTeX\ University}
      %\begin{abstract}
%This is abstract text: This simple document shows very basic features of \LaTeX{}.
%\lstset { %
%    language=C++,
%    %backgroundcolor=\color{black!5}, % set backgroundcolor
%    basicstyle=\footnotesize,% basic font settings
%}
\begin{document}
  \maketitle
  \tableofcontents
  \flushbottom
  \section{Theory of EFT and the equations and Stress tensor}
%  section{Introduction}
\label{Sec1}
\setcounter{equation}{0}
%%%%%%%%%%%%%%%%%%%%%%%%%%%%%%%%%%%%%%%%%%%%%
\section{Questions:}
What is $\delta P/\delta\rho$ here and is it comparable with other papers? Yes\\
Discuss with Martin about the red term $P_{XX}$ why we do not agree? \\
What about "a" factor from physical to conformal perturbation in Stress tensor??  No we dont need it since from the begining everyhthing is in conformal time! \\
-Scale factror difference in $T_{0i}$

%%%%%%%%%%%%%%%%%%%%%%%%%%%%%%%%%%%%%%%%%%%%%
\section{Todo:}
-Is $X$ function of (t,x) or the field? since if it is the function of field derivative and field derivative is independent of field so $\partial X/\partial \varphi=0$. It is not allowed to take $X (t,\vec{x})$ when we know its functionality and it is not a function of $\varphi$. Why we cant write the equations 6,16-20? so the difference is   $- \mathcal{H} (1+w) $ in my calculation of $T_{00}$ instead of $-3 \mathcal{H}(1+w)$\\
  - A mistake in gauge transformation, where is it? \\
  - Are the equations in box are true?\\
  - The difference between "parabolic" and "elliptic" vector method? Since  I want to define "$T_i^0$" of k-essence... \\
  - What are the checks should be done? Gevolution transfer function at z=0 compared with hi-class results? The effect of k-essence field on matter power? Stability tests? what should be done exactly for stability tests? \\
   -vector elliptic, The difference between vector elliptic and parabolic?\\
  - $\mathcal{H}'$  in the code?! does $(\mathcal{H}^{(n+1)}-\mathcal{H}^{n})/d\tau$ makes sense? \\
 - Why in Gevolution source, $\Phi$ and $\chi$ has the same name in Fourier space? "scalarFT"? \\
 -According to the equation of 120, we need to have two mode $k$ and $k'$ in Fourier space to solve the field equation?! How we should solve it?!
  - Check stress tensor turning on vector elliptic, what is vector elliptic...? \\
  - Implement the IC for $\pi'$ in Gevolution. \\
  - The IC from hi-class, do some checks to find where it goes crazy! \\
  - Write down the full field update equation in theory and implement in the code and track the transfer function!\\
The updating metric/ particles in the sub steps of field update. \\
- Background results.
- Perturbation results, for $\phi$ and $T_{\mu \nu}$  \\
-Try to solve the differential equation in mathematica in 1D, for $c_s^2 ->0$. \\
-Check estimator method, what is the error? \\
- Do the calculation for  kessence $T_{\mu \nu}$, discuss about perturbation in conformal time and physical time. \\
- Compare the linear solution with hi-class results. \\
-Then do non linear run and compare \\
- Solve the initial condition problem. \\
- Put the result for initial condition which is produced by Gevolution. \\
-Fix the problem of kessence $T_{\mu \nu}$ \\
-Solve the differential equation for $\phi$ and see how it should behave. \\
- Add Lorenzo's file for looking at the field. \\
- compute $\delta_{\pi}$ and $\theta_{\pi}$ \\
Note that there is an error because in each loop we assume that $\Psi'$ and $\Phi'$ for kessence updates are constant!!! For 10 time update forexample!
Also according to leap frog and the fact that we update $\pi'$ half step in the first loop while we do not update Hubble constant! so we are making slightly different initial condition which we assume does not matter since the ODE goes to the attractor .. \\
Some points: The units are very important, like wavenumber which is $h/Mpc$ and $\pi$ is in unit of $Mpc$ and $'$ in Gevolution is in terms of conformal time which is in box units!. \\
%\label{Sec2}
%\setcounter{equation}{0}
%%%%%%%%%%%%%%%%%%%%%%%%%%%%%%%%%%%%%%%%%%%%%%%%

 \section{K-essence field equation from EFT action}
 
 We take the metric in ADM form as below,
  \be
  ds^2= -N (t,\vec{x}) ^2 d t^2+ h_{ij } (t,\vec{x}) \Big( dx^i+N^i (t,\vec{x}) dt   \Big) \Big( dx^j+N^j (t,\vec{x}) dt   \Big)
  \ee
  where 
  \be
  N(t,\vec{x})= \bar{N} (t) e^{\epsilon \delta N (t,\vec{x})}
  \ee
  \be
  N^i=\epsilon \sigma^{0i} (t,\vec{x})
  \ee
  \be
    h_{ij}=a^2 \Big( e^{2 \zeta (t,\vec{x}) \epsilon} \delta_{ij} + \epsilon \sigma_{ij} (t,\vec{x})   \Big)
  \ee
  $\epsilon$ shows the order of terms in the scheme.\\
  For the first order equations we can define $\sigma_{ij}=(\partial_i \partial_j- \frac{\nabla^2}{3} \delta_{ij}) B(t,\vec{x}) \epsilon$ and $N^i=\delta ^{ij} \partial_j \psi (t,\vec{x}) \epsilon$  since we can separate the scalar, vector and tensor equations. \\
  On the other hand in second order equations we do observe the mixing of the scalar, vector and tensor equations according to $ T^{\mu \nu} \frac{\delta g_{\mu \nu}}{\delta (scalars)}$, which cannot be written as a derivative of a scalar equation and suggest general definition of  $\sigma_{ij}$ with four degrees of freedom (1 scalar, 1 vector and 2 tensor degrees of freedom). Since here we are only interested in  scalar field equation we do not care about the details of $\sigma_{ij}$.
  \section{Definitions}
 The inverse of the metric is defined by the inverse of the  matrix.
\be
N_i=h_{ij} N^j
\ee
Christoffel symbols:
\be
\Gamma_{\zeta \rho}^{\mu}= \frac{g^{\mu \xi }}{2} \left(  g_{\xi \zeta ,\rho }+ g_{\xi \rho ,\zeta } - g_{\rho \zeta ,\xi }   \right )
\ee
\be
K_{ij}=\frac{1}{2 N (t,\vec{x})} \left [  \dot{h}_{ij} - \nabla_{i} N_{j} - \nabla_{j} N_{i}  \right ]= \frac{1}{2 N (t,\vec{x})} \left [  \dot{h}_{ij} - \partial_{i} N_{j} - \partial_{j} N_{i} -2  \Gamma_{i j}^{l} N_l  \right ] 
\ee
\be
\delta K= K_i^i(t,\vec{x}) -\bar {K}_i^i (t)
\ee
Full metric,
\be
g_{00}= -N^2(t,\vec{x})+ h_{ij} N^i N^j \,,\, g_{ij}=h_{ij} \, , \, g_{0i}=g_{i0}=h_{ij}N^j
\ee
Riemann tensor
\be
R^{\rho}_{\sigma \mu \zeta}= \partial_{\mu} \Gamma_{\zeta \sigma}^{\rho}- \partial_{\zeta} \Gamma_{\mu \sigma}^{\rho} + \Gamma_{\mu \lambda}^{\rho} \Gamma_{\zeta \sigma}^{\lambda} -  \Gamma_{\zeta \lambda}^{\rho} \Gamma_{\mu \sigma}^{\lambda}
\ee
Ricci tensor;
\be
R_{\mu \rho}=R ^{\eta}_{\mu  \eta  \rho}
\ee
Ricci scalar;
\be
R=g^{\mu \rho} R_{\mu \rho}
\ee
 
 \subsection{Stuckelberg trick}
\be
f(t) \longrightarrow f(t) +  \dot{f} (t) \pi+ \frac{1}{2} \ddot{f }(t) \pi^2  + \frac{1}{6} \dddot{f }(t) \pi^3
\ee
\be
\Lambda(t) \longrightarrow \Lambda(t) +  \dot{\Lambda} (t) \pi+ \frac{1}{2} \ddot{\Lambda}(t) \pi^2 + \frac{1}{6} \dddot{\Lambda }(t) \pi^3
\ee
\be
M_2^4(t) \longrightarrow M_2^4(t) +  \dot{ M_2^4} (t) \pi+ \frac{1}{2}    \ddot{ M_2^4 }(t)  \pi^2 + \frac{1}{6} \dddot{M_2^4 }(t) \pi^3
\ee
\be
m_3^3(t) \longrightarrow m_3^3(t) +  \dot{m_3^3} (t) \pi+ \frac{1}{2} \ddot{m_3^3}(t) \pi^2 + \frac{1}{6} \dddot{m_3^3 }(t) \pi^3
\ee
\be
g^{00} \longrightarrow g^{00} + 2 g^{0 \mu} \partial_{\mu} \pi + g^{\rho \nu} \partial_{\rho} \partial_{\nu} \pi
\ee
\be
\partial_0 \longrightarrow \left( 1- \dot{\pi} - \dot{\pi}^2\right) \partial_0
\ee
\be
\partial_i \longrightarrow \partial_i-  \left( 1- \dot{\pi} \right) \partial_i \pi \partial_0
\ee
\be
N \longrightarrow N \left(1-\dot{\pi} + \dot{\pi}^2+N^i\partial_i \pi + \frac{1}{2} N^2 h^{ij} \partial_i \pi \partial_j \pi \right)
\ee
\be
N^i \longrightarrow N^i(1- \dot{\pi} ) +(1- 2 \dot{\pi}) N^2 h^{ik} \partial_k \pi
\ee
\be
h_{ij} \longrightarrow h_{ij}- N_i \partial_j \pi -N_j \partial_i - N^2\partial_i \pi \partial_j \pi
\ee
\begin{align}
\delta K  \longrightarrow &   \delta K -3 \left ( \dot{H} \pi +\frac{1}{2} \ddot{H} \pi^2 \right ) - (1-\dot{\pi}) N h^{ij} \partial_i \partial_j \pi +\frac{1}{2} \partial_i h^{ij} \partial_j \pi  \nonumber \\ &+\frac{H}{2 a^2} \delta ^{ij}\partial_i \pi \partial_j \pi + \frac{2}{a^2} \delta ^{ij} \partial_i  \pi  \partial_j    \dot{\pi} -\frac{2}{a^2} \delta ^{ij} \partial_i N \partial_j \pi
\end{align}
{\color{red}The last Stuckelberg trick (on K) is not true for second order, so one should write the Stuckelberg using the mathematica!}
%\begin{align}
%\delta K  \longrightarrow &   \delta K -3 \left ( \dot{H} \pi +\frac{1}{2} \ddot{H} \pi^2 \right ) - (1-\dot{\pi}) N h^{ij} \partial_i \partial_j \pi +\frac{1}{2} \partial_i h^{ij} \partial_j \pi  \nonumber \\ &+\frac{H}{2 a^2} \delta ^{ij}\partial_i \pi \partial_j \pi + \frac{2}{a^2} \delta ^{ij} \partial_i  \pi  \partial_j    \dot{\pi} -\frac{2}{a^2} \delta ^{ij} \partial_i N \partial_j \pi
%\end{align}
The EFT action is;
\be
S=\sqrt{-g} \left [ \frac{M_*^2}{2} f(t) R -\Lambda (t) -c(t) g^{00} +\frac{M_2^4(t)}{2} \left (g^{00} + \frac{1}{\bar{N}^2} \right )^2    -  \frac{m_3^3(t)}{2} \delta K  \left (g^{00} + \frac{1}{\bar{N}^2} \right )    \right ]
\ee

The scalar field dynamics is obtained by varying the action with respect to the $\pi$. \\
  To change the gauge from unitary to Newtonian we use the following transformation in the variables. (Note that from now on we follow the notation of Gevolution where $\Psi$ is perturbation in time component and $\Phi$ is for spatial component while EFT papers are opposite)
 \be
 \delta N \rightarrow \Psi \, , \, \zeta \rightarrow-\Phi \, , \, \psi   \rightarrow0,  \,  \,  B \rightarrow0
 \ee

\begin{align}
\frac{1}{\sqrt{-g}} \frac{\delta S}{\delta \pi}|_{\text{First order} }&=  B_{\Psi} \Psi+  B_{\dot{\Psi}} \dot{\Psi} +
B_{\Phi} \Phi + B_{\dot{\Phi}} \dot{\Phi}  + B_{\ddot{\Phi}} \ddot{\Phi}+B_{\pi} \pi +   B_{\dot{\pi}} \dot{\pi} + B_{\ddot{\pi}} \ddot{\pi}  
\nonumber  \\& 
- \frac{k^2}{a^2} \left( B^{(2)}_{\Psi}\Psi +  B^{(2)}_{\Phi}\Phi+ B^{(2)}_{\dot{\Phi}} \dot{\Phi} + B^{(2)}_{\pi}\pi \right) + \frac{k^4}{a^4} \left(B^{(4)}_{\Phi}\Phi +B^{(4)}_{\pi}\pi  \right)
\end{align}
where,
\be
B_{\Psi}=12 c H+2 \dot{c} +3 m_3^3 (3H^2+2\dot{H}) -6 M_*^2\dot{f} (\dot{H} + 2H^2) + 3H \left[ 4M_2^4 +\dot{(m_3^3)}\right]+4 \dot{(M_2^4)}
\ee
The last equation is different with Essential building paper, because of a typo in the paper. Moreover in eq. 191 the second one is not equivalent to first one, there is a sign difference according to eq. 153 and taking derivative.
\be
B_{\dot{\Psi}}=2c + 4 M_2^4 +3 H (m_3^3 -M_*^2 \dot{f})
\ee
\be
B_{\Phi}=0
\ee
\be
B_{\dot{\Phi}}=3 \left[ 2c +3 H m_3^3-4 H M_*^2 \dot{f} +\dot{m_3^3}\right]
\ee
\be
B_{\ddot{\Phi}}=3( m_3^3 -M_*^2 \dot{f})
\ee
\be
B_{{\pi}}=- \Big[ -3\dot{m_3^3} \dot{H} - 6 \dot{H} c + 3 M_*^2 (\ddot{H} +4 H \dot{H})\dot{f} - 9 H \dot{H} m_3^3- 3 m_3^3 \ddot{H}  \Big]
\ee
\be
B_{\dot{\pi}}=- 2\Big[ 3 H (c+ 2 M_2^4) +\dot{c} + 2\dot{M_2^4} \Big]
\ee
\be
B_{\ddot{\pi}}=- 2\Big[  c+ 2 M_2^4 \Big]
\ee
\be
B^{(2)}_{{\Psi}}= \Big[  m_3^3  - M_*^2 \dot{f}\Big]
\ee
\be
B^{(2)}_{{\Phi}}= 2  M_*^2 \dot{f} 
\ee
\be
B^{(2)}_{\dot{\Phi}}=
  0
\ee
\be
B^{(2)}_{{\pi}}=
 \Big[ 2c  + \dot{m_3^3}+ H m_3^3 \Big]
\ee
\be
B^{(4)}_{{\Phi}}=0
\ee
\be
B^{(4)}_{{\pi}}=
0
\ee
The relevant second order, short wave correction terms in Fourier space are,
\begin{align}
 \frac{1}{\sqrt{-g}} \frac{\delta S}{\delta \pi}|_{\text{short wave} }=  & \int \int d^3k d^3 k' e^{i(\vec{k}+\vec{k}') . \vec{x}}  \Bigg [   -\frac{k^2}{a^2} C^{(2)}_{\Psi \Psi} \Psi \Psi  -\frac{k^2}{a^2} C^{(2)}_{\Psi \Phi} \Psi \Phi
   -\frac{k^2}{a^2} C^{(2)}_{\Psi \pi} \Psi \pi  -\frac{k^2}{a^2} C^{(2)}_{\Psi \dot{\Psi}} \Psi \dot{\Psi}  -\frac{k^2}{a^2} C^{(2)}_{\Psi \dot{\Phi}} \Psi \dot{\Phi}   -\frac{k^2}{a^2} C^{(2)}_{\Psi \dot{\pi}} \Psi \dot{\pi}  
  \nonumber  \\& 
   -\frac{k^2}{a^2} C^{(2)}_{\Phi \Psi} \Phi \Psi 
             -\frac{k^2}{a^2} C^{(2)}_{\Phi \Phi} \Phi \Phi 
     -\frac{k^2}{a^2} C^{(2)}_{\Phi \pi} \Phi \pi 
      -\frac{k^2}{a^2} C^{(2)}_{\Phi \dot{\Psi}} \Phi \dot{\Psi}  
      -\frac{k^2}{a^2} C^{(2)}_{\Phi \dot{\Phi}} \Phi \dot{\Phi}  
             -\frac{k^2}{a^2} C^{(2)}_{\Phi \dot{\pi}} \Phi \dot{\pi}  
  \nonumber  \\& 
   -\frac{k^2}{a^2} C^{(2)}_{\pi \Psi} \pi \Psi 
    -\frac{k^2}{a^2} C^{(2)}_{\pi \Phi} \pi \Phi 
      -\frac{k^2}{a^2} C^{(2)}_{\pi \pi} \pi \pi 
  -\frac{k^2}{a^2} C^{(2)}_{{\pi} \dot{\Psi}} {\pi} \dot{\Psi}
    -\frac{k^2}{a^2} C^{(2)}_{{\pi} \dot{\Phi}} {\pi} \dot{\Phi}
    -\frac{k^2}{a^2} C^{(2)}_{{\pi} \dot{\pi}} {\pi} \dot{\pi}
          \nonumber \\&  
  %//////////////////////////
  -\frac{k^2}{a^2} C^{(2)}_{\dot{\Psi} \Psi} \dot{\Psi} \Psi
    -\frac{k^2}{a^2} C^{(2)}_{\dot{\Psi} \Phi} \dot{\Psi} \Phi
  -\frac{k^2}{a^2} C^{(2)}_{\dot{\Psi} \pi} \dot{\Psi} \pi
  -\frac{k^2}{a^2} C^{(2)}_{\dot{\Psi} \dot{\Psi}} \dot{\Psi}  \dot{\Psi}
  -\frac{k^2}{a^2} C^{(2)}_{\dot{\Psi} \dot{\Phi}} \dot{\Psi}  \dot{\Phi}
  -\frac{k^2}{a^2} C^{(2)}_{\dot{\Psi} \dot{\pi}} \dot{\Psi}  \dot{\pi}
          \nonumber \\&  
  -\frac{k^2}{a^2} C^{(2)}_{\dot{\Phi} \Psi} \dot{\Phi} \Psi
    -\frac{k^2}{a^2} C^{(2)}_{\dot{\Phi} \Phi} \dot{\Phi} \Phi
  -\frac{k^2}{a^2} C^{(2)}_{\dot{\Phi} \pi} \dot{\Phi} \pi
  -\frac{k^2}{a^2} C^{(2)}_{\dot{\Phi} \dot{\Psi}} \dot{\Phi}  \dot{\Psi}
  -\frac{k^2}{a^2} C^{(2)}_{\dot{\Phi} \dot{\Phi}} \dot{\Phi}  \dot{\Phi}
  -\frac{k^2}{a^2} C^{(2)}_{\dot{\Phi} \dot{\pi}} \dot{\Phi}  \dot{\pi}
            \nonumber \\&  
  -\frac{k^2}{a^2} C^{(2)}_{\dot{\pi} \Psi} \dot{\pi} \Psi
    -\frac{k^2}{a^2} C^{(2)}_{\dot{\pi} \Phi} \dot{\pi} \Phi
  -\frac{k^2}{a^2} C^{(2)}_{\dot{\pi} \pi} \dot{\pi} \pi
  -\frac{k^2}{a^2} C^{(2)}_{\dot{\pi} \dot{\Psi}} \dot{\pi}  \dot{\Psi}
  -\frac{k^2}{a^2} C^{(2)}_{\dot{\pi} \dot{\Phi}} \dot{\pi}  \dot{\Phi}
  -\frac{k^2}{a^2} C^{(2)}_{\dot{\pi} \dot{\pi}} \dot{\pi}  \dot{\pi}
      %///////////////////
                  \nonumber \\&  
          -\frac{\vec{k}.\vec{k}'}{a^2}  C^{1,1}_{\Psi \Psi} \Psi \Psi 
  -\frac{\vec{k}.\vec{k}'}{a^2}  C^{1,1}_{\Psi \Phi} \Psi \Phi 
     -\frac{\vec{k}.\vec{k}'}{a^2}  C^{1,1}_{\Psi \pi} \Psi \pi 
          -\frac{\vec{k}.\vec{k}'}{a^2}  C^{1,1}_{\Psi \dot{\Psi}} \Psi \dot{\Psi} 
          -\frac{\vec{k}.\vec{k}'}{a^2}  C^{1,1}_{\Psi \dot{\Phi}} \Psi \dot{\Phi} 
          -\frac{\vec{k}.\vec{k}'}{a^2}  C^{1,1}_{\Psi \dot{\pi}} \Psi \dot{\pi} 
  \nonumber \\&  
           -\frac{\vec{k}.\vec{k}'}{a^2}  C^{1,1}_{\Phi \Phi} \Phi \Phi 
     -\frac{\vec{k}.\vec{k}'}{a^2}  C^{1,1}_{\Phi \pi} \Phi \pi 
          -\frac{\vec{k}.\vec{k}'}{a^2}  C^{1,1}_{\Phi \dot{\Psi}} \Phi \dot{\Psi} 
          -\frac{\vec{k}.\vec{k}'}{a^2}  C^{1,1}_{\Phi \dot{\Phi}} \Phi \dot{\Phi} 
          -\frac{\vec{k}.\vec{k}'}{a^2}  C^{1,1}_{\Phi \dot{\pi}} \Phi \dot{\pi} 
            \nonumber \\&  
     -\frac{\vec{k}.\vec{k}'}{a^2}  C^{1,1}_{\pi \pi} \pi \pi 
          -\frac{\vec{k}.\vec{k}'}{a^2}  C^{1,1}_{\pi \dot{\Psi}} \pi \dot{\Psi} 
          -\frac{\vec{k}.\vec{k}'}{a^2}  C^{1,1}_{\pi \dot{\Phi}} \pi \dot{\Phi} 
          -\frac{\vec{k}.\vec{k}'}{a^2}  C^{1,1}_{\pi \dot{\pi}} \pi \dot{\pi} 
 \nonumber \\&  
          -\frac{\vec{k}.\vec{k}'}{a^2}  C^{1,1}_{ \dot{\Psi} \dot{\Psi}}  \dot{\Psi} \dot{\Psi} 
          -\frac{\vec{k}.\vec{k}'}{a^2}  C^{1,1}_{ \dot{\Psi} \dot{\Phi}}  \dot{\Psi} \dot{\Phi} 
          -\frac{\vec{k}.\vec{k}'}{a^2}  C^{1,1}_{ \dot{\Psi} \dot{\pi}}  \dot{\Psi} \dot{\pi} 
 \nonumber \\&  
          -\frac{\vec{k}.\vec{k}'}{a^2}  C^{1,1}_{ \dot{\Phi} \dot{\Phi}}  \dot{\Phi} \dot{\Phi} 
          -\frac{\vec{k}.\vec{k}'}{a^2}  C^{1,1}_{ \dot{\Phi} \dot{\pi}}  \dot{\Phi} \dot{\pi} 
 \nonumber \\&  
          -\frac{\vec{k}.\vec{k}'}{a^2}  C^{1,1}_{ \dot{\pi} \dot{\pi}}  \dot{\pi} \dot{\pi} 
 \nonumber \\&  
 + C_{\ddot{\Phi}} (\Phi,\Psi,\pi) \ddot{\Phi} + C_{\ddot{\Psi}} (\Phi,\Psi,\pi) \ddot{\Psi}+ C_{\ddot{\pi}} (\Phi,\Psi,\pi) \ddot{\pi}
 \Bigg ] 
  \text{ .}
\end{align}
Where non zero terms are,
\be
C^{(2)}_{\Phi \Phi}= 4 M_*^2 \dot{f} \text{ .}
\ee
\be
C^{(2)}_{\Phi \pi}= 2  M_*^2 \ddot{f}     \text{ .}
\ee
\be
C^{(2)}_{\Psi \Psi}=  - m_3^3 \text{ .}
\ee
\be
C^{(2)}_{\Psi \Phi}= 2(m_3^3 - M_*^2 \dot{f}) \text{ .}
\ee
\be
C^{(2)}_{\Psi \pi}=  (\dot{m_3^3} -M_*^2 \ddot{f}) \text{ .}
\ee
\be
C^{(2)}_{\Psi \dot{\pi}}= m_3^3\text{ .}
\ee
\be
C^{(2)}_{\pi \Phi}=4c + 2 H m_3^3 + 2 \dot{m_3^3}\text{ .}
\ee
\be
C^{(2)}_{\pi \Psi}= -4 M_2^4-4 H m_3^3\text{ .}
\ee
\be
C^{(2)}_{\pi \pi}=  -3m_3^3 \dot{H} +  H \dot{m_3^3} + 2 \dot{c} +  \ddot{m_3^3}   \text{ .}
\ee
\be
C^{(2)}_{\pi \dot{\Phi}}=  -4 m_3^3  \text{ .}
\ee
\be
C^{(2)}_{\pi \dot{\Psi}}=  -m_3^3  \text{ .}
\ee
\be
C^{(2)}_{\pi \dot{\pi}}=  4 M_2^4  \text{ .}
\ee
%/////////////////////////////////////////////////////////
\be
C^{1,1}_{\Phi \Phi}=	- M_*^2 \dot{f}	\text{ .}
\ee
\be
C^{1,1}_{\Psi \Phi}= - (m_3^3 -M_*^2 \dot{f}) \text{ .}
\ee
\be
C^{1,1}_{\Phi \pi}= - \left (  m_3^3 H + 2c + \dot{m_3^3}  \right )      \text{ .}
\ee
\be
C^{1,1}_{\Psi \pi}=  2 (-m_3^3 H + c- 2 M_2^4 +\dot{m_3^3 })    \text{ .}
\ee
\be
C^{1,1}_{\Psi \Psi}= - M_*^2 \dot{f}	\text{ .}
\ee
\be
C^{1,1}_{\pi \pi}= -\frac{1}{2} \left (   m_3^3 H^2 - 2 (\dot{c} +2  \dot{M_2^4} -2 m_3^3 \dot{H})+H (-4 M_2^4 +\dot{m_3^3} )  \right )	\text{ .}
\ee
\be
C^{1,1}_{\dot{\Phi} \pi}=  -4 m_3^3	\text{ .}
\ee
\be
C^{1,1}_{\dot{\pi} \pi}=   2 \left( 4 M_2^4 - H m_3^3- \dot{m_3^3}\right) 	\text{ .}
\ee
\be
C^{1,1}_{\dot{\pi} \Psi}=  2 m_3^3	\text{ .}
\ee
\be
C_{\ddot{\pi}} (\Psi,\Phi,\pi)= 	- m_3^3 \frac{k^2}{a^2} {\pi}\text{ .}
\ee
Some Important notes: \\
In order to write the equation we must write all the terms, since its possible we have made a mistake somewhere! (So all the equations which have been written up to now must change). \\
Be sure that all the terms are written, for example in the last notes I forgot to write the terms $\nabla^2 \Phi \dot{\pi}$, so write all the terms even if they are zero! Or just check in mathematica and write all non zero terms! \\
By $C_{\ddot{\Psi}} $ I mean the coefficient of $\ddot{\Psi}$ and by  $C^{(1)}_{\dot{\Phi}}$ I mean the coefficient of $\partial_i \dot{\Phi}$ and $ C^{(2)}_{\dot{\Psi} \Phi}$ the coefficient of $\nabla^2\dot{ \Psi }\Phi$. \\
\\
To ensure that everything is right we write the terms in mathematica and cross check! So in the mathematica notebook $GevFieldequs\_pi\_13Feb2018.nb$ after writing down the equation from the mathematica which is checked with Filippo's notebook, we derive all the terms in different orders as following,
\begin{figure}[H]
 \includegraphics[scale=0.5]{0_mathe} 
 \end{figure}
\begin{figure}[H]
 \includegraphics[scale=0.5]{1_mathe} 
 \end{figure}
 \begin{figure}[H]
 \includegraphics[scale=0.5]{2_mathe} 
 \end{figure}
 \begin{figure}[H]
 \includegraphics[scale=0.5]{3_mathe} 
 \end{figure}
For the k-essence case we have (note below equation 85 in \url{https://arxiv.org/pdf/1411.3712.pdf})
\begin{align}
& \alpha_B= \alpha_H=\alpha_M=\alpha_T=0 \nonumber \\ &
\alpha_K=\frac{2\bar{X} P _X + 4 \bar{X}^2 P_{XX}}{M^2 H^2 }  \nonumber \\ &
c_s^2=\frac{-2 \dot{H}}{\alpha_K H^2 } - \frac{\rho + P}{\alpha_K M^2 H^2}
\end{align}
After translation between two different language according to table 1 of  \url{https://arxiv.org/pdf/1411.3712.pdf} and equation 24-25 of \url{https://arxiv.org/pdf/1210.0201.pdf} we get,
\begin{align}
 & m_4^2=\tilde{m}_4^2=\bar{\lambda}=0 \nonumber  \\ &
 f=1 \longrightarrow M^2=3M_{pl}^2,  \;  \;  m_3^3=0 , \;  \; \alpha_k=\frac{2c +4 M_2^4}{M^2 H^2}=  \frac{\Omega (1+w)}{c_s^2} \\ \nonumber &
 \Lambda= \frac{\bar{\rho} (1-w)}{2}, \; \; c=\frac{\bar{\rho} (1+w)}{2}, \; \; 4 M_2^4=\bar{\rho} (1+w) (\frac{1}{c_s^2}-1)
\end{align}
\subsection{Friedmann equations}
In this language the Friedman equations are,
\be
3{H}^2 M_{pl}^2= \rho_m + \rho_{scf}
\ee
Note that in Gevolution we have,
\be
\mathcal{H}^2=\frac{8 \pi G}{3} (\Omega_m a^{-3} +\Omega_{rad} a^{-4} +\Omega_{kess} a^{-3(1+w)} +\Omega_L a^{0} )
\ee
Where the critical density in here assumed 1, ie. $H_0^2=\mathcal{H}_0^2=\frac{8 \pi G}{3}$
and 
\begin{align}
&\frac{\ddot{a}}{a} = - \frac{1}{6M^2_{pl}} \left (\rho_{tot} +3 P_{tot}\right ) \\ \nonumber &
3 H^2 + 2\dot{H} = \frac{-1}{M^2_{pl}} \left( P_{m} + P_{scf} (X, \varphi) \right)
\end{align}
Equivalently it can be written,
\be
\dot{H}= \frac{-(2 \bar{X} P_{,X} + \rho_m+ P_m)}{6 M^2_{pl}}
\ee
where
\be
\bar{X}=\frac{1}{2} \; \; P_{,X} = \bar{\rho} (1+w)
\ee
and,
\be
\frac{\dot{\Omega}}{\Omega}= -3 H(1+w) -\frac{2\dot{H}}{H}
\ee
which is the same as equation 3.5 of \url{https://arxiv.org/pdf/1404.3713.pdf} 
The non-zero terms are:\\
{\color{red} Note that it is assumed that "w" is constant, $\bar{\rho}$ is density of k-essence field and continuity equation for k-essence field is $\dot{\bar{\rho}} +3 H \bar{\rho} (1+w)=0 $}.
\\Note that in order to compare with other papers, like eq. 113 of \url{https://arxiv.org/pdf/1411.3712.pdf} , since in all the terms we have $\bar{\rho}$ we can divide the equation to $3 M_{pl}^2 H^2$ and write everything in terms of $\Omega$ or $\alpha_i$. So of we compare the below result with eq.113 we have 3$M_{pl}^2$ factor a sign difference.
\be
B_{\Psi}=12 c H+2 \dot{c}  + 3H ( 4M_2^4)+4 \dot{(M_2^4)} =  \frac{\dot{\bar{\rho}} (1+w)}{c_s^2} + 3 {H} \bar{\rho} (1+w) \Big( \frac{1}{c_s^2}+1 \Big)=3 {H} \bar{\rho} (1+w) \Big( 1- \frac{w}{c_s^2} \Big )
\ee
\be
B_{\dot{\Psi}}=2c + 4 M_2^4 =  \frac{\bar{\rho} (1+w)}{c_s^2}
\ee
\be
B_{\dot{\Phi}}=6 c = 3 \bar{\rho} (1+w)
\ee
\be
B_{{\pi}}= 6 \dot{H} c= 3 \dot{H} \bar{\rho} (1+w)
\ee
\be
B_{\dot{\pi}}=- 2\Big[ 3 H (c+ 2 M_2^4) +\dot{c} + 2\dot{M_2^4} \Big] = \frac{3 H w (1+w) \bar{\rho} }{c_s^2}
\ee
\be
B_{\ddot{\pi}}=- 2\Big[  c+ 2 M_2^4 \Big]=- \frac{  \bar{\rho}(1+w) }{c_s^2}
\ee
\be
B^{(2)}_{{\pi}}=
2c  =\bar{\rho}(1+w) 
\ee
First order terms several times has checked, everything seems cosistent.

\be
C^{(2)}_{\pi \Phi}=4c =  2 \bar{\rho} (1+w) \text{ .}
\ee
\be
C^{(2)}_{\pi \Psi}= -4 M_2^4=-\bar{\rho} (1+w) (\frac{1}{c_s^2}-1)\text{ .}
\ee
\be
C^{(2)}_{\pi \pi}=   2 \dot{c}=-3 H  \bar{\rho} (1+w) ^2   \text{ .}
\ee
\be
C^{(2)}_{\pi \dot{\pi}}=  4 M_2^4=\bar{\rho} (1+w) (\frac{1}{c_s^2}-1)  \text{ .}
\ee
\be
C^{1,1}_{\Phi \pi}= -   2c =-   \bar{\rho} (1+w)      \text{ .}
\ee
\be
C^{1,1}_{\Psi \pi}=  2 c- 4 M_2^4 =\bar{\rho} (1+w) (2-\frac{1}{c_s^2})      \text{ .}
\ee
\be
C^{1,1}_{\pi \pi}= -\frac{1}{2} \left (  - 2 (\dot{c} +2  \dot{M_2^4} )+H (-4 M_2^4 )  \right )=-\frac{\bar{\rho} H (1+w)} {2 c_s^2} \Big(2+3w+c_s^2  \Big) 		\text{ .}
\ee
\be
C^{1,1}_{\dot{\pi} \pi}=   2 (4 M_2^4)=2\bar{\rho} (1+w) (\frac{1}{c_s^2}-1) 	\text{ .}
\ee
The leading order terms are checked several times with Mathematica notebook. \\
The final equation is:
\begin{align} 
 &3 {H} \bar{\rho} (1+w) \Big( 1- \frac{w}{c_s^2} \Big )
 \Psi + \frac{ \bar{\rho} (1+w)}{c_s^2} \dot{\Psi} + 3 \bar{\rho} (1+w) \dot{\Phi} +3 \dot{H} \bar{\rho} (1+w) \pi + \frac{3 {H} \bar{\rho} (1+w) w}{c_s^2} \dot{\pi} -\frac{  \bar{\rho} (1+w)}{c_s^2} \ddot{\pi} 
 \nonumber \\ 
 &
+ \bar{\rho} (1+w) \frac{\nabla^2 \pi}{a^2}
%////////////////  Second order temrs
  +2  \bar{\rho} (1+w) \Phi  \frac{\nabla^2 \pi }{a^2}   
  %//////////////// 
  -   \bar{\rho} (1+w) (\frac{1}{c_s^2}-1)  \Psi \frac{\nabla^2 \pi }{a^2}   
  %////////////////
  - 3 H \bar{\rho} (1+w)^2 \pi \frac{\nabla^2 \pi }{a^2}  
      \nonumber \\ &
      %////////////////
        +  \bar{\rho} (1+w) (\frac{1}{c_s^2}-1)    \dot{\pi } \frac{\nabla^2 {\pi }}{a^2}   
        %//////////////// 
             -\bar{\rho} (1+w)  \frac{\nabla  \Phi . \nabla \pi }{a^2} 
   %//////////////// 
        +\bar{\rho} (1+w) (2-\frac{1}{c_s^2}) \frac{\nabla  \Psi . \nabla \pi }{a^2}   
   %//////////////// 
   \nonumber \\ &
 -\frac{\bar{\rho} H (1+w)} {2 c_s^2} \Big(2+3w+c_s^2  \Big)\frac{\nabla  \pi . \nabla \pi } {a^2}
    %//////////////// 
    +2  \bar{\rho} (1+w) (\frac{1}{c_s^2}-1)\frac{\nabla  \pi . \nabla \dot{\pi} } {a^2}
     =0
  \end{align}
Simplifying the expression gives,
\begin{align} 
 & 3 {H}  \Big( 1- \frac{w}{c_s^2} \Big )
\Psi + \frac{ 1}{c_s^2} \dot{\Psi} + 3 \dot{\Phi} +3 \dot{H} \pi + \frac{3 {H} w}{c_s^2} \dot{\pi} -\frac{  1}{c_s^2} \ddot{\pi} + \frac{\nabla^2 \pi }{a^2}
   % Second order terms
     +2   \Phi  \frac{\nabla^2 \pi }{a^2}   
  %//////////////// 
  -   (\frac{1}{c_s^2}-1)  \Psi \frac{\nabla^2 \pi }{a^2}   
        \nonumber \\ &
  %////////////////
  - 3 H (1+w)\pi \frac{\nabla^2 \pi }{a^2}  
      %////////////////
        +   (\frac{1}{c_s^2}-1)    \dot{\pi } \frac{\nabla^2 {\pi }}{a^2}   
        %//////////////// 
             - \frac{\nabla  \Phi . \nabla \pi }{a^2} 
   %//////////////// 
        +(2-\frac{1}{c_s^2}) \frac{\nabla  \Psi . \nabla \pi }{a^2}   
   %//////////////// 
 -\frac{H} {2 c_s^2} \Big(2+3w+c_s^2  \Big)\frac{\nabla  \pi . \nabla \pi } {a^2}
    \nonumber \\ &
    %//////////////// 
    +2   (\frac{1}{c_s^2}-1)\frac{\nabla  \pi . \nabla \dot{\pi} } {a^2}     =0 \label{fineq}
%  -(\frac{1}{c_s^2}-1) \nabla^2 \Psi \pi+ 2 \nabla^2 \Phi \pi - 3 H (1+w) \pi \nabla^2 \pi  + (\frac{1}{c_s^2}-1) \pi \nabla^2 \dot{\pi}   \nonumber \\ &+ (2-\frac{1}{c_s^2})\nabla \Psi \nabla \pi - \nabla \Phi \nabla \pi -\frac{H} {2 c_s^2} \Big(2+3w+c_s^2  \Big) \nabla \pi \nabla \pi =0
  \end{align} 
  We have checked that the top equation agrees with equation 113 of \url{https://arxiv.org/pdf/1411.3712.pdf} up to first order. 
  \subsection{Checking the linear equations in mathematica}
  To make sure we cross check the equations in mathematica as following,
  \begin{figure}[H]
 \includegraphics[scale=0.5]{mathe_cross_1} 
 \end{figure}
   \begin{figure}[H]
 \includegraphics[scale=0.5]{mathe_cross_2} 
 \end{figure}
   \begin{figure}[H]
 \includegraphics[scale=0.5]{mathe_cross_3} 
 \end{figure}
   \begin{figure}[H]
 \includegraphics[scale=0.5]{mathe_cross_4} 
 \end{figure}
   \begin{figure}[H]
 \includegraphics[scale=0.5]{mathe_cross_5} 
 \end{figure}
  \subsection{The equation in terms of conformal time}
  To express the last expression in terms of conformal time we only need to follow the following transformations, \\
  \be
  a d \tau= dt
  \ee
  The relation between Hubble and conformal Hubble is;
\be
\mathcal{H} (\tau)=\frac{1}{a(\tau) }\frac{d a(\tau)}{d \tau }= \frac{1}{a(t) } \frac{d a (t) }{d t} \frac{d t }{ d\tau}= a H(t)
\ee
and for the derivative,
\begin{align}
& \dot{H}= \frac{-\mathcal{H}^2+ \mathcal{H}'}{a^2} \nonumber \\ &
\mathcal{H}'=a^2 \Big[ H^2 + \dot{H}\Big ]
\end{align}
The time derivative of any function of time like $\pi(t)$ transforms as,
\be
\pi(x,t)=\pi(x,\tau)
\ee
But both are the perturbation in physical time hypersurfaces.
\be
\dot{\pi}=\frac{\partial \pi}{\partial \tau}  \frac{\partial \tau}{\partial t} = \frac{1}{a(\tau)} \pi '
\ee
\be
\ddot{\pi}=\frac{\partial \tau}  {\partial t} \frac{\partial }{\partial \tau} (\frac{1}{a}\pi ' ) = \frac{1}{a} (-\frac{a ' \pi'}{a^2}+ \frac{\pi''}{a})= \frac{1}{a^2} \Big(- \mathcal{H} \pi' + \pi'' \Big)
\ee
So the equation \ref{fineq} becomes,
\begin{align} 
 & 3 \frac{\mathcal{H}
}{a} \Big( 1- \frac{w}{c_s^2} \Big )\Psi + \frac{ 1}{c_s^2} \frac{\Psi'}{a}+ 3 \frac{\Phi'}{a} + 3  \frac{-\mathcal{H}^2 + \mathcal{H}'}{a^2} \pi + \frac{3 \mathcal{H} w}{ a c_s^2} \frac{\pi'}{a}  -\frac{  1}{ a^2 c_s^2} \Big(- \mathcal{H} \pi' + \pi'' \Big) + \frac{\nabla^2 \pi }{a^2}
% Second order terms
\nonumber \\ &
     +2   \Phi  \frac{\nabla^2 \pi }{a^2}   
  %//////////////// 
  -   (\frac{1}{c_s^2}-1)  \Psi \frac{\nabla^2 \pi }{a^2}   
  %////////////////
  - 3 \mathcal{H} (1+w)\pi \frac{\nabla^2 \pi }{a^3}  
      %////////////////
        +   (\frac{1}{c_s^2}-1) \frac{ \pi' \nabla^2 {\pi }}{a^3}   
        %//////////////// 
             - \frac{\nabla  \Phi . \nabla \pi }{a^2} 
   %//////////////// 
        +(2-\frac{1}{c_s^2}) \frac{\nabla  \Psi . \nabla \pi }{a^2}   
   %//////////////// 
    \nonumber \\ &
 -\frac{\mathcal{H}} {2 a c_s^2} \Big(2+3w+c_s^2  \Big)\frac{\nabla  \pi . \nabla \pi } {a^2}
    %//////////////// 
    +2   (\frac{1}{c_s^2}-1)\frac{\nabla  \pi . \nabla {\pi'} } {a^3}       =0 \label{fineq}
%  -(\frac{1}{c_s^2}-1) \nabla^2 \Psi \pi+ 2 \nabla^2 \Phi \pi - 3 H (1+w) \pi \nabla^2 \pi  + (\frac{1}{c_s^2}-1) \pi \nabla^2 \dot{\pi}   \nonumber \\ &+ (2-\frac{1}{c_s^2})\nabla \Psi \nabla \pi - \nabla \Phi \nabla \pi -\frac{H} {2 c_s^2} \Big(2+3w+c_s^2  \Big) \nabla \pi \nabla \pi =0
  \end{align} 
   Multiplying to $-a^2 c_s^2$ gives:
%  \begin{empheq}[box=\tcbhighmath]{equation}
 \begin{align} 
 &\pi_{\text{phys}}'' - \mathcal{H} \Big (1+ 3w \Big)\pi_{\text{phys}}' -3 {a c_s^2 \mathcal{H}}\Big( 1- \frac{w}{c_s^2} \Big )\Psi -a \, {\Psi'}- 3 c_s^2 a \,{\Phi'} 
  -3  c_s^2 \Big({-\mathcal{H}^2 + \mathcal{H}'} \Big) \pi_{\text{phys}} 
 - c_s^2 {\nabla^2 \pi_{\text{phys}} }
             \nonumber
   \\
    &
    % Second order terms
     -2 c_s^2  \Phi  {\nabla^2 \pi_{\text{phys}} }  
  %//////////////// 
  +   (1-c_s^2)  \Psi {\nabla^2 \pi_{\text{phys}} }
  %////////////////
  +3 c_s^2 \mathcal{H} (1+w)\pi_{\text{phys}} \frac{\nabla^2 \pi_{\text{phys}} }{a}  
      %////////////////
        -   (1-c_s^2) \frac{ \pi_{\text{phys}}' \nabla^2 {\pi_{\text{phys}} }}{a}   
        %//////////////// 
             +c_s^2 {\nabla  \Phi . \nabla \pi_{\text{phys}} }
               \nonumber 
               \\
                &
   %//////////////// 
        -(2 c_s^2-1) {\nabla  \Psi . \nabla \pi_{\text{phys}} }  
   %//////////////// 
 +\frac{\mathcal{H}} {2 a} \Big(2+3w+c_s^2  \Big){\nabla  \pi_{\text{phys}} . \nabla \pi_{\text{phys}} } 
    %//////////////// 
    -2   (1-c_s^2)\frac{\nabla  \pi_{\text{phys}} . \nabla {\pi_{\text{phys}}'} } {a}       =0
   %  -(\frac{1}{c_s^2}-1) \nabla^2 \Psi \pi+ 2 \nabla^2 \Phi \pi - 3 H (1+w) \pi \nabla^2 \pi  + (\frac{1}{c_s^2}-1) \pi \nabla^2 \dot{\pi}   \nonumber \\ &+ (2-\frac{1}{c_s^2})\nabla \Psi \nabla \pi - \nabla \Phi \nabla \pi -\frac{H} {2 c_s^2} \Big(2+3w+c_s^2  \Big) \nabla \pi \nabla \pi =0
  \end{align} 
%\end{empheq}
It is very important to note that $\pi_{\text{phys}}$ here is the perturbation on physical time hypersurfaces, while in hi-class (equation 2.16 of https://arxiv.org/pdf/1605.06102.pdf) it is perturbation on conformal time hypersurfaces. The relation between $\pi_{\text{physical}}$ and $\pi_{\text{conformal}}$ is as following,
\be
\pi_{\text{conf}}= \frac{\delta \varphi_{\text{phys}}}{\bar{\dot{\varphi}} \, a} = \frac{\pi_{\text{phys}}}{a}
\ee
{\color{blue}
\be
\pi_{phys}'=(a \pi_{con})'=a (\mathcal{H} \pi_{con}+ \pi'_{con})
\ee
}
{\color{blue}
\be
\pi_{phys}''=(a \pi_{con})''=a( \mathcal{H}' \pi_{con}+2 \mathcal{H} \pi'_{con} +\pi''_{con}+ \mathcal{H}^2 \pi_{con} )
\ee
}
Substituting gives,
\begin{align} 
 &\pi_{\text{phys}}'' - \mathcal{H} \Big (1+ 3w \Big)\pi_{\text{phys}}' -3 {a c_s^2 \mathcal{H}}\Big( 1- \frac{w}{c_s^2} \Big )\Psi -a \, {\Psi'}- 3 c_s^2 a \,{\Phi'} 
  -3  c_s^2 \Big({-\mathcal{H}^2 + \mathcal{H}'} \Big) \pi_{\text{phys}} 
 - c_s^2 {\nabla^2 \pi_{\text{phys}} }
             \nonumber
   \\
    &
    % Second order terms
     -2 c_s^2  \Phi  {\nabla^2 \pi_{\text{phys}} }  
  %//////////////// 
  +   (1-c_s^2)  \Psi {\nabla^2 \pi_{\text{phys}} }
  %////////////////
  +3 c_s^2 \mathcal{H} (1+w)\pi_{\text{phys}} \frac{\nabla^2 \pi_{\text{phys}} }{a}  
      %////////////////
        -   (1-c_s^2) \frac{ \pi_{\text{phys}}' \nabla^2 {\pi_{\text{phys}} }}{a}   
        %//////////////// 
             +c_s^2 {\nabla  \Phi . \nabla \pi_{\text{phys}} }
               \nonumber 
               \\
                &
   %//////////////// 
        -(2 c_s^2-1) {\nabla  \Psi . \nabla \pi_{\text{phys}} }  
   %//////////////// 
 +\frac{\mathcal{H}} {2 a} \Big(2+3w+c_s^2  \Big){\nabla  \pi_{\text{phys}} . \nabla \pi_{\text{phys}} } 
    %//////////////// 
    -2   (1-c_s^2)\frac{\nabla  \pi_{\text{phys}} . \nabla {\pi_{\text{phys}}'} } {a}       =0
   %  -(\frac{1}{c_s^2}-1) \nabla^2 \Psi \pi+ 2 \nabla^2 \Phi \pi - 3 H (1+w) \pi \nabla^2 \pi  + (\frac{1}{c_s^2}-1) \pi \nabla^2 \dot{\pi}   \nonumber \\ &+ (2-\frac{1}{c_s^2})\nabla \Psi \nabla \pi - \nabla \Phi \nabla \pi -\frac{H} {2 c_s^2} \Big(2+3w+c_s^2  \Big) \nabla \pi \nabla \pi =0
  \end{align} 
%  \begin{empheq}[box=\tcbhighmath]{equation}
 \begin{align} 
 &{\color{blue} \mathcal{H}' \pi_{con}+2 \mathcal{H} \pi'_{con} +\pi''_{con}+ \mathcal{H}^2 \pi_{con} }- \mathcal{H} \Big (1+ 3w \Big)({\color{blue} \mathcal{H} \pi_{con}+ \pi'_{con}}) -3 { c_s^2 \mathcal{H}}\Big( 1- \frac{w}{c_s^2} \Big )\Psi 
    - \, {\Psi'}
 - 3 c_s^2  \,{\Phi'} 
             \nonumber
   \\
    &
  -3  c_s^2 \Big({-\mathcal{H}^2 + \mathcal{H}'} \Big) \pi_{\text{conf}} 
 - c_s^2 {\nabla^2 \pi_{\text{conf}} }
    % Second order terms
     -2 c_s^2  \Phi  {\nabla^2 \pi_{\text{conf}} }  
  %//////////////// 
  +   (1-c_s^2)  \Psi {\nabla^2 \pi_{\text{conf}} }
  %////////////////
  +3 c_s^2 \mathcal{H} (1+w)\pi_{\text{conf}} {\nabla^2 \pi_{\text{conf}} }
              \nonumber
   \\
    &
      %////////////////
        -   (1-c_s^2) { \color{blue}(\mathcal{H} \pi_{con}+ \pi'_{con})} \nabla^2 {\pi_{\text{conf}} } 
        %//////////////// 
             +c_s^2 {\nabla  \Phi . \nabla \pi_{\text{conf}} }
   %//////////////// 
        -(2 c_s^2-1) {\nabla  \Psi . \nabla \pi_{\text{conf}} }  
   %//////////////// 
 +\frac{\mathcal{H}} {2 } \Big(2+3w+c_s^2  \Big){\nabla  \pi_{\text{conf}} . \nabla \pi_{\text{conf}} } 
                                 \nonumber
   \\
    &
    %//////////////// 
     -2   (1-c_s^2){\nabla  \pi_{\text{conf}} . \nabla {\color{blue} {(\mathcal{H} \pi_{con}+ \pi'_{con})} }}       =0
   %  -(\frac{1}{c_s^2}-1) \nabla^2 \Psi \pi+ 2 \nabla^2 \Phi \pi - 3 H (1+w) \pi \nabla^2 \pi  + (\frac{1}{c_s^2}-1) \pi \nabla^2 \dot{\pi}   \nonumber \\ &+ (2-\frac{1}{c_s^2})\nabla \Psi \nabla \pi - \nabla \Phi \nabla \pi -\frac{H} {2 c_s^2} \Big(2+3w+c_s^2  \Big) \nabla \pi \nabla \pi =0
  \end{align} 
%\end{empheq}

%\begin{empheq}[box=\tcbhighmath]{equation}
% \begin{align} 
% &{\color{blue}\mathcal{H}' \pi_{con}+\mathcal{H} \pi'_{con} +\pi''_{con} }- \mathcal{H} \Big (1+ 3w \Big)({\color{blue} \mathcal{H} \pi_{con}+ \pi'_{con}}) -3 { c_s^2 \mathcal{H}}\Big( 1- \frac{w}{c_s^2} \Big )\Psi - \, {\Psi'}- 3 c_s^2  \,{\Phi'} 
%           \nonumber
%   \\
%    &
%  -3  c_s^2 \Big({-\mathcal{H}^2 + \mathcal{H}'} \Big) \pi_{\text{conf}} 
% - c_s^2 {\nabla^2 \pi_{\text{conf}} }
%    % Second order terms
%     -2 c_s^2  \Phi  {\nabla^2 \pi_{\text{conf}} }  
%  %//////////////// 
%  +   (1-c_s^2)  \Psi {\nabla^2 \pi_{\text{conf}} }
%  %////////////////
%  +3 c_s^2 \mathcal{H} (1+w)\pi_{\text{conf}} {\nabla^2 \pi_{\text{conf}} }
%      %////////////////
%                                      \nonumber
%   \\
%    &
%        -   (1-c_s^2) { \pi_{\text{conf}}' \nabla^2 {\pi_{\text{conf}} }} 
%        %//////////////// 
%             +c_s^2 {\nabla  \Phi . \nabla \pi_{\text{conf}} }
%   %//////////////// 
%        -(2 c_s^2-1) {\nabla  \Psi . \nabla \pi_{\text{conf}} }  
%   %//////////////// 
% +\frac{\mathcal{H}} {2 } \Big(2+3w+c_s^2  \Big){\nabla  \pi_{\text{conf}} . \nabla \pi_{\text{conf}} } 
%                                 \nonumber
%   \\
%    &
%    %//////////////// 
%     -2   (1-c_s^2){\nabla  \pi_{\text{conf}} . \nabla {\pi_{\text{conf}}'} }       =0
%   %  -(\frac{1}{c_s^2}-1) \nabla^2 \Psi \pi+ 2 \nabla^2 \Phi \pi - 3 H (1+w) \pi \nabla^2 \pi  + (\frac{1}{c_s^2}-1) \pi \nabla^2 \dot{\pi}   \nonumber \\ &+ (2-\frac{1}{c_s^2})\nabla \Psi \nabla \pi - \nabla \Phi \nabla \pi -\frac{H} {2 c_s^2} \Big(2+3w+c_s^2  \Big) \nabla \pi \nabla \pi =0
%  \end{align} 
%\end{empheq}
%\begin{empheq}[box=\tcbhighmath]{equation}
 \begin{align} 
 &{\color{blue}\pi''_{conf} +\mathcal{H}(1- 3w) \pi'_{conf} } -3 { c_s^2 \mathcal{H}}\Big( 1- \frac{w}{c_s^2} \Big )\Psi - \, {\Psi'}- 3 c_s^2  \,{\Phi'} + {\color{blue}
 \Big( 3\mathcal{H}^2 (c_s^2 -w) + \mathcal{H}' (1-3c_s^2)\Big) \pi_{\text{conf}} }
           \nonumber
   \\
    &
 - c_s^2 {\nabla^2 \pi_{\text{conf}} }
    % Second order terms
     -2 c_s^2  \Phi  {\nabla^2 \pi_{\text{conf}} }  
  %//////////////// 
  +   (1-c_s^2)  \Psi {\nabla^2 \pi_{\text{conf}} }
  %////////////////
  +3 c_s^2 \mathcal{H} (1+w)\pi_{\text{conf}} {\nabla^2 \pi_{\text{conf}} }
      %////////////////
                                      \nonumber
   \\
    &
        -   (1-c_s^2)  { \color{blue}(\mathcal{H} \pi_{con}+ \pi'_{con}) } \nabla^2 {\pi_{\text{conf}} }
        %//////////////// 
             +c_s^2 {\nabla  \Phi . \nabla \pi_{\text{conf}} }
   %//////////////// 
        -(2 c_s^2-1) {\nabla  \Psi . \nabla \pi_{\text{conf}} }  
   %//////////////// 
                                    \nonumber
   \\
    &
 +\frac{\mathcal{H}} {2 } \Big(2+3w+c_s^2  \Big){\nabla  \pi_{\text{conf}} . \nabla \pi_{\text{conf}} } 
    %//////////////// 
     -2   (1-c_s^2){\nabla  \pi_{\text{conf}} . {\color{blue}  \nabla {  (\mathcal{H} \pi_{con}+ \pi'_{con})   }}}     =0
   %  -(\frac{1}{c_s^2}-1) \nabla^2 \Psi \pi+ 2 \nabla^2 \Phi \pi - 3 H (1+w) \pi \nabla^2 \pi  + (\frac{1}{c_s^2}-1) \pi \nabla^2 \dot{\pi}   \nonumber \\ &+ (2-\frac{1}{c_s^2})\nabla \Psi \nabla \pi - \nabla \Phi \nabla \pi -\frac{H} {2 c_s^2} \Big(2+3w+c_s^2  \Big) \nabla \pi \nabla \pi =0
  \end{align} 
%\end{empheq}
%\begin{empheq}[box=\tcbhighmath]{equation}
 \begin{align} 
 &{ \pi''_{con} +\mathcal{H}(1- 3w) \pi'_{con} } +3 {  \mathcal{H}}\Big( -c_s^2+ {w} \Big )\Psi - \, {\Psi'}- 3 c_s^2  \,{\Phi'} + {
 \Big( 3\mathcal{H}^2 (c_s^2 -w) + \mathcal{H}' (1-3c_s^2)\Big) \pi_{\text{conf}} }
           \nonumber
   \\
    &
 - c_s^2 {\nabla^2 \pi_{\text{conf}} }
    % Second order terms
     -2 c_s^2  \Phi  {\nabla^2 \pi_{\text{conf}} }  
  %//////////////// 
  +   (1-c_s^2)  \Psi {\nabla^2 \pi_{\text{conf}} }
  %////////////////
  +3 c_s^2 \mathcal{H} (1+w)\pi_{\text{conf}} {\nabla^2 \pi_{\text{conf}} }
      %////////////////
                                      \nonumber
   \\
    &
        -   (1-c_s^2)  { (\mathcal{H} \pi_{con}+ \pi'_{con}) } \nabla^2 {\pi_{\text{conf}} }
        %//////////////// 
             +c_s^2 {\nabla  \Phi . \nabla \pi_{\text{conf}} }
   %//////////////// 
        -(2 c_s^2-1) {\nabla  \Psi . \nabla \pi_{\text{conf}} }  
   %//////////////// 
                                    \nonumber
   \\
    &
 +\frac{\mathcal{H}} {2 } \Big(2+3w+c_s^2  \Big){\nabla  \pi_{\text{conf}} . \nabla \pi_{\text{conf}} } 
    %//////////////// 
     -2   (1-c_s^2){\nabla  \pi_{\text{conf}} . {  \nabla {  (\mathcal{H} \pi_{con}+ \pi'_{con})   }}}     =0
   %  -(\frac{1}{c_s^2}-1) \nabla^2 \Psi \pi+ 2 \nabla^2 \Phi \pi - 3 H (1+w) \pi \nabla^2 \pi  + (\frac{1}{c_s^2}-1) \pi \nabla^2 \dot{\pi}   \nonumber \\ &+ (2-\frac{1}{c_s^2})\nabla \Psi \nabla \pi - \nabla \Phi \nabla \pi -\frac{H} {2 c_s^2} \Big(2+3w+c_s^2  \Big) \nabla \pi \nabla \pi =0
  \end{align} 
%\end{empheq}
If we drop the "conf" from the fields but remembering that the fields are perturbation in constant conformal time hypersurfaces, we have,
%\begin{empheq}[box=\tcbhighmath]{equation}
 \begin{align} 
 &{ \pi''+\mathcal{H}(1- 3w) \pi' } +3 {  \mathcal{H}}\Big( -c_s^2+ {w} \Big )\Psi - \, {\Psi'}- 3 c_s^2  \,{\Phi'} + {
 \Big( 3\mathcal{H}^2 (c_s^2 -w) + \mathcal{H}' (1-3c_s^2)\Big) \pi }
           \nonumber
   \\
    &
 - c_s^2 {\nabla^2 \pi }
    % Second order terms
     -2 c_s^2  \Phi  {\nabla^2 \pi }  
  %//////////////// 
  +   (1-c_s^2)  \Psi {\nabla^2 \pi}
  %////////////////
  +3 c_s^2 \mathcal{H} (1+w)\pi {\nabla^2 \pi }
      %////////////////
        -   (1-c_s^2)  { (\mathcal{H} \pi+ \pi') } \nabla^2 {\pi }
                                       \nonumber
   \\
    &
        %//////////////// 
             +c_s^2 {\nabla  \Phi . \nabla \pi}
   %//////////////// 
        -(2 c_s^2-1) {\nabla  \Psi . \nabla \pi }  
   %//////////////// 
 +\frac{\mathcal{H}} {2 } \Big(2+3w+c_s^2  \Big){\nabla  \pi . \nabla \pi} 
    %//////////////// 
     -2   (1-c_s^2){\nabla  \pi . {  \nabla {  (\mathcal{H} \pi+ \pi')   }}}     =0
  \end{align} 
%\end{empheq}
%
The linear equation which is comparable with class is as following;
%\end{empheq}
%\begin{empheq}[box=\tcbhighmath]{equation}
 \begin{align} 
 &{ \pi''_{con} +\mathcal{H}(1- 3w) \pi'_{con} } +3 {  \mathcal{H}}\Big( -c_s^2+ {w} \Big )\Psi - \, {\Psi'}- 3 c_s^2  \,{\Phi'} + {
 \Big( 3\mathcal{H}^2 (c_s^2 -w) + \mathcal{H}' (1-3c_s^2)\Big) \pi_{\text{conf}} }
           \nonumber
   \\
    &
 - c_s^2 {\nabla^2 \pi_{\text{conf}} } =0
    % Second order terms==0
  \end{align} 
%\end{empheq}
%%%%%%%%%%%%%%%%%%%%%
The followed expression is needed to be the same as EFT papers equation with using the fact that they write the equation for perturbation on constant physical time hypersurfaces. \\
In order to compare the result with other results (ex. equation 113 of https://arxiv.org/pdf/1411.3712.pdf) we write down the coefficients in the paper and try to compare it with our result: \\
The coefficient of $\ddot{\pi}$ in the paper is:
\be
H^2 \alpha_K  \ddot{\pi}_{phys}= \frac{ H^2 \Omega (1+w)}{c_s^2} \ddot{\pi}_{phys}
\ee
by multiplying to $\frac{ c_s^2}{ a (1+w) \Omega H^2}$ we get 1! We divided by "a" since it is introduced by $\pi_{\text{conf}}$.  Moreover it is important to note that $\ddot{\pi}= \frac{-\mathcal{H} \pi' + \pi''}{a^2}$, so in our expression we also multiply to $a^2$ to get 1 for coefficient of $\pi''$ and $-\mathcal{H} \pi'$. So at the end we multiply all terms by $\frac{a c_s^2}{ (1+w) \Omega H^2}$. In sum,
\be
\frac{ H^2 \Omega (1+w)}{c_s^2} \ddot{\pi}_{phys} \times \frac{ a c_s^2}{  (1+w) \Omega H^2}  =  \pi''_{\text{conf}} - \mathcal{H} \pi'_{\text{conf}}
\ee
The next term is $\dot{\pi}$,
\be
 H^2 \alpha_k (3H +2 \frac{\dot{H}}{H} + \frac{\dot{\alpha_k}}{\alpha_k} ) \dot{\pi}_{phys} = - \frac{3 H^3 w (1+w) \Omega  }{c_s^2} \dot{\pi}_{phys} \xrightarrow{  \times \frac{ ac_s^2}{(1+w) \Omega H^2}} -3 w \mathcal{H} \pi'_{\text{conf}}
\ee
Where we have used $\frac{\dot{\alpha_k}}{\alpha_k} = \frac{\dot{\Omega}}{\Omega}= -3 H (1+w) -2 \frac{\dot{H}}{H}$ and $\dot{H}+ \frac{\rho_m + P_m}{2 M^2} = - \frac{2 \bar{X} P_{,X}}{2 M^2}=\frac{- \bar{\rho}(1+w)}{2 M^2} $.  \\
%Again by multiplying to $  \frac{ ac_s^2}{(1+w) \Omega H^2}$ we get 
%$-\frac{3 w H}{a}$ which is the coefficient of $\dot{\pi}_{conf}$ for us it would be $-{3 \mathcal{H} ^2w}$ which is coefficient of ${\pi}'_{conf}$ 
$\pi$ coefficient:
\be
6 \dot{H}(\dot{H} + \frac{\rho_m + P_m}{2M^2}) \pi_{phys} = -3 a^2 c_s^2 \dot{H} \pi_{\text{conf}} = -3 c_s^2(-\mathcal{H}^2 + \mathcal{H}') \pi_{\text{conf}}
\ee
Note that $\Omega= \frac{\rho}{3 M_{pl}^2 H^2}=  \frac{\rho}{M^2 H^2}$. \\
The coefficient of $\nabla^2 \pi$ is :
\be
2 (\dot{H} + \frac{\rho_m + P_m}{2M^2}) \frac{\nabla^2 \pi_{phys}}{a^2} \xrightarrow{  \times \frac{ ac_s^2}{(1+w) \Omega H^2}} -c_s^2  \nabla^2 \pi_{\text{conf}}
\ee
The coefficient of $\dot{\Psi}$, (note that $\Psi$ and $\Phi$ are not the same in our convention and the paper!)
\be
- H^2 \alpha_k \dot{\Psi} = -\frac{\Omega (1+w) }{c_s^2} \Psi'/a \xrightarrow{  \times \frac{ ac_s^2}{(1+w) \Omega H^2}} -\Psi'
\ee
The coefficient of $\dot{\Phi}$, ($\dot{\Psi}$ in the paper)
\be
6(\dot{H} + \frac{\rho_m + P_m}{2M^2}) \dot{\Phi} = \frac{ -3 \bar{\rho}(1+w)}{M^2}  {\Phi'}/a\xrightarrow{  \times \frac{ ac_s^2}{(1+w) \Omega H^2}}  -3 \, c_s^2 \Phi'
\ee
The coefficient of ${\Psi}$, (${\Phi}$ in the paper)
\begin{align}
 & \left [ 6(\dot{H} + \frac{\rho_m + P_m}{2M^2}) + H \alpha_K \left ( -3  H -2 \frac{\dot{H}}{H} - \frac{\dot{\alpha}_K}{\alpha_K } \right ) \right ] H \Psi= \Big( -3 H^2 \Omega (1+w) + \frac{\Omega (1+w)}{c_s^2} 3 H^2 w)\Big ) \mathcal{H}/a  \Psi \nonumber. \\ &
 \xrightarrow{  \times \frac{ ac_s^2}{(1+w) \Omega H^2}} -3 \mathcal{H} (c_s^2 -w )\Psi
\end{align}
So we get exactly the same first order equations as the references! \\
\subsection{Numeric olver}
Note that ${H'}$ can be determined by all the matter contents of the universe not by k-essence alone, the continuity equation for k-essence or matter gives the dynamics of density. \\
The field equation is:
we take $d \tau=\tau_{n+1}-\tau_n $ and $x_{i,j,k} $ as lattice point. We solve the differential equation numerically as following;
\be
\pi_v= {\pi}'
\ee
\be
\pi^{n}= \pi ^{n-1}+\pi_v ^{n-\frac{1}{2}} d \tau
\ee
\be \label{eq3}
\pi_v ^{n+\frac{1}{2}}=\pi_v ^{n-\frac{1}{2}} + {\pi''} ^{(n)}  d \tau
\ee

We define the laplacian in code as following,
\begin{align}
& \nabla^2 \pi =-\frac{\pi^{n}_{i-1,j,k}+\pi^{n}_{i+1,j,k} +\pi^{n}_{i,j-1,k} +\pi^{n}_{i,j+1,k}+\pi^{n}_{i,j,k-1}+\pi^{n}_{i,j,k+1} -6 \pi^{n}_{i,j,k}  }{ a^2 dx^2}  
\end{align}
Moreover in order to get scalar in the vertices the derivatives like $\nabla \pi . \nabla \pi $ should be defined symmetric .
So we can rewrite the equation \ref{eq3} as below;
\begin{align} 
 &{\color{blue}\pi''_{con} +\mathcal{H}(1- 3w) \pi'_{con} } -3 { c_s^2 \mathcal{H}}\Big( 1- \frac{w}{c_s^2} \Big )\Psi - \, {\Psi'}- 3 c_s^2  \,{\Phi'} + {\color{blue}
 \Big( 3\mathcal{H}^2 (c_s^2 -w) + \mathcal{H}' (1-3c_s^2)\Big) \pi_{\text{conf}} }
           \nonumber
   \\
    &
 - c_s^2 {\nabla^2 \pi_{\text{conf}} }
    % Second order terms
     -2 c_s^2  \Phi  {\nabla^2 \pi_{\text{conf}} }  
  %//////////////// 
  +   (1-c_s^2)  \Psi {\nabla^2 \pi_{\text{conf}} }
  %////////////////
  +3 c_s^2 \mathcal{H} (1+w)\pi_{\text{conf}} {\nabla^2 \pi_{\text{conf}} }
      %////////////////
                                      \nonumber
   \\
    &
        -   (1-c_s^2)  { \color{blue}(\mathcal{H} \pi_{con}+ \pi'_{con}) } \nabla^2 {\pi_{\text{conf}} }
        %//////////////// 
             +c_s^2 {\nabla  \Phi . \nabla \pi_{\text{conf}} }
   %//////////////// 
        -(2 c_s^2-1) {\nabla  \Psi . \nabla \pi_{\text{conf}} }  
   %//////////////// 
                                    \nonumber
   \\
    &
 +\frac{\mathcal{H}} {2 } \Big(2+3w+c_s^2  \Big){\nabla  \pi_{\text{conf}} . \nabla \pi_{\text{conf}} } 
    %//////////////// 
     -2   (1-c_s^2){\nabla  \pi_{\text{conf}} . {\color{blue}  \nabla {  (\mathcal{H} \pi_{con}+ \pi'_{con})   }}}     =0
   %  -(\frac{1}{c_s^2}-1) \nabla^2 \Psi \pi+ 2 \nabla^2 \Phi \pi - 3 H (1+w) \pi \nabla^2 \pi  + (\frac{1}{c_s^2}-1) \pi \nabla^2 \dot{\pi}   \nonumber \\ &+ (2-\frac{1}{c_s^2})\nabla \Psi \nabla \pi - \nabla \Phi \nabla \pi -\frac{H} {2 c_s^2} \Big(2+3w+c_s^2  \Big) \nabla \pi \nabla \pi =0
  \end{align} 
\begin{align} 
 &\pi_v ^{n+\frac{1}{2}}=\pi_v ^{n-\frac{1}{2}} - d \tau \Big [ \mathcal{H}^{(n)} (1-3w)\frac{(\pi_{v  \; {i,j,k}}^{n+\frac{1}{2}} +\pi_{v \; {i,j,k}}^{n-\frac{1}{2}} )}{2} -3 {c_s^2 \mathcal{H}^{(n)}}\Big( 1- \frac{w}{c_s^2} \Big )\Psi^{(n) }
 -  \frac{{\Psi}^{(n)}-{\Psi}^{(n-1)} }{d \tau}
    \nonumber
     \\
      &
      - 3c_s^2  \, \frac{{\Phi}^{(n)}-{\Phi}^{(n-1)} }{d \tau}    
   +\Big( 3\mathcal{H}^2 (c_s^2 -w) + \mathcal{H}' (1-3c_s^2) \Big) \pi^{(n)}   - c_s^2 {\nabla^2 \pi ^{(n)}}  
   %
   + (1-c_s^2)\Psi^{(n)} {\nabla^{2} \pi^{(n)}  }    
          \nonumber
     \\
      &
      %___________
    - 2 c_s^2  \Phi ^{(n)}  {\nabla^2 \pi^{(n)}}
          %___________
     + {3 c_s^2  \mathcal{H}^{(n)} (1+w) }\pi^{(n)} {\nabla^2 \pi^{(n)} }   
               %___________
     -  (1-c_s^2)
 \Big[ \frac{(\pi_{v  \; {i,j,k}}^{n+\frac{1}{2}} +\pi_{v \; {i,j,k}}^{n-\frac{1}{2}} )}{2}  + \mathcal{H} \pi^{(n)} \Big] {\nabla^2  \pi^{(n)}} 
        \nonumber
     \\
       &
               %___________
    - (2 c_s^2-1) {\nabla  \Psi^{(n)}  . \nabla \pi ^{(n)} }
            %___________
    + c_s^2 {\nabla  \Phi ^{(n)} . \nabla \pi^{(n)}  }                %___________
              +\frac{\mathcal{H}^{(n)}} {2 } \Big(2+3w+c_s^2  \Big) \,{\nabla  \pi^{(n)} . \nabla \pi^{(n)} }  
                      %___________
                           \nonumber
     \\
       &
          -2(1-c_s^2) \nabla  \pi^{(n)} .  \Big( \frac{{ \nabla  ( \pi_{v  \; {i,j,k}}^{n+\frac{1}{2}} +\pi_{v \; {i,j,k}}^{n-\frac{1}{2}} ) }  } {2}  + \mathcal{H}  \nabla\pi^{(n)} \Big) 
    \Big]
\end{align}
After some simplification we get,
\begin{align} 
%
 &\pi_v ^{n+\frac{1}{2}}= \frac{1}{1+ d\tau   \mathcal{H}^{(n)}  (1-3w) /2 - d\tau (1-c_s^2) \nabla^2 \pi^{(n)}/2} \times \Bigg[ \pi_v ^{n-\frac{1}{2}} - d \tau \Big [(1- 3w)\mathcal{H}^{(n)}   \frac{\pi_{v \; {i,j,k}}^{n-\frac{1}{2}} }{2}
     \nonumber
     \\
      &
  -3 { c_s^2 \mathcal{H}^{(n)}}\Big( 1- \frac{w}{c_s^2} \Big )\Psi^{(n) }
 - \, \frac{{\Psi}^{(n)}-{\Psi}^{(n-1)} }{d \tau}
      - 3  c_s^2  \, \frac{{\Phi}^{(n)}-{\Phi}^{(n-1)} }{d \tau}    
   +\Big( 3\mathcal{H}^2 (c_s^2 -w) + \mathcal{H}' (1-3c_s^2) \Big)\pi^{(n)}  
             \nonumber
     \\
      &
       - c_s^2 {\nabla^2 \pi ^{(n)}}  
   %
   + (1-c_s^2)\Psi^{(n)} {\nabla^{2} \pi^{(n)}  }    
      %___________
    - 2 c_s^2  \Phi ^{(n)}  {\nabla^2 \pi^{(n)}}
          %___________
     + {3 c_s^2  \mathcal{H}^{(n)} (1+w) }\pi^{(n)} {\nabla^2 \pi^{(n)} }   
          \nonumber
     \\
      &
               %___________
     -  (1-c_s^2)
 \Big( \frac{\pi_{v \; {i,j,k}}^{n-\frac{1}{2}} }{2} +\mathcal{H}  \pi^{(n)}  \Big) {\nabla^2  \pi^{(n)}} 
               %___________
    - (2 c_s^2-1) {\nabla  \Psi^{(n)}  . \nabla \pi ^{(n)} }
            %___________
    + c_s^2 {\nabla  \Phi ^{(n)} . \nabla \pi^{(n)}  }  
                              \nonumber
     \\
       & 
            %___________
              +\frac{\mathcal{H}^{(n)}} {2 } \Big(2+3w+c_s^2  \Big) \,{\nabla  \pi^{(n)} . \nabla \pi^{(n)} }  
                      %___________                    
         -2(1-c_s^2) \Big( \frac{{ \nabla  ( \pi_{v  \; {i,j,k}}^{n+\frac{1}{2}} +\pi_{v \; {i,j,k}}^{n-\frac{1}{2}} ) }  } {2}  + \mathcal{H}  \nabla\pi^{(n)} \Big) 
    \Big] \Bigg]
\end{align}
Since we have $\nabla \pi_v $ in the equation we use predictor corrector method as following,\\
In the first step we predict that the term $\nabla \pi \nabla \pi_v$ is small so we approximate $\nabla \pi_v ^{(n)}$ with $\nabla \pi_v^{n-\frac{1}{2}}$ then we calculate $\pi_v^{n+\frac{1}{2}}$ according to the formula with the guess, then we use the new $\pi_v^{n+\frac{1}{2}}$ into the full equations to correct $\pi_v^{n+\frac{1}{2}}$ . \\
 We have taken $\pi_{v  \; {i,j,k}}^{n} =\frac{(\pi_{v  \; {i,j,k}}^{n+\frac{1}{2}} +\pi_{v \; {i,j,k}}^{n-\frac{1}{2}} )}{2} $. Then we need to calculate $\mathcal{H}'$, ${\Psi}'$ and  ${\Phi}'$ in each loop, to calculate ${\Psi}'$ we save two $\Psi$ in each loop. \\
 On the other hand we have $\mathcal{H}'$ according to the Friedman equation, where we try to save $\mathcal{H}'$ from $a''$  and $\mathcal{H}$.
 %***************************
 %***************************
  %***************************
 %***************************
 %***************************
 %***************************
 \section{Stress tensor {\color{red}check again and make sure everything is ok } }
The most general action for a scalar field coupled to Einstein gravity is;
\be
S=\frac{1}{16 \pi G} \int \sqrt{-g} R d^4 x + \int \sqrt{-g} P (X, \varphi) d^4 x
\ee
The metric convention is $(-,+,+,+)$ and $X=- \frac{1}{2}  g^{\mu \nu}\partial _\mu \phi \partial_\nu \phi$. We  assume the scalar action as a matter sector which contributes to stress energy tensor,
\be
T^{\mu\nu}\equiv \dfrac {+2}{\sqrt {-g}}\dfrac {\delta \mathcal{ L}}{\delta g_{\mu\nu}}=\dfrac {2}{\sqrt {-g}}\dfrac {\delta \left[ \sqrt {-g}P\left( X,\varphi \right) \right] }{\delta g_{\mu\nu}}
=
\dfrac {2}{\sqrt {-g}}[- \dfrac {1 }{2 \sqrt{-g}} \frac{\delta g}{\delta g_{\mu\nu}}P\left( X,\varphi\right) +\dfrac {\delta P\left( X,\varphi\right) }{\delta g_{\mu\nu}}\sqrt {-g}]
\ee
According to appendix \ref{A1},
\be
\dfrac {\delta \sqrt {-g}}{\delta g_{\mu\nu}}=\dfrac {-1}{2\sqrt {-g}}\dfrac {\delta g}{\delta g_{\mu\nu}}=\dfrac {-1}{2\sqrt {-g}}\dfrac {g\delta g_{\mu\nu}g^{\mu\nu}}{\delta g_{\mu\nu}}=\dfrac {\sqrt {-g}}{2}g^{\mu\nu}
\ee
\be
T^{\mu\nu}=2\dfrac {\delta P\left( X,\varphi\right) }{\delta g_{\mu\nu}} + g^{\mu\nu}P\left( X,\varphi\right)
\ee
\be
T_{\rho \sigma}=g_{\mu \rho} g_{\nu \sigma} T^{\mu \nu}= \Big[ 2 g_{\mu \rho} g_{\nu \sigma}  \dfrac {\delta P\left( X,\varphi\right) }{-g_{\mu \rho'} g_{\nu \sigma'}  \delta g ^{\sigma' \rho'}} + g_{\mu \rho} g_{\nu \sigma}  g^{\mu\nu}P\left( X,\varphi\right) \Big]= -2\dfrac {\delta P \left( X,\varphi\right) }{\delta g^{\rho \sigma}}+g_{\rho \sigma}P\left( X,\varphi\right)
\ee
Where we have used $\delta g_{\mu \nu}= - g_{\mu \rho} g_{\nu \sigma} \delta g^{\rho \sigma}$.
\begin{align}
X=-\dfrac {1}{2}g^{\mu\nu}\partial_{\mu}\varphi\partial_{\nu}\varphi \longrightarrow  \delta X=-\dfrac {1}{2}\delta g^{\mu\nu}\partial_{\mu}\varphi\partial_{\nu}\varphi-\dfrac {1}{2}g^{\mu\nu}\partial_{\mu}\delta \varphi\partial_{\nu}\varphi-\dfrac {1}{2}g^{\mu\nu}\partial_{\mu}\varphi\partial_{\nu}\delta\varphi
\end{align}
so,
\be
\dfrac {\partial X}{\partial g^{\mu\nu}}=-\dfrac {\partial_{\mu}\varphi\partial_{\nu}\varphi}{2}
\ee
\be
\dfrac {\delta P}{\delta g^{\mu\nu}}=\dfrac {\partial P}{\partial X}\dfrac {\partial X}{\partial g^{\mu\nu}}+ \cancel{\dfrac {\partial P}{\partial\varphi}\dfrac {\partial\varphi}{\partial g^{\mu\nu}}}=\dfrac {\partial P}{\partial X}\dfrac {\partial X}{\partial g^{\mu\nu}}=-\dfrac {\partial_{\mu}\varphi\partial_{\nu}\varphi}{2}P_{,X}
\ee
\be
T_{\mu\nu}=g_{\mu\nu}P\left( X,\varphi\right) +P_{,X}\partial_{\mu}\varphi\partial_{v}\varphi \; , \;
T_{\mu\nu}=\left( \rho+p\right) u_{\mu}u_{\nu}+p g_{\mu\nu}
\ee
\be
u_{\mu}=\dfrac {\partial_{\mu}\varphi}{\sqrt {-\partial_{\mu}\varphi\partial^{\mu}\varphi}}\rightarrow u_{\mu}=\dfrac {\partial_{\mu}\varphi}{\sqrt {2X}} , \rho=2XP_{,X}-P \; , \; p=P  \label{eq10}
\ee
We assume that field is a monotonic function of time in background which is perturbed in each constant physical time hypersurfaces. \\
It is important to notice that in the previous chapter $\pi$ was the perturbation in constant physical time hypersurfaces, so to be consistent,  although at the we want to express everything in terms of conformal time in Gevolution but we keep $\pi$ as perturbation in physical time.
\be
\varphi_{0}\left( \tau+\pi\left( \tau,\overrightarrow {x}\right) \right) =\varphi_{0}\left( \tau \right) +\dfrac {\partial\varphi_{0}}{\partial  \tau }\pi+\dfrac {\partial^{2}\varphi_{0}}{2\partial^{2} \tau}\pi^{2}+\ldots
\ee
We can choose $\varphi_0(\tau)=\tau$ for simplicity, using the following ansatz for the metric,
\be
g_{\mu\nu}=a(\tau)^2 \Big [-e^{2\Psi}d\tau^{2}+ e^{-2\Phi}dr^{2} \Big]
\ee
where $\tau$ is the conformal time.
\be
\delta g^{(1)}_ { 00}=-2\, a^2 \Psi \, \; \; , 
\delta g^{(1)}_{ij}= -2 a^{2} \Phi \delta_{ij}
\ee
Where $\delta g^{(1)}_ { 00}$ means the first order metric in pertubations.  The inverse of metric is defined as following,
\be
g^{\mu\nu}=\frac{1}{a^2} \Big [-e^{-2\Psi}d\tau^{2}+e^{2\Phi}dr^{2}  \Big ]
\ee
\be
\delta g_{(1)}^{00}=+\frac{2\Psi}{a^2} \, \; \; , 
\delta g_{(1)}^{ij}= + \frac{2\Phi \delta^{ij} }{a^2}
\ee
We have,
\be
X=\dfrac {-1}{2}g^{\mu\nu} (\tau + \pi,x)\partial_{\mu}\left( \tau+\pi\right) \partial_{\nu}\left( \tau+\pi\right) 
\ee
We expand X perturbatively,
\be
X=\overline {X}+\delta X_{1}+ \delta X_{2}+\ldots
\ee
\be
\overline {X}=-\dfrac {1}{2}\bar{g}^{00}\partial_{0} \tau \partial_{0} \tau=+\dfrac {1}{2 a^2}\\
\ee
\be
\delta X_{1}={\bar{X}'} \pi-\dfrac {1}{2}\delta g_{(1)}^{00}\partial_{0} \tau \partial_{0} \tau-\dfrac {1}{2} \bar{g}^{00}\partial_{0} \tau \partial_{0}\pi-\dfrac {1}{2} \bar{g}^{00}\partial_{0}\pi\partial_{0} \tau-\dfrac {1}{2}\bar{g}^{ij}\partial_{i}\pi\partial_{j}\pi
\ee
where ${\bar{X}'} \pi =-\dfrac {1}{2}\bar{g}^{00'} \pi \partial_{0} \tau \partial_{0} \tau = -\frac{\mathcal{H}}{a^2} \pi $.
\be
 \delta X_{1}=\frac{1}{a^2} \Big [- \mathcal{H} \pi-\Psi+{\pi'}- \frac{(\vec{\nabla} \pi)^2}{2 }  +O\left( \varepsilon^{2}\right) \Big]
\ee
We do not need to calculate $X_2$ since the energy momentum constraint adds at most one spatial derivative which does not add the second order terms to first order. So
\bea
 & P\left(\varphi_0( \tau+\pi) ={\varphi_0}+ {{\varphi_0}}' \pi,\overline{X}+ \delta X_1+\delta X_2 \right)  = \overline{P}\left( \varphi( \tau),\overline {X}\right) 
+ {\dfrac {\partial\overline {P}}{\partial \varphi_0}} \pi+\dfrac {1}{2} {\dfrac {\partial^{2}\overline {P}}{\partial \varphi_0^2}}\pi^{2}
+
\nonumber \\ &
\dfrac {\partial\overline {P}}{\partial\overline {X}}\delta X_1+\dfrac {1}{2}\dfrac {\partial P}{\partial X^{2}}\delta X_1^{2}+\dfrac {\partial^2 P}{\partial X \partial \varphi_0}\delta X_1 \pi +\dfrac {1}{2}\dfrac {\partial P}{\partial X^{2}}\delta X_2 + \mathcal{O}(\epsilon^3).
\eea
\\
 Note that here $\pi$ is perturbation in $\tau$ conformal time.%The term $\dfrac {\partial\overline {P}}{\partial \tau}$ becomes  $\dfrac {\partial\overline {P}}{\partial \tau}=\dfrac {\partial\overline {P}}{\partial  {\varphi}} \dfrac{\partial {\varphi}}{\partial \tau}+ \dfrac {\partial\overline {P}}{\partial  {X}} \dfrac{\partial {X}}{\partial \tau} =P_{,\varphi} \dot{\bar{\varphi}} + \bar{P}_{\bar{X}}$. Because $\varphi$ and $\partial_{\mu} \varphi$ are independent variables not function of $\tau$. \\
The adiabatic sound speed ({\color{red}why?!}) is defined as below,
\be
%c^{2}_{s}\equiv \frac{\delta P}{\delta \rho} =\dfrac {\bar{P}_{,X} \delta X + \bar{P}_{,\varphi} \delta \varphi}{\bar{\rho}_{,X} \delta X  +\bar{\rho}_{,\varphi} \delta  \varphi}=\dfrac {\bar{P}_{,X}}{\bar{\rho}_{,X}}=\dfrac {\bar{P}_{,X}}{\bar{P}_{,X}+2\bar{X}\bar{P}_{,XX}} 
c^{2}_{s}\equiv \dfrac {\bar{P}_{,X}}{\bar{\rho}_{,X}}=\dfrac {\bar{P}_{,X}}{\bar{P}_{,X}+2\bar{X}\bar{P}_{,XX}} 
\ee
and $\Omega$
\be
\Omega= \frac{\bar{\rho}}{3 M_{pl}^2 H^2}= \frac{ a^2 \bar{\rho}}{3 M_{pl}^2 \mathcal{H}^2}=\frac{{2\bar{X} \bar{P}_{,X}-\bar{P}}}{3 M_{pl}^2 H^2} \label{22}
\ee
Where we have used $ \rho=2XP_{,X}-P$
\be
\omega=\dfrac {\overline {P}}{\overline {\rho}}=\dfrac {\overline {P}}{2\overline {X} \, \overline{P}_{,X}-\overline {P}} \label{23}
\ee
Moreover we have,
\be
\rho'=2X' P_{,X} + 2 X P_{,X \varphi} \varphi' +2 X P_{,X X} X' - P_{,\varphi} \varphi' - P_{,X} X'= (2 X P_{,X \varphi} - P_{,\varphi}) \varphi' + (P_{,X} + 2 X P_{,XX})X'
\ee
In the background level and using $\bar{X}=\frac{1}{2 a^2}$, $\bar{X}'=-\frac{\mathcal{H}}{a^2}$, $\varphi'=1$
\be
\rho' =\frac{P_{,X}'}{a^2} - P'  - (P_{,X} + \frac{ P_{,XX}}{a^2}) \frac{\mathcal{H}}{a^2}
\ee
So we can write the function $P$ and it derivative in terms of $\Omega$, $\omega$ and $c_s^2$,
\be
\bar{P}_{X}= a^2 \bar{P} (1+\frac{1}{\omega}) \; \; \; \; \;  \; \bar{P} _{,XX}=a^2  \bar{P}_{,X} \frac{1-c_s^2}{c_s^2} =a^4  \bar{P} (1+\frac{1}{\omega}) (\frac{1}{c_s^2} -1 )
\label{Pbarder}
\ee
So according to \ref{22} and \ref{23}\\
\be
\bar{P}=  3 M_{pl}^2 H^2 \Omega \, \omega = \frac{ 3 M_{pl}^2 \mathcal{H}^2 \Omega\, \omega }{a^2} \label{Pbar}
\ee
Moreover,
\be
\frac{\bar{P}'}{\bar{P}}=\frac{2 \mathcal{H}' }{\mathcal{H}} + \frac{\Omega'}{\Omega} + \cancel{\frac{w'}{w} }- 2\mathcal{H}
\ee
%Where according to continuity equation \ref{Conteqgg}, we can write,
%\be
%\frac{\bar{P}'}{\bar{P}}=-3 (1+w) \mathcal{H}- \mathcal{H}
%\ee
%where we have assumed that $w'=0$ and we have used the below relation,
%\be
% \frac{\Omega'}{\Omega}=- \frac{(1+3  w)}{2} \mathcal{H}- \frac{ \mathcal{H'}}{\mathcal{H}}
%\ee
Also we have the relation for $\bar{P}'_{,X}$ as following,
\be
\frac{\bar{P}'_{,X}}{\bar{P}_{,X}}=\frac{\bar{P}'}{\bar{P}}+2 \mathcal{H}
\ee
So we have,
\begin{align}
 -a^2 \bar{P}'  + \bar{P}'_{,X}&=-a^2 \bar{P}' + a^2 \bar{P} (1+\frac{1}{w}) (\frac{\bar{P}'}{\bar{P}}+2 \mathcal{H}) = a^2 \frac{\bar{P}'}{w} + 2 \mathcal{H} a^2 \bar{P} (1+\frac{1}{w})
 \\ \nonumber&
 = \frac{a^2 \bar{P}}{w} \Big[ 2 \frac{\mathcal{H}'}{\mathcal{H}}+ \frac{\Omega'}{\Omega} + 2 \mathcal{H} w\Big]
\end{align}
It is very important to note that by $\bar{P}(\varphi,X)'$ we mean $\bar{P}(\varphi,X)_{,\varphi}$ and $\bar{P}_{,\tau}=\bar{P}_{,\varphi} \varphi' + \bar{P}_{,X} X' $
The relation between Hubble and conformal Hubble is;
\be
\mathcal{H} (\tau)=\frac{1}{a(\tau) }\frac{d a(\tau)}{d \tau }= \frac{1}{a(t) } \frac{d a (t) }{d t} \frac{d t }{ d\tau}= a H(t)
\ee
and for the derivative,
\begin{align}
& \dot{H}= \frac{-\mathcal{H}^2+ \mathcal{H}'}{a^2} \nonumber \\ &
\mathcal{H}'=a^2 \Big[ H^2 + \dot{H}\Big ]
\end{align}
Now we can construct stress tensor up to first order in Gevolution's perturbation scheme,
\begin{align} \label{eqTmunuI}
T_{\mu \nu} &= P g_{\mu \nu} + P_{,X} \partial_{\mu} \varphi \partial_{\nu} \varphi
 \\
  \nonumber
   & =(\bar{g}_{\mu \nu} + \delta g^{(1)}_{\mu \nu}) (\bar{P}+\bar{P}' \pi+\bar{P}_{,X} \delta X_1) + (\bar{P}_{,X}+\bar{P}'_{,X} \pi+\bar{P}_{,XX} \delta X_1) \partial_{\mu} (\tau+ \pi) \partial_\nu (\tau+\pi)+ \ldots
\\ \nonumber & 
= \Big[ \bar{g}_{\mu \nu} \bar{P} 
+
 \bar{P}_{,X} \partial_{\mu} \tau \partial_{\nu} \tau \Big] \epsilon ^0 
+
\Big[( \bar{g}_{\mu \nu} \bar{P}'+ \bar{P}'_{,X} )\pi+\bar{g}_{\mu \nu}  \bar{P}_{,X} \delta X_1 
+
 \delta g^{(1)}_{\mu \nu} \bar{P} 
 +
  \bar{P}_{,X}  \left ( \partial_{\mu} \pi \partial_{\nu} \tau  
  +
  \partial_{\mu} \tau \partial_{\nu} \pi  \right ) 
    \nonumber \\ &
  +
   \delta X_1 \bar{P}_{,XX}   \partial_{\mu} \tau \partial_{\nu} \tau  
      +
    \bar{P}_{,X}   \partial_{\mu} \pi \partial_{\nu} \pi
    +  \delta X_1 {\color{red} \bar{P}_{,XX}  \big(  \partial_{\mu} \tau \partial_{\nu} \pi  +   \partial_{\mu} \pi \partial_{\nu} \tau  \big) }
    + \cancelto{\mathcal {O}(\epsilon^{2})}{\delta X_1 \bar{P}_{,XX}   \partial_{\mu} \pi \partial_{\nu} \pi  
   } \Big ] 
 \; \; \;+ \mathcal {O}(\epsilon^{2}) 
\end{align}
It is important to know that $\bar{g}_{\mu \nu}' \bar{P}=0$ since the metric is not a function of scalar field, so it is no affected by changing the scalar field which is a dynamical variable. 
{\color{red}The red color is the one I have tension with Filippo's calculation.}
%\newpage
\subsection{Derivative of determinant} 
Assume an invertible matrix M,
\be
\det\left( M+\delta M \right) =\det \left( M\left( 1+M^{-1}\delta M\right) \right)
\ee
where $\delta M$ is a small change in the matrix M. According to the properties of determinant $\det\left( AB\right) =\det\left( A\right) \det\left( B\right) 
$ we have,
\be
\det\left( M\left( 1+\delta M\right) \right) =\det M\det\left( I+\delta M\right) 
\ee
%According to Cayley-Hamilton theorem: \\
For a $n \times n$ matrix M, the characteristic polynomial is defined by p($\lambda$)=$\det (A- \lambda I)$=$(-1)^n \Big[ \lambda ^n +c _1 \lambda ^{n-1} + c _2 \lambda ^{n-2}  +...+c_n \Big]$  where $c_n=(-1)^n \det( A)$, $c_1=tr (A)$. So,
\be
\det\left( I+\delta M\right) =p(\lambda=1)=1^{n}+1^{n-1}tr\left( \delta M\right) +O\left( \delta M^{2} \right) 
\ee
On the other hand,
\begin{align}
\delta\det M&=\det\left( M+\delta M\right) -\det\left( M\right) =\det M\det\left( I+\delta M\right) -\det\left( M\right) \nonumber \\ &
=\det M (1+ tr(\delta M)) -\det\left( M\right)= \det M  \, tr (\delta M)
\end{align}
So for the metric we can write the same statement to get the result,
\be
\delta g =\delta \det g_{\mu \nu}=  \det (g_{\mu \nu}+ \delta g_{\mu \nu}) - \det (g_{\mu \nu}) = \det (g_{\mu \nu}) tr (\delta g_{\mu \nu}) = g \, \delta g_{\mu\nu}g^{\mu\nu}
\ee
Pay attention to the relation between $\delta g^{\mu \nu}$ and $\delta g_ {\mu \nu}$ which shows that $\delta g_{\mu \nu}$ is not a tensor!
\be
g_{\mu\nu}g^{\nu\rho}=\delta^{\rho}_{\mu} \rightarrow
\delta g_{\mu\nu}g^{\nu\rho}+g_{\mu\nu}\delta g^{\nu\rho}=0 
\ee
\be
\delta g^{\nu\rho}=-g^{\nu\sigma}\delta g_{\sigma\mu}g^{\mu\rho}
\ee

\subsection{$T_{00}$ component of energy momentum tensor}
According to previous equation $T_{00} $ component is,
\begin{align} \label{T00}
T_{00} &=
   \Big[ \bar{g}_{0 0} \bar{P} 
+
 \bar{P}_{,X} \partial_{0} \tau \partial_{0} \tau \Big] \epsilon ^0 
+
\Big[( -a^2 \bar{P}' +\bar{P}'_{,X} )\pi+\bar{g}_{0 0}  \bar{P}_{,X} \delta X_1 
+
 \delta g^{(1)}_{0 0} \bar{P} 
 +
  \bar{P}_{,X}  \left ( \partial_{0} \pi \partial_{0} \tau  
  +
  \partial_{0} \tau \partial_{0} \pi  \right ) 
    \nonumber
 \\
  &
  +
   \delta X_1 \bar{P}_{,XX}   \partial_{0} \tau \partial_{0} \tau 
   +
    \cancelto{\mathcal {O}(\epsilon^{2}) 
} { \bar{P}_{,X}  } \partial_{0} \pi \partial_{0} \pi  +   \cancelto{\mathcal {O}(\epsilon^{2})}  {2 \pi'   } \delta X_1  \bar{P}_{,XX}  \Big ]
%::::::::::::::::::::::::::::::::::::::::::::::::::::::::::::::::::::::::::::::::::::::::::::::::::::::::::::::::
%::::::::::::::::::::::::::::::::::::::::::::::::::::::::::::::::::::::::::::::::::::::::::::::::::::::::::::::::
  \nonumber
 \\
  &
  =
  [-a^2  \bar{P} 
+
 \bar{P}_{,X}  ] \epsilon ^0 
+
\Big[a^2 (- \bar{P}' +\frac{\bar{P}'_{,X}}{a^2} )\pi- a^2 \bar{P}_{,X} \delta X_1 
-
 2 a^2 \Psi \bar{P} 
 +
 2 \bar{P}_{,X}   {\pi'}
  +
    {\color{red}
   \delta X_1 \bar{P}_{,XX} }
  \Big ] 
+ \mathcal {O}(\epsilon^{2}) 
%::::::::::::::::::::::::::::::::::::::::::::::::::::::::::::::::::::::::::::::::::::::::::::::::::::::::::::::::
%::::::::::::::::::::::::::::::::::::::::::::::::::::::::::::::::::::::::::::::::::::::::::::::::::::::::::::::::
  \nonumber
 \\
  &
  =
  [-a^2 \bar{P} 
+
a^2 \bar{P}  (1+\frac{1}{w}) ] \epsilon ^0 
+
\Big[a^2 \rho' \pi
-
 2 a^2  \Psi \bar{P} 
 +
 2  a^2 \bar{P}  (1+\frac{1}{w})  {\pi'}
  +
  a^4 \bar{P}  (1+\frac{1}{w}) (\frac{1}{c_s^2}-1) \,   (\delta X_1 )
   \Big ] 
+ \mathcal {O}(2) 
%::::::::::::::::::::::::::::::::::::::::::::::::::::::::::::::::::::::::::::::::::::::::::::::::::::::::::::::::
%::::::::::::::::::::::::::::::::::::::::::::::::::::::::::::::::::::::::::::::::::::::::::::::::::::::::::::::::
  \nonumber
 \\
  &
  =
 \frac{ a^2 \bar{P}}{w} \,\epsilon ^0 
+
\frac{ a^2\bar{ P}}{w}   \Big[ -3 \mathcal{H} (1+w) \pi
-
 2   w \Psi
 +
 2  (1+w)  {\pi'}
  +
  a^2 (1+w) (\frac{1}{c_s^2}-1) \, \Big(  -\Psi+{\pi'}-  \frac{(\vec{\nabla} \pi)^2}{2} \Big)
   \Big ] \epsilon^1
%+ \mathcal {O}(\epsilon^{3/2}) 
%::::::::::::::::::::::::::::::::::::::::::::::::::::::::::::::::::::::::::::::::::::::::::::::::::::::::::::::::
%::::::::::::::::::::::::::::::::::::::::::::::::::::::::::::::::::::::::::::::::::::::::::::::::::::::::::::::::
%  \nonumber
% \\
%  &
%  =
%\frac{a^2 \bar{ P}}{w} \,\epsilon ^0 
%+
%\frac{a^2 \bar{ P}}{w}   \Big[ (2 \frac{\mathcal{H}'}{\mathcal{H}}+ \frac{\Omega'}{\Omega}  ) \pi+  a^2 (1+w) (\frac{1}{c_s^2}- 2)  \delta X_1 
%-
% 2 w \Psi
% +
% 2  (1+w)  {\pi'}
% \Big ] 
%+ \mathcal {O}(\epsilon^{2}) 
%%::::::::::::::::::::::::::::::::::::::::::::::::::::::::::::::::::::::::::::::::::::::::::::::::::::::::::::::::
%%::::::::::::::::::::::::::::::::::::::::::::::::::::::::::::::::::::::::::::::::::::::::::::::::::::::::::::::::
%  \nonumber
% \\
%  &  
%  =
%3 M_{pl}^2 \mathcal{H}^2 \Omega  \,\epsilon ^0 
%+
%3 M_{pl}^2  \mathcal{H}^2 \Omega    \Bigg[ (2 \frac{\mathcal{H}'}{\mathcal{H}}+ \frac{\Omega'}{\Omega}  ) \pi+ (1+w) (\frac{1}{c_s^2}- 2)  \Big[\mathcal{H} \pi -\Psi+{\pi'}-  \frac{(\vec{\nabla} \pi)^2}{2} \Big ]
%-
%  \nonumber
% \\
%  &
% 2 w \Psi
% +
% 2  (1+w)  {\pi'}
% \Bigg ]
%+ \mathcal {O}(\epsilon^{2})  
%%::::::::::::::::::::::::::::::::::::::::::::::::::::::::::::::::::::::::::::::::::::::::::::::::::::::::::::::::
%%::::::::::::::::::::::::::::::::::::::::::::::::::::::::::::::::::::::::::::::::::::::::::::::::::::::::::::::::
%  \nonumber
% \\
%  &
%  =
%3 M_{pl}^2  \mathcal{H}^2 \Omega \Bigg[  1 +(2 \frac{\mathcal{H}'}{\mathcal{H}}+ \frac{\Omega'}{\Omega}  ) \pi+ \Psi \Big (- (1+w) (\frac{1}{c_s^2}- 2)-2 w  \Big ) + {\pi'} \Big ( (1+w) (\frac{1}{c_s^2}- 2 )+2 (1+w)   \Big) 
%   \nonumber
% \\
%  & - \frac{(\vec{\nabla} \pi)^2}{2}  \Big ( (1+w) (\frac{1}{c_s^2}- 2 )  \Big )
% \Bigg]
 %::::::::::::::::::::::::::::::::::::::::::::::::::::::::::::::::::::::::::::::::::::::::::::::::::::::::::::::::
%::::::::::::::::::::::::::::::::::::::::::::::::::::::::::::::::::::::::::::::::::::::::::::::::::::::::::::::::
   \nonumber
 \\
  &
  =
3 M_{pl}^2  \mathcal{H}^2 \Omega \Bigg[  1-3\mathcal{H} (1+w)  \pi+ \Psi \Big (2 - \frac{1+w}{c_s^2}  \Big ) + {\pi'} \Big ( \frac{1+w}{c_s^2}   \Big)  - \frac{(\vec{\nabla} \pi)^2}{2}   (1+w) (\frac{1}{c_s^2}- 2 ) 
 \Bigg]+  \mathcal {O}(\epsilon^{2}) 
\end{align}
Where we have used \ref{Pbarder} and \ref{Pbar}.
Finaly the $T_{00}$ component is;
%\begin{empheq}[box=\mymath ]{equation*}
%If we convert the $\pi$ to be the perturbation in constant physical time hypersurfaces (not constant conformal time) we get,
%$\pi_{phys}= \pi_{conf}$
%\begin{empheq}[box=\mymath ]{equation*}
\begin{align}
T_{00}=  
3 M_{pl}^2   \mathcal{H}^2\Omega \Bigg[  1-3\mathcal{H} (1+w)  \pi+  \Psi \Big (2 - \frac{1+w}{c_s^2}  \Big ) + {\color{blue} ({\pi'}+ \mathcal{H} \pi) } \Big ( \frac{1+w}{c_s^2}   \Big)  -   \frac{(\vec{\nabla} \pi)^2}{2}    (1+w) (\frac{1}{c_s^2}- 2 ) 
 \Bigg]+  \mathcal {O}(\epsilon^{2}) 
\end{align}
%\end{empheq}
Now we compare the result with equation 147 of https://arxiv.org/pdf/1411.3712.pdf:\\
It is clear that the coefficient of $\pi$ is the same, but we should remember that we changed a sign to get this result ($\rho'$) and also $H \pi_{phys}$ = $\mathcal{H} \pi_{\text{conf}}$. \\
${\color{blue} \dot{\pi_{phys}} = \partial_t(a \pi_{conf})=(a \dot{\pi_{con}} + aH \pi_{con} -\Psi)= \pi_{con}'+\mathcal{H} \pi_{con} -\Psi}$ \\
The other terms in the paper are:
\be
 H^2 \alpha_k \mathcal{P}= \frac{\bar{\rho } (1+w)}{c_s^2}(\dot{\pi}- \Psi)=  \frac{\bar{\rho } (1+w)}{c_s^2}({\pi'}_{\text{conf}}- \Psi)
\ee
$\Psi$ here means $\Phi$ in the paper. Moreover $\dot{\pi}_{phys}=\pi_{conf}'$. The extra $2 \Psi$ term here comes from the fact that $T_{00}=g_{00} T^0_0$. \\
In sum, up to first order we get the same equatuin as the references. Moreover the extra $a^2$ factor is in $\Omega =  \frac{a^2 \bar{\rho}}{3 M_{pl}^2 \mathcal{H}^2}$
\subsection{$T_{0i}$ component of energy momentum tensor}
According to the equation \ref{eqTmunuI} we can calculate $T_{0i}$ component of energy momentum tensor
\begin{align}
T_{\mu \nu}    & =(\bar{g}_{\mu \nu} + \delta g^{(1)}_{\mu \nu}) (\bar{P}+\bar{P}' \pi+\bar{P}_{,X} \delta X_1) + (\bar{P}_{,X}+\bar{P}'_{,X} \pi+\bar{P}_{,XX} \delta X_1) \partial_{\mu} (\tau+ \pi) \partial_\nu (\tau+\pi)+ \ldots
%\label{eqTmunu}
\end{align}
So $T_{0i}$ reads;
\begin{align} \label{T0i}
T_{0 i} &
= \Big[\cancel{\bar{g}_{0 i}} \bar{P} 
+
 \bar{P}_{,X} \partial_{0} \tau \partial_{i} \tau \Big] \epsilon ^0 
+
\Big[ \cancel{\bar{g}_{0 i}} \bar{P}' \pi+ \cancel{\bar{g}_{0 i}}  \bar{P}_{,X} \delta X_1 
+
 \delta g^{(1)}_{0 i} \bar{P} 
 +
  \bar{P}_{,X}  \Big( \partial_{0} \pi \cancel{ \partial_{i} \tau  }
  +
  \partial_{0} \tau \partial_{i} \pi  \Big ) 
      \nonumber  \\&
  +
   \delta X_1 \bar{P}_{,XX}   \partial_{0} \tau  \cancel{\partial_{i} \tau  }
   +
     \delta X_1 \bar{P}_{,XX}   \partial_{0} \tau  \partial_{i} \pi
  +
    {\bar{P}_{,X} } \partial_{0} \pi \partial_{i} \pi \; \;   \Big ]
+ \mathcal {O}(\epsilon^{2}) 
\nonumber 
\\ 
&
= [0] \epsilon ^0 
+
\Big[
 \delta g^{(1)}_{0 i} \bar{P} 
 +
  \bar{P}_{,X}   \partial_{0} \tau \partial_{i} \pi +  {\bar{P}_{,X} } \partial_{0} \pi \partial_{i} \pi + \delta X_1 \bar{P}_{,XX}    \partial_{i} \pi
  \Big ] 
+ \mathcal {O}(\epsilon^{2}) 
\nonumber 
\\ 
&
= 
3 M_{pl}^2 \mathcal{H}^2 \Omega \Bigg[
 w \, \frac{\delta g^{(1)}_{0 i} }{a^2}
 +
   (1+w )\,  \partial_{i} \pi + (1+w) \, \pi' \partial_{i} \pi  -(1+ w)(\frac{1}{c_s^2}-1)  \frac{(\vec{\nabla} \pi)^2}{2} \partial_{i} \pi 
   \Bigg ]
+ \mathcal {O}(\epsilon^{2}) 
%\label{eqTmunu}
\end{align}
If we neglect vector perturbation and keeping only short wave correction in first order we have,
%\begin{empheq}[box=\mymath]{equation*}
\begin{align}
T_{0i}= 
3 M_{pl}^2 \mathcal{H}^2 \Omega \Bigg[
    (1+w )\,  \partial_{i} \pi -(1+ w)(\frac{1}{c_s^2}-1)   \frac{(\vec{\nabla} \pi)^2}{2} \partial_{i} \pi 
   \Bigg ]
+ \mathcal {O}(\epsilon^{2}) 
\end{align}
%\end{empheq}
Note that $T_{0i}=T_{i0}$. To first order the followed expression is the same as equation 148 which is $q_D=-\bar{\rho} (1+w) \pi$, which means $T_{0i} = -\bar{\rho} (1+w) \partial_i \pi/a = -\bar{\rho} (1+w) \partial_i \pi_{\text{conf}} $, since $T_{0i}=g_{00}T^0_i=- T^0_i$ and $g_{00}$ in the paper is "-1".
\subsection{$T_{ij}$ component of energy momentum tensor}
It is noteworthy to mention that by $P'$ we mean $P_{\varphi}$ while the expression for $P_{\tau}= P_{\varphi} \varphi' + P_{X} X'$ since P is a function of $\varphi$ and $X$.
Again using the equation \ref{eqTmunuI}
\begin{align}
T_{\mu \nu} &
= \Big[ \bar{g}_{\mu \nu} \bar{P} 
+
 \bar{P}_{,X} \partial_{\mu} \tau \partial_{\nu} \tau \Big] \epsilon ^0 
+
\Big[ ({\bar{g}_{\mu \nu}} \bar{P}' +\bar{P}'_{, X})\pi+ \bar{g}_{\mu \nu}  \bar{P}_{,X} \delta X_1 
+
 \delta g^{(1)}_{\mu \nu} \bar{P} 
 +
  \bar{P}_{,X}  \left ( \partial_{\mu} \pi \partial_{\nu} \tau 
  +
  \partial_{\mu} \tau \partial_{\nu} \pi  \right ) 
       \nonumber \\ &
  +
   \delta X_1 \bar{P}_{,XX}   \partial_{\mu} \tau \partial_{\nu} \tau  
   +
    \bar{P}_{,X}   \partial_{\mu} \pi \partial_{\nu} \pi  + 
    +  \delta X_1 \bar{P}_{,XX}  \big(  \partial_{\mu} \tau \partial_{\nu} \pi  +   \partial_{\mu} \pi \partial_{\nu} \tau \big ) \Big]
+ \mathcal {O}(\epsilon^{3/2}) 
%\label{eqTmunu}
\end{align}.
So $T_{ij}$ is;
\begin{align} \label{Tij}
T_{i j} &
= \Big[ \bar{g}_{i j} \bar{P} 
+
 \bar{P}_{,X} \cancel{ \partial_{i} \tau}\partial_{j} \tau \Big] \epsilon ^0 
+
\Big[ ({\bar{g}_{i j}} \bar{P}' \pi+\cancel{\bar{P}'_{, X}} \partial_{i} (\tau+\pi) \partial_{j}(\tau+\pi))\pi+ \bar{g}_{i j}  \bar{P}_{,X} \delta X_1 
+
 \delta g^{(1)}_{i j} \bar{P} 
        \nonumber \\ &
 +
  \bar{P}_{,X}  \left ( \partial_{i} \pi \cancel{\partial_{j} \tau  }
  +
\cancel{  \partial_{i} \tau }\partial_{j} \pi  \right ) 
  +
   \delta X_1 \bar{P}_{,XX}   \cancel{ \partial_{i} \tau }\partial_{j} \tau  
   +
    \bar{P}_{,X}   \partial_{i} \pi \partial_{j} \pi \Big ]
+ \mathcal {O}(\epsilon^{2}) 
\nonumber \\ & 
%::::::::::::::::::::::::::::::::::::::::::::::::::::::::::::::::::::::::::::::::::::::::::::::::::::::::::::::::
%::::::::::::::::::::::::::::::::::::::::::::::::::::::::::::::::::::::::::::::::::::::::::::::::::::::::::::::::
= \Big[ a^2 \delta_{ij} \bar{P} 
 \Big] \epsilon ^0 
+
\Big[ a^2   (\delta_{ij}  \bar{P}' + P_{,X} \bar{X}' )\pi+ a^2  \delta_{ij}   \bar{P}_{,X} \delta X_1 
-
 2 a^2 \Phi \delta_{ij} \bar{P} 
     +
    \bar{P}_{,X}   \partial_{i} \pi \partial_{j} \pi \Big ] 
+ \mathcal {O}(\epsilon^{2}) 
\nonumber \\ & 
%::::::::::::::::::::::::::::::::::::::::::::::::::::::::::::::::::::::::::::::::::::::::::::::::::::::::::::::::
%::::::::::::::::::::::::::::::::::::::::::::::::::::::::::::::::::::::::::::::::::::::::::::::::::::::::::::::::
= \frac{ a^2 \bar{P}}{w}   \Big[ w \delta_{ij} 
\Big] \epsilon ^0 
+
\frac{ a^2  \bar{P}}{w} \Big[ w   \delta_{ij}  \bar{P}_{,\tau}  /\bar{P} \pi+  \delta_{ij}   (1+w)  \Big (-{\mathcal{H}} \pi-\Psi+{\pi'}- \frac{ (\vec{\nabla} \pi)^2 }{2 } \Big ) 
-
 2  w \Phi \delta_{ij} 
        \nonumber \\ &
     +
   (1+w)  \partial_{i} \pi \partial_{j} \pi \Big ] 
+ \mathcal {O}(\epsilon^{2}) 
\nonumber \\ & 
%::::::::::::::::::::::::::::::::::::::::::::::::::::::::::::::::::::::::::::::::::::::::::::::::::::::::::::::::
%::::::::::::::::::::::::::::::::::::::::::::::::::::::::::::::::::::::::::::::::::::::::::::::::::::::::::::::::
=3 M_{pl}^2 \mathcal{H}^2 \Omega    \Bigg[w \delta_{ij} -3 \mathcal{H} w (1+w)\pi \delta_{ij} 
+
 \delta_{ij}   (1+w)  \Big (-\Psi+{\pi'}- \frac{(\vec{\nabla} \pi)^2  }{2 }  \Big )
-
 2 w \Phi \delta_{ij} 
     +
   {(1+w)}  \partial_{i} \pi \partial_{j} \pi \Bigg ]
          \nonumber \\ & 
%:::::::::::::::::::::::::::::::::::::::::::::::::::::::::::::::::::::::::::::::::::::::::::::::::::::::::::::::: 
%:::::::::::::::::::::::::::::::::::::::::::::::::::::::::::::::::::::::::::::::::::::::::::::::::::::::::::::::
\end{align}
As a result we can write;
%\begin{empheq}[box=\mymath]{equation*}
\begin{align}
T_{ij}=&3 M_{pl}^2 \mathcal{H}^2 \Omega  \Bigg[  w \delta_{ij}  -3 \mathcal{H} w (1+w)\pi \delta_{ij}
-
 2 w \Phi \,  \delta_{ij} + (1+w) ( {\color{blue} ({\pi'}+ \mathcal{H} \pi) }-\Psi)  \, \delta_{ij}
 \nonumber  \\ &
 - \frac{1+w}{2 }   (\vec{\nabla} \pi)^2 \, \delta_{ij} +({1+w}  )\partial_{i} \pi \partial_{j} \pi    \Bigg ] 
     + \mathcal {O}(\epsilon^{2}) 
\end{align}
%\end{empheq}
Comparing with equation 150 of https://arxiv.org/pdf/1411.3712.pdf:
\be
T_{ij}^{(paper)}= g_{ii} T^i_{j} =a^2 \Big(\dot{p} \pi + \frac{\rho_D + p_D}{M^2} \times M^2( \dot{\pi} -\Psi) \Big) \delta_{ij} = \delta_{ij} a^2[-3 \frac{\mathcal{H}}{a} \bar{\rho} (1+w) a \pi_{\text{conf}} + \bar{\rho} (1+w) ( {\pi}'_{\text{conf}} -\Psi))]
\ee
Which is the same  and extra $a^2$ coefficient in their paper  is in the definition of $\Omega$ which is defined by physical Hubble constant! Moreover extra term -2$w \Phi$ comes from $\delta g_{ik} T^{k}_j$
The diagonal components read; 
\be
T_{ii}=3 M_{pl}^2 \mathcal{H}^2 \Omega   \Bigg[   w -3 \mathcal{H} w (1+w) \pi
-
 2 w  \, \Phi  - (1+w)  \, \Psi +  (1+w) \,  {\pi'} 
 + \frac{1+w}{2 }    (\vec{\nabla} \pi)^2    \Bigg ] 
     + \mathcal {O}(\epsilon^{3/2}) 
\ee
The off diagonal components are; 
\be
T_{ij}=3 M_{pl}^2 \mathcal{H}^2 \Omega (1+w)   \,   \partial_{i} \pi \partial_{j} \pi   
     + \mathcal {O}(\epsilon^{3/2}) 
\ee
\section{Field transfer function}
Before trying to get the field transfer function we try to do some tests on different results in order to catch all the physics we are dealing with.

\section{Gevolution implementation  {\color{red} Need to be checked and use the corrected stress tensor and check with other references!} }
In Gevolution we should use $ M^2_{pl}= 1/8 \pi G$.\\
%And it seems the normalization factor is $-3 \mathcal{H}_0^2T_0^0/8\pi G$ \\
So we have:
%\begin{empheq}[box=\mymath]{equation}
\begin{align}
T_{0i}= 
3 M_{pl}^2 \mathcal{H}^2 \Omega \Bigg[
    (1+w )\,  \partial_{i} \pi -(1+ w)(\frac{1}{c_s^2}-1)   \frac{(\vec{\nabla} \pi)^2}{2} \partial_{i} \pi 
   \Bigg ]
+ \mathcal {O}(\epsilon^{2}) 
\end{align}

\begin{align}
 & T_0^0 (Gev)=-a^3 {T_{0}^{0}}=   {3 a M_{pl}^2   \mathcal{H}^2\Omega} \Bigg[1+ \frac{1+w}{c_s^2} \Big(- 3 \mathcal{H}c_s^2 \pi- \Psi+  {\pi'}  -  \Big(1-2 c_s^2 \Big) 
 \frac{(\vec{\nabla} \pi)^2}{2} \Big )   \Bigg ]
\nonumber \\ &
T^{0}_{i}(Gev)=a^3 T^0_i = -a T_{0i} = -{3 a M_{pl}^2   \mathcal{H}^2\Omega}  (1+w)\Big[1 - (\frac{1}{c_s^2} -1)  \frac{(\vec{\nabla} \pi)^2}{2}  \Big ] \partial _i \pi 
\nonumber \\ &
T_{j}^{i}(Gev)= a^3 T_j^i = {3 a M_{pl}^2   \mathcal{H}^2\Omega w} \Bigg ( 1+  \frac{1+w}{w}\Big [ -3 \mathcal{H} w \pi- \Psi + \pi' -  \frac{(\vec{\nabla} \pi)^2}{2}   \Big] \delta_{j}^{i}  + \frac{1+w}{w} \delta^{i k} \partial_k \pi \partial_j \pi  \Bigg) 
\end{align}
%\end{empheq}
Note that $\dot{H}$ is determined by all the matter contents of the universe not by k-essence alone, the continuity equation for k-essence or matter gives the dynamics of density. \\
About the unit of $T^0_0$ note that it is $\bar{\rho}_{kessence} [1+\delta \rho/\bar{\rho} ]$ and since in Gevolution it is multiplied to $-a^3$ and since critical density at redshift zero is 1 so we have\\
\be
3 a M_{pl}^2   \mathcal{H}^2\Omega = a^3   \frac{\bar{\rho}_D}{\rho_{crit} ^{0}=1} = a^3   \frac{\bar{\rho}^0_D a ^{-3(1+w)}}{\rho_{crit} ^{0}}= \Omega^{0}_{kess} a^{-3w}
\ee
%\begin{empheq}[box=\mymath]{equation}
\begin{align}
 & T_0^0 (Gev)=  \Omega^0_{kess} a^{-3 w}  \Bigg[1+ \frac{1+w}{c_s^2} \Big(- 3 \mathcal{H}c_s^2 \pi- \Psi+   {\color{blue} ({\pi'}+ \mathcal{H} \pi) }  -  \Big(1-2 c_s^2 \Big) 
 \frac{(\vec{\nabla} \pi)^2}{2} \Big )   \Bigg ]
\nonumber \\ &
T^{i}_{0}(Gev)= - \Omega^0_{kess} a^{-3 w} (1+w) \Big[1 - (\frac{1}{c_s^2} -1)  \frac{(\vec{\nabla} \pi)^2}{2}  \Big ] \partial _i \pi 
\nonumber \\ &
T_{j}^{i}(Gev)= w  \, \Omega^0_{kess} a^{-3 w} \Bigg ( 1+  \frac{1+w}{w}\Big [ -3 \mathcal{H} w \pi- \Psi +   {\color{blue} ({\pi'}+ \mathcal{H} \pi) } -  \frac{(\vec{\nabla} \pi)^2}{2}   \Big] \delta_{j}^{i}  + \frac{1+w}{w} \delta^{i k} \partial_k \pi \partial_j \pi  \Bigg) 
\end{align}
%\end{empheq}
{\color{red} What is $\delta P/\delta\rho$ here and is it comparable with other papers?}
In Gevolution we extract $\delta T_0^0/ \bar{T}_0^0$ which scales out the coefficient $\Omega^0_{kess} a^{-3 w} $. \\
The field equation is:
we take $d \tau=\tau_{n+1}-\tau_n $ and $x_{i,j,k} $ as lattice point. We solve the differential equation numerically as following;
\be
\pi_v= {\pi}'
\ee
\be
\pi^{n}= \pi ^{n-1}+\pi_v ^{n-\frac{1}{2}} d \tau
\ee
\be \label{eq3}
\pi_v ^{n+\frac{1}{2}}=\pi_v ^{n-\frac{1}{2}} + {\pi''} ^{(n)}  d \tau
\ee

We define the laplacian in code as following,
\begin{align}
& \nabla^2 \pi =-\frac{\pi^{n}_{i-1,j,k}+\pi^{n}_{i+1,j,k} +\pi^{n}_{i,j-1,k} +\pi^{n}_{i,j+1,k}+\pi^{n}_{i,j,k-1}+\pi^{n}_{i,j,k+1} -6 \pi^{n}_{i,j,k}  }{ a^2 dx^2}  
\end{align}
Moreover in order to get scalar in the vertices the derivatives like $\nabla \pi . \nabla \pi $ should be defined symmetric .
So we can rewrite the equation \ref{eq3} as below;
After some simplification we get,
\begin{align} 
%
 &\pi_v ^{n+\frac{1}{2}}= \frac{1}{1+ d\tau   \mathcal{H}^{(n)}  (1-3w) /2 - d\tau (1-c_s^2) \nabla^2 \pi^{(n)}/2} \times \Bigg[ \pi_v ^{n-\frac{1}{2}} - d \tau \Big [(1- 3w)\mathcal{H}^{(n)}   \frac{\pi_{v \; {i,j,k}}^{n-\frac{1}{2}} }{2}
     \nonumber
     \\
      &
  -3 { c_s^2 \mathcal{H}^{(n)}}\Big( 1- \frac{w}{c_s^2} \Big )\Psi^{(n) }
 - \, \frac{{\Psi}^{(n)}-{\Psi}^{(n-1)} }{d \tau}
      - 3  c_s^2  \, \frac{{\Phi}^{(n)}-{\Phi}^{(n-1)} }{d \tau}    
   +\Big( 3\mathcal{H}^2 (c_s^2 -w) + \mathcal{H}' (1-3c_s^2) \Big)\pi^{(n)}  
             \nonumber
     \\
      &
       - c_s^2 {\nabla^2 \pi ^{(n)}}  
   %
   + (1-c_s^2)\Psi^{(n)} {\nabla^{2} \pi^{(n)}  }    
      %___________
    - 2 c_s^2  \Phi ^{(n)}  {\nabla^2 \pi^{(n)}}
          %___________
     + {3 c_s^2  \mathcal{H}^{(n)} (1+w) }\pi^{(n)} {\nabla^2 \pi^{(n)} }   
          \nonumber
     \\
      &
               %___________
     -  (1-c_s^2)
 \Big( \frac{\pi_{v \; {i,j,k}}^{n-\frac{1}{2}} }{2} +\mathcal{H}  \pi^{(n)}  \Big) {\nabla^2  \pi^{(n)}} 
               %___________
    - (2 c_s^2-1) {\nabla  \Psi^{(n)}  . \nabla \pi ^{(n)} }
            %___________
    + c_s^2 {\nabla  \Phi ^{(n)} . \nabla \pi^{(n)}  }  
                              \nonumber
     \\
       & 
            %___________
              +\frac{\mathcal{H}^{(n)}} {2 } \Big(2+3w+c_s^2  \Big) \,{\nabla  \pi^{(n)} . \nabla \pi^{(n)} }  
                      %___________                    
         -2(1-c_s^2) \Big( \frac{{ \nabla  ( \pi_{v  \; {i,j,k}}^{n+\frac{1}{2}} +\pi_{v \; {i,j,k}}^{n-\frac{1}{2}} ) }  } {2}  + \mathcal{H}  \nabla\pi^{(n)} \Big) 
    \Big] \Bigg]
\end{align}
Since we have $\nabla \pi_v $ in the equation we use predictor corrector method as following,\\
In the first step we predict that the term $\nabla \pi \nabla \pi_v$ is small so we approximate $\nabla \pi_v ^{(n)}$ with $\nabla \pi_v^{n-\frac{1}{2}}$ then we calculate $\pi_v^{n+\frac{1}{2}}$ according to the formula with the guess, then we use the new $\pi_v^{n+\frac{1}{2}}$ into the full equations to correct $\pi_v^{n+\frac{1}{2}}$ . \\
 We have taken $\pi_{v  \; {i,j,k}}^{n} =\frac{(\pi_{v  \; {i,j,k}}^{n+\frac{1}{2}} +\pi_{v \; {i,j,k}}^{n-\frac{1}{2}} )}{2} $. Then we need to calculate $\mathcal{H}'$, ${\Psi}'$ and  ${\Phi}'$ in each loop, to calculate ${\Psi}'$ we save two $\Psi$ in each loop. \\
% \begin{align} 
% &\pi'' - \mathcal{H} \Big (1+ 3w \Big)\pi' -3 {a c_s^2 \mathcal{H}}\Big( 1- \frac{w}{c_s^2} \Big )\Psi -a \, {\Psi'}- 3 c_s^2 a \,{\Phi'} 
%  +3  c_s^2 \Big({-\mathcal{H}^2 + \mathcal{H}'} \Big) \pi 
% - c_s^2 {\nabla^2 \pi }
%     + (1-c_s^2)\pi {\nabla^2 \Psi }
%      - 2 c_s^2 \pi {\nabla^2 \Phi }
%            \nonumber
%   \\
%    &
%      + 3 c_s^2  H (1+w)\pi {\nabla^2 \pi }   
%  -  (1-c_s^2)
%   \pi \frac{\nabla^2\pi ' }{a}   
%   - (2 c_s^2-1) {\nabla  \Psi . \nabla \pi }
% + c_s^2 {\nabla  \Phi . \nabla \pi }  
%   +\frac{\mathcal{H}} {2 a } \Big(2+3w+c_s^2  \Big) \,{\nabla  \pi . \nabla \pi }     =0 
%%  -(\frac{1}{c_s^2}-1) \nabla^2 \Psi \pi+ 2 \nabla^2 \Phi \pi - 3 H (1+w) \pi \nabla^2 \pi  + (\frac{1}{c_s^2}-1) \pi \nabla^2 \dot{\pi}   \nonumber \\ &+ (2-\frac{1}{c_s^2})\nabla \Psi \nabla \pi - \nabla \Phi \nabla \pi -\frac{H} {2 c_s^2} \Big(2+3w+c_s^2  \Big) \nabla \pi \nabla \pi =0
%  \end{align} 
%%%%%%%%%%%%%%%%%%
%%%%%%%%%%%%%%%%%%
%%%%%%%%%%%%%%%%%%


\section{The constraints from stress energy tensor  {\color{red} Need to be checked and use the corrected stress tensor and check with other references!}}
Here we use the stress tensor conservation to obtain the field evolution. Easy way of obtaining the equations (Euler and continuity) is to derive the equations for a general fluid  then calculate it for the special case of k-essence.
We define the energy-momentum tensor as,
\begin{align} \label{tmunu22}
& T_{0}^{0}=-(\bar{\rho}^{d} + \delta \rho^{d} )\\ \nonumber &
T_{i}^{0}=(\bar{\rho} + \bar{P}) v_i= - T^{i}_{0}\\ \nonumber &
T_{j}^{i}=(\bar{P}^{d}+\delta P^{d}) \delta_{i}^{j} +\Sigma_{j}^{i}, \; \; \; \; \Sigma_{i}^{i}=0,
\end{align}
Note that all of the fluid quantities here are symbolic to simplify the equations. So the $\delta \rho^{d}$ here is not necessarily same as $\delta \rho $ in previous section, they are two different quantities in general. Moreover we have allowed anisotropic pressure part $\Sigma_{ij}$ since the perturbative scheme let second order contributions to first order equations. \\
Now we calculate $T_{\mu \nu}$ for a general fluid to find k-essence correspondence. By $T_{\mu \nu}=g_{\mu \rho} T^{\rho}_{\nu}$ we get,
\begin{align} 
& T_{00}=a^2 \Big (\bar{\rho}^d + \delta \rho ^d + 2 \bar{\rho}^d \Psi \Big)\\ \nonumber &
T_{i0}=- a^2(\bar{\rho}^d + \bar{P}^d) v_i=  T_{0i}\\ \nonumber &
T_{ji}=a^2 (\bar{P}^{d}+\delta P^{d} - 2  \bar{P}^{d} \Phi) \delta_{ij} +a^2 \Sigma_{j i}, 
\end{align}
According to equation \ref{T00}, we have,
\be \label{rhobar}
\bar{\rho}^{d}= \frac{3 M_{pl}^2   \mathcal{H}^2\Omega}{a^2}, \; \; 
\;  \; \; \;   \delta \rho^{d}=\frac{3 M_{pl}^2   \mathcal{H}^2\Omega (1+w)}{c_s^2 \, a^2} \Bigg[-  \mathcal{H}c_s^2 \pi- \Psi+   {\pi'}  -  \Big(1-2 c_s^2 \Big) 
 \frac{(\vec{\nabla} \pi)^2}{2}   \Bigg]
\ee 
Using equation \ref{T0i} we have,
\be
u_{i}^{d}= -\Big[1+ \pi' +(\frac{1}{c_s^2} -1) \Big(-\Psi +\pi' - \frac{(\vec{\nabla} \pi)^2}{2}  \Big ) \Big ] \partial _i \pi  , \; \; 
\;  \; \; \; \bar{P}^d= w  \bar{\rho}^d 
\ee
and from equation \ref{Tij}
\be
 \delta P ^{d} = \frac{3 M_{pl}^2   \mathcal{H}^2\Omega(1+ w)}{a^2}  \Big [ -3 \mathcal{H} w \pi- \Psi + \pi' +  \frac{(\vec{\nabla} \pi)^2}{2}   \Big], \; \; 
\;  \; \; \; 
\Sigma_{ij}=\frac{3 M_{pl}^2   \mathcal{H}^2\Omega (1+w) \partial_i \pi \partial_j \pi}{a^2}
\ee
Where we have used background continuity equation \ref{Conteqgg}.\\
It is important to note that,
\be
  \delta P ^{d} -  c_s^2 \delta \rho^{d} =\frac{3 M_{pl}^2   \mathcal{H}^2\Omega (1+w)}{a^2} \Bigg ( (-3 w +c_s^2)\mathcal {H} \pi + (1-c_s^2) (\vec{\nabla} \pi)^2 \Bigg)
\ee
The top equations seem correct according to comparison with equations (147-150) of \url{https://arxiv.org/pdf/1411.3712.pdf}. \\
According to energy-momentum conservation,
\be
T^{\mu}_{ \nu \, ; \mu }= \partial_{\mu} T_{ \nu} ^{\mu}+ \Gamma^{\mu}_{\rho \mu} T_{\nu}^{ \rho}-\Gamma^{\rho}_{\mu \nu} T_{\rho} ^{ \mu}  =0
\ee
The background constraint is,
\be \label{conteq}
\bar{\rho}'^{d}+3 \mathcal{H}(\bar{\rho}^{d} + \bar{P}^{d})=0
\ee
Defining the below variables;
\be
 \Sigma_{j}^{ i}=T_j^i - \delta_j^i T_k^k/3, \,\,\;\; (\bar{\rho} +\bar{P} )\theta = \partial _k  \delta ^{k j}\delta T_j^0.  \,\,\;\  (\bar{\rho} +\bar{P} ) \sigma=(\delta ^{kj} \partial_i \partial_k - \frac{1}{3} \delta_{i}^j) \Sigma_j^i
\ee
The Euler and continuity equations in Fourier space read,
\begin{align}
 & {\delta'}^{d} = -(1+w) (\theta^{d}  + 3 \Phi ') -3 \mathcal{H} \Big({\delta P^{d}/\delta \rho^{d}} -w \Big)\delta^{d}  
     \\  \nonumber &
 {\theta'}^{d} = -\mathcal{H} \Big( 1 - 3 w   \Big )  \theta^{d} - \frac{{w'}}{1+w} \theta ^{d}+ \frac{\delta P^{d} \delta \rho^{d}}{1+w} \, k^2 \delta^{d}  - k^2 \sigma +k^2 \Psi
 \label{eqEul}
\end{align}

where  $\delta \rho^{d}= \rho^{d} -\bar{\rho}^{d}$, $\delta P^{d} =P ^{d}-\bar{P}^{d}$, $\theta= i \vec{k}. \vec{u}$. \\
\subsection{Euler and continuity equations {\color{red} Need to be checked and use the corrected stress tensor and check with other references!}}
We can write Euler and continuity equations for k-essence case easily just by substituting the below quantities into the general equations.
\begin{align} 
 & \delta^{d} =\frac{\rho^d -\bar{\rho }^d}{\bar{\rho}^d} =  \frac{1+w}{c_s^2} \Bigg[ -  \, c_s^2 \mathcal{H} \pi- \Psi+   {\pi'}  -  \Big(1-2 c_s^2 \Big) 
 \frac{(\vec{\nabla} \pi)^2}{2}   \Bigg] 
 \nonumber
   \\ 
   &
 \theta^d=\vec{\nabla} .\vec{u}^d=\delta^{ij} \partial_j u_i ^d=- \Big[1+ \pi' +(\frac{1}{c_s^2} -1) \Big(-\Psi +\pi' - \frac{(\vec{\nabla} \pi)^2}{2}  \Big ) \Big ] \ \partial ^2 \pi  
  \nonumber
   \\ 
   &
   \frac{\delta P ^{d} }{ \delta \rho^{d} }=  c_s^2 \Bigg [1 +\frac{ (-3 w +c_s^2)\mathcal {H} \pi + (1-c_s^2) (\vec{\nabla} \pi)^2   }{ - 3c_s^2 \mathcal{H} \pi- \Psi+   {\pi'}  -  \Big(1-2 c_s^2 \Big) 
 \frac{(\vec{\nabla} \pi)^2}{2}  }
\Bigg ]
   \nonumber
   \\ 
   &
   \sigma=\frac{(\delta ^{kj} \partial_i \partial_k )\Sigma_j^i}{(\bar{\rho} +\bar{P} )} =\partial^2 \pi \partial^2 \pi
 \end{align} \label{imp_2}
 {\color{red} {Strange relation for $\delta P/\delta \rho?$}} \\
 Background part of Continuity equation read \\
 \begin{align} \label{Conteqgg}
& \frac{\bar{\rho}' }{\rho}=- 3 \mathcal{H} (1+w) \\ \nonumber &
 \frac{2\mathcal{H}'}{\mathcal{H}} + \frac{\Omega'}{\Omega}-2 \mathcal{H}=- 3 \mathcal{H} (1+w) \end{align}
 It is easy to check that if we substitute the value of $\bar{\rho}$ eq.\ref{rhobar} in the continuity equation \ref{conteq} we get the result. \\
 To get first order continuity equation we write $\delta^d$ as following,
 \begin{align}
 \delta'^{d}= \frac{1+w}{c_s^2} \pi''+ \delta'^{r} 
 \end{align} 
 where $\delta^ r$ is all other terms in $\delta^{d}$ except $\pi'$. We separate since this is the only second derivative in the constraint, so we can use Runge Kutta method and easily implement without having the equation completely. So we can write,
 \be
 \pi'' = \frac{c_s^2}{1+w} \Bigg[ -\delta'^r-(1+w) (\theta^{d}  + 3 \Phi ') -3 \mathcal{H} \Big({\delta P^{d}/\delta \rho^{d}} -w \Big)\delta^{d}     \Bigg]
 \ee
 Since we know the values of fluid quantities in terms of $\pi, \Psi...$ we can easily implement in the code without getting the equation explicitly. \\
 {\color{red}{We should check that this equation gives the same constraint as field constraint in EFT form for k-essence case.}} \\
 Moreover, we do not need to consider Euler equation since it does not give any extra equations.
\section{Field transfer function}
Before trying to get the field transfer function we try to do some tests on different results in order to catch all the physics we are dealing with.
\subsection{Gauge transformation}
Before going forward we introduce the gauge transformation for different quantities. \\
Considering a general coordinate transformation from coordinate system $x^{\mu}$ to another $\hat{x} ^{\mu}$ 
\be
x^{\mu} \longrightarrow {\hat{x}}^{\mu}  = x^{\mu} + \epsilon^{\mu} (x)
\ee
The metric tensor will be transformed as,
\be
\hat{g}_{\mu \nu} (\hat{x})=  \frac{\partial x^{\lambda}}{\partial \hat{x}^{\mu} } \frac{\partial x^{\rho}}{\partial \hat{x}^{\nu} } g _{\rho \lambda} (x)
\ee
This is a pure coordinate transformation which changes background and perturbation quantities in general. It is more convenient to work with "gauge transformation" which affect only the perturbations. So after coordinate transformation we relabel coordinates by dropping the prime on the coordinate argument, which means that we are dealing with two different points, since $x$ in two coordinates are assigned with two different but near physical points. Moreover gauge transformation guarantees that the background quantities remain unchanged. Since it is built to consider the points in different coordinates which have the same background values.\\ 
So the gauge transformation on the metric tensor gives 
\begin{align}
\Delta h_{\mu \nu} (x)   \equiv  &\hat{g}_{\mu \nu}(x) - g_{\mu \nu} (x) \\ \nonumber &
=\hat{g}_{\mu \nu}(\hat{x}^{ \kappa} -\epsilon^{\kappa}) - g_{\mu \nu} (x) =\hat{g}_{\mu \nu}(\hat{x} ) - \partial_{\rho } g_{\mu \nu} (x^{}) \epsilon^{\rho}  - g_{\mu \nu} (x)
 \\ \nonumber &
 =   \frac{\partial x^{\lambda}}{\partial( x^{\mu} +\epsilon ^{\mu}) } \frac{\partial x^{\rho}}{\partial( x^{\nu} +\epsilon ^{\mu})} g_{\rho \lambda} (x) - g_{\mu \nu} (x) - \partial_{\rho } g_{\mu \nu} (x^{}) \epsilon^{\rho}  - g_{\mu \nu} (x)
  \\ \nonumber &
 = - \bar{g} _{\lambda \mu} (x) \frac{\partial \epsilon ^{\lambda} (x) }{\partial x^{\nu}} -  \bar{g} _{\lambda \nu} (x) \frac{\partial \epsilon ^{\lambda} (x) }{\partial x^{\mu}} - \frac{\partial \bar{g}_{\mu \nu} (x)  }{\partial x^{\lambda } } \epsilon^{\lambda} (x) 
\end{align}
We denote the gauge transformation with $\Delta$. It is obvious that the background metric under gauge transformation remains unchanged ({\color{red}{How?}}), but it may contribute in first order which is taken into account in the previous equation.
\be
\Delta \bar{g} _{\mu \nu}= \hat{\bar{g}}_{\mu \nu} (t) -  \bar{g} _{\mu \nu} (t) = 0 + \mathcal{O } (\epsilon)
\ee 
\\
If we consider the coordinate transformation as following,
\begin{align}
& {\hat{x}}^0 = x^ 0 + \alpha ,  \nonumber \\ &
{\hat{\vec{x}}} ={\vec{x}} +\vec{\nabla} \beta(\tau,x) + \vec{\epsilon } (\tau,x), \; \; \; \; \; \vec{\nabla} . \vec{\epsilon}=0
\end{align}
We get
\be
{\hat{g}_{\mu \nu }}  (x)=  g_{\mu \nu } (x) - g_{\mu \beta } (x) \partial _{\nu} \epsilon^{\beta} -  g_{\alpha \nu } (x) \partial _{\mu} \epsilon^{\alpha}-\epsilon^{\alpha} \partial _{\alpha} g_{\mu \nu } (x) 
\ee
We easily can obtain the transformed metric perturbations in the new gauge (synchronous),
%\begin{empheq}[box=\tcbhighmath]{equation*}
\begin{align}
&\hat {\Psi} (\tau,\vec{x}) = \Psi(\tau,\vec{x}) - \alpha' (\tau,\vec{x})- \mathcal{H} \alpha(\tau,\vec{x}) \nonumber \\ &
\hat {\Phi} (\tau,\vec{x}) = \Phi(\tau,\vec{x}) +\frac{1}{3} \nabla^2 \beta(\tau,\vec{x})  + \mathcal{H} \alpha(\tau,\vec{x}) 
\end{align}
%\end{empheq}

Where $\Phi$ and $\Psi$ are metric perturbations in Newtonian gauge. To find the gauge transformation for fluid quantities, we use,
\be
T^{\mu} _ {\nu}(Sync) = \frac{\partial {\hat{x}} ^{\mu} }{\partial {x} ^{\sigma}} \frac{\partial {x} ^{\rho}}{\partial {\hat{x}} ^{\nu}} T^{\rho} _ {\sigma}(Newt) 
\ee
where ${\hat{x}}^{\mu}$ and $x^{\mu}$ denote he synchronous and Newtonian coordinates respectively. It follows to linear order that
\begin{align}
& T_0^0 (Sync) =T_0^0 (Newt),\\
\nonumber &
  T_0^j (Sync) =T_0^j (Newt) + i k_j \alpha (\bar{\rho} + \bar{P}) \\
\nonumber &
   T_i^j (Sync) =T_i^j (Newt)
 \end{align}
  where $\alpha = {\hat{x}}^0 - x^0=({h'} + 6 {\eta'})/2 k^2$ in terms of Synchronous gauge perturbations, it is useful since we want to use hi-class which is written in synchronous gauge. \\
From the definition of the density contrast $\delta = \delta \rho /\bar{\rho} = - \delta T_0^0 /\bar{\rho} $ we obtain,
%\begin{empheq}[box=\tcbhighmath]{equation*}
\begin{align}
& \delta (Sync) = \delta (Newt) -\alpha \frac{\dot{\bar{\rho}}}{\rho}
 \\
\nonumber & 
\theta (Sync) = \theta (Newt) - \alpha k ^2 \\
\nonumber &
\delta P (Sync) = \delta P (Newt) - \alpha \dot{\bar{P}}  \\ 
\nonumber &
\sigma  (Sync) = \sigma  (Newt)
\end{align}
%\end{empheq}
The obtain the gauge transformation for the scalar field and its derivative, we use the fact that it is a scalar under the general coordinate transformation $\hat{\varphi }({\hat{x}}^{\mu})=\varphi ({x}^{\mu})$ , where $\varphi  (x^{\mu} ) =t + \pi (x^{\mu})$ according to our time slicing.
\begin{align}
\Delta \varphi (x) \equiv &\hat{ \varphi}(x) - \varphi (x) = \hat{\varphi}(\hat{x} - \epsilon) -\varphi (x) \\ \nonumber &
=  \hat{\varphi} (\hat{x}) - \partial _ {\mu} \varphi \, \epsilon ^{\mu} - \varphi (x)= - \partial _ {\mu} \varphi \, \epsilon ^{\mu}
 \\ \nonumber &
 =- \partial_{\mu} (t+ \pi (x)) \epsilon^{\mu} = \epsilon ^0 = -\alpha  
\end{align}
To gauge transform the derivative of scalar field we should notice first how it transforms under coordinate transformation and using the fact that $\varphi$ is a scalar.
\begin{align}
 \frac{ \partial \hat{\varphi} (\hat{x})}{ \partial \hat{x}^0} & = \frac{ \partial \varphi  (x)}{ \partial x ^{\mu}}   \frac{\partial x^{\mu}}{\partial \hat{x}^0}
 = \frac{ \partial (t+ \pi (x))}{ \partial x ^{\mu}}   \frac{1}{\partial_{\mu} x^0 + \partial_{\mu} \alpha}
 \\ \nonumber &
 = (\delta_{\mu} ^0 + \partial_{\mu} \pi (x)) \left(\delta^{\mu}_0 - \partial_{\mu } \alpha \right)= 1+ \pi'(x) -  \alpha' (x)
\end{align}
\be
\hat{\pi}'(\hat{x})= \pi '(x) -\alpha '  (x)
\ee
where " $'$ " means time derivative.
So the gauge transformation on $\varphi '$ gives,
\begin{align}
\Delta { \varphi'(x)}\equiv & \hat{\varphi} ' (x)-  \varphi ' (x)=\hat{\varphi} ' (\hat{x}-\epsilon) -  \varphi ' (x) 
 \\ \nonumber &
 =\hat{\varphi'} (\hat{x}) - \partial _ {\mu} \varphi' \, \epsilon ^{\mu} - \varphi' (x)=- \alpha' - \cancel{\partial_{\mu} (1+ \pi '(x)) \epsilon^{\mu}}
\end{align}
\be
\pi_{Sync}=\pi_{Newt} - \alpha
\ee
\be
\pi'_{Sync}=\pi'_{Newt} -\alpha'
\ee
\subsection{Comparison the $\alpha$ from hi-class and density transfer function in the class  {\color{red} Get something consistent from below calculation!}}
\subsubsection{$\delta \rho$ transfer function}
According to equation \ref{eq10} we can write,
\begin{align}
\rho &=2 (\bar{X}+\delta X)(\bar{P}_{,X} +\bar{P}'_{,X} \pi + \bar{P}_{,XX} \delta X)- \bar{P} - \bar{P}_{,X} \delta X -  \bar{P}' \pi\\
\nonumber &
=(2 \bar{X}\bar{P}_{,X}- \bar{P} )+(2 \bar{X}  \bar{P}_{,XX} + \bar{P}_{,X}) \delta X + 2 \bar{X} \bar{P}'_{,X} \pi -\bar{P}' \pi
\nonumber  \\&
=\Big ( \frac{2}{ 2 a^2} a^2 \bar{P} (1+\frac{1}{w})-\bar{P} \Big)+ \Bigg(  \frac{1}{a^2}a^4 \bar{P} (1+\frac{1}{w} )(\frac{1}{c_s^2}-1) + a^2 \bar{P} (\frac{1}{w}+1) \Bigg) \delta X- \frac{\bar{P}}{w} (1+w)\mathcal{H} \pi
\nonumber  \\&
= \frac{\bar{P}}{w}+ \frac{\bar{P}}{w} \Big(  a^2  (1+w) (\frac{1}{c_s^2}-1) + a^2 (1+w) \Big) \delta X-\frac{\bar{P}}{w} (1+w)\mathcal{H} \pi
\nonumber  \\&
= \frac{\bar{P}}{w} \Bigg[1+  \Big( (1+w) (\frac{1}{c_s^2}-1) +  (1+w) \Big) \Big( -\Psi +\pi ' \Big) \Bigg] - \frac{\bar{P}}{w} (1+w)\mathcal{H} \pi
\nonumber
  \\&
= \frac{\bar{P}}{w} \Bigg[1- \mathcal{H} (1+w) \pi+ \frac{1+w}{c_s^2} \Big( -\Psi +\pi ' \Big) \Bigg]
\end{align}
where we have used equation \ref{Pbarder}-\ref{Pbar} and we have neglected $(\nabla \pi)^2$ term.
Finally we can get the density contrast  as following,
\be
\delta_{newt}=\frac{\rho- \bar{\rho}}{\bar{\rho}}=\frac{1+w}{c_s^2} \Big ( -c_s^2 \mathcal{H} \pi_{newt}-\Psi_{newt}+\pi' _{newt}\Big )
\ee
%Using the equation \ref{deltarho} we can get the density contrast as following,
%\be
%\delta=\frac{\rho- \bar{\rho}}{\bar{\rho}}=\frac{1+w}{c_s^2} \Big ( -\Psi+\pi' -  \frac{\vec{(\nabla} \pi)^2}{2} \Big )
%\ee
where $\bar{\rho}= \frac{\bar{P}}{w}$. The last relation shows that density contrast transfer function is obtained from time derivative of the field and the metric not the field . It is a good test to check that we can recover density contrast transfer function from field derivative and metric $\Psi$ transfer functions. The transfer function is defined as below:
\be
T_{kessence}(k,z_i)=\frac{\delta_{kessence} (k,z=z_i)}{\mathcal{R}(k,z=\infty)}
\ee
Note that the notation here is the same as class and hi-class and in the hi-class and class code the variables are normalized to the unity curvature perturbation so we do not need to normalize them. \\
One good test of the obtained field transfer function from hi-class is comparing the density transfer function which calculated from the last formula with the one we get from class fluid description.\\
The field equations are written in Synchronous gauge in the hi-class while we have calculated density contrast transfer function in the Newtonian gauge. So wee need to calculate density contrast transfer function in Newtonian gauge in terms of Synchronous gauge variable as following,
\begin{align}
\delta_{Newt}&=\frac{1+w}{c_s^2} \Big (  -\mathcal{H} (3c_s^2-1)  \pi_{newt}-\Psi_{newt}+\pi'_{newt} \Big )= \frac{1+w}{c_s^2} \Big (  -\mathcal{H} (3c_s^2-1) \mathcal{H} (\pi_{synch} + \alpha)-\alpha ' - \mathcal{H} \alpha+\pi'_{synch} +\alpha' \Big ) \\ \nonumber
&
= \frac{1+w}{c_s^2}\Big ( -\mathcal{H} (3c_s^2-1) \pi_{synch}- \mathcal{H} \alpha (1+c_s^2)+\pi'_{synch} \Big )
\nonumber \\ 
&
\delta_{Synch}=\delta_{Syn} + \alpha \frac{{\bar{\rho'}}}{\bar{\rho}}
\label{deltaeq}
\end{align}
where $\alpha=({h'} + 6 {\eta'})/2 k^2$, $\Psi=\frac{1}{2 k^2} \Big [ {h^{''}}+ 6 {\eta^{''}} + \frac{a'}{a}  (h'+6 \eta ') \Big]= {\alpha' + \mathcal{H}\alpha.}$  according to equation 18 of  {\color{blue}{https://arxiv.org/pdf/astro-ph/9506072.pdf}} \\
Here we want to compare the calculated $\delta _{Newt}$ from the field and metric perturbations in the hi-class with the class output in both Newtonian and Synchronous gauge.\\ 
According to the background equations ${\rho'} + 3 \mathcal{H} (\rho + w \rho)=0 $ for constant $w$ we have,
\be
\bar{\rho}_{kessence} (a) =a ^{-3(1+w)}
\ee
so we can write,
\be
 \frac{{\bar{\rho'}}}{\bar{\rho}}= -3 (1+w) \mathcal{H}
 \ee
First we compare  $\frac{(\delta_{Newt} - \delta_{Syn})}{-3 (1+w) \mathcal{H}}$ with $\alpha$ that we read from the hi-class. Note that according to the code units $\mathcal{H}(z=0)=H(z=0)=2.25 \times10^{-4} /Mpc $. 
\\
 In the below figure we have compared the $\alpha$ in hi-class with what we get from the calculation in the class. These two quantities agree well.
\begin{figure}[H]
\begin{center}
\captionsetup{,margin=1cm}
\includegraphics[width=0.60\textwidth]{alpha_class} 
\caption{The  $\alpha$ is in hi-class and class is compared. As it is clear they agree well for the the limit $c_s^2 \to 1$. }
%\label{f1}
\end{center}
\end{figure}
%\begin{figure}[htbp!]
%\begin{center}
%\captionsetup{,margin=1cm}
%\includegraphics[width=0.60\textwidth]{alphaerror} 
%\caption{The  relative error of $\alpha$ in class and hi-class is shown. It is clear that  the relative error is very large in high wavenumbers.}
%%\label{plt2}
%\end{center}
%\end{figure}
So we have,
\be
\delta_{Newt}= \frac{1+w}{c_s^2} \Big ( -\mathcal{H} (3c_s^2-1)  (\pi_{synch} + \alpha)- \mathcal{H} \alpha+\pi'_{synch}  \Big )
\ee
\be
\delta_{Synch}= \delta_{newt} + 3 \mathcal{H} (1+w) \alpha
\ee
\subsection{comparing the $\Psi$ function in Newtonian and synchronous gauge in class and hi-class}
Here we compare the metric fluctuation $\Psi$  in Newtonian gauge with one we obtain from Synchronous gauge parameters in class and hi-class codes to make sure we are dealing with the same functions. The $\Psi$ in Synchronous gauge reads as following,
\be
\Psi=\frac{1}{k^2} \Big [ {h^{''}}+ 6 {\eta^{''}} + \frac{a'}{a}  (h'+6 \eta ') \Big]= \alpha' + \mathcal{H}\alpha
\ee
In the figure \ref{psicomp} we compare the $\Psi$ in Newtonian gauge from what we get in terms of Synchronous gauge quantities.
\begin{figure}[H]
\begin{center}
\captionsetup{,margin=1cm}
\includegraphics[width=0.60\textwidth]{psi_comp.jpg} 
\caption{$\Psi$ in Newtonian gauge and synchronous gauge in class and hi-class are compared. They all agree. }
\label{psicomp}
\end{center}
\end{figure}

\subsection{Comparing the  ${\pi}$ transfer function directly from hi-class and what we get from class indirectly}
From the equation for $\theta$, equation \ref{imp_2} we can obtain $\pi$ in Fourier space,
\be
\pi_{newt}=\frac{\theta_k\text{(newt)}}{k^2}  \; \; \; \text{(class)}
\ee
while in the hi-class we can easily obtain $\pi_{newt}$ by gauge transformation as following,
\be
\pi_{newt}= \pi_{synch}+ \alpha
\ee
An important key is that k in hi-class internally is in $1/Mpc$ unit while is class output is $h/Mpc$.\\
We assume two different models, one with $w_0=-0.9, c_s^2=1$ and the other with $w_0=-0.9, c_s^2=10^{-6}$ and we compare the results in both gauges. %The normalization factor in EFTcamb is assumed $\mathcal{H}$ and in class and hi-class the transfer function is assumed to be normalized to curvature function $\mathcal{R}$. So in the figure we have multiplied the EFTcamb result to $\mathcal{H}$ , hi-class and class result to $\delta_{\mathcal{R}}(k)= \frac{\sqrt{2 \pi^2 A_s(\frac{k}{k_p})^{n_s-1}}}{k^{3/2}}$.
\begin{figure}[H]
\begin{center}
\captionsetup{,margin=1cm}
\includegraphics[width=0.60\textwidth]{pi_comp} 
\caption{$\pi$ comparison in Newtonian gauge for $w_0=-0.9, c_s^2=1$. Note that here we have use wrong gauge transformation for $\pi$ with different sign}
%\label{plt2}
\end{center}
\end{figure}


\subsection{Test 2: Density transfer in the limit of $c_s^2 \rightarrow 0 , w \rightarrow -1$ which should be the same as matter density transfer function}
In the limit of zero sound speed and $w_0 \rightarrow -1$ {\color{red}(must be $w \rightarrow 0$) }we expect that the k-essence field behave like dark matter. In the fluid description in the class it can be seen easily. In the below figure the density contrast transfer function for k-essence fluid and dark matter fluid is plotted. \\
We expect the same behaviour for the obtained k-essence field from hi-class or EFTcamb. So a good criterion is checking the behaviour in the mentioned limit.
\begin{figure}[H]
\begin{center}
\captionsetup{,margin=1cm}
\includegraphics[width=0.60\textwidth]{cs0w1_fld.jpg} 
\caption{The behaviour of fluid and dark matter transfer function in the zero sound speed and $w_0 =-1$ limit is shown.  }
%\label{psicomp}
\end{center}
\end{figure}
Now we want to compare the result of density contrast transfer function of k-essence field in EFTcamb and hi-class with density contrast transfer of  fluid in the class. \\
I do not need to do this part, since I get good results in previous section.
%The density constrast $\delta$ in the EFTcamb and hi-class is calculated,
%\be
%\delta=\frac{1+w_0}{c_s^2} (\pi'- \Psi) =10^4 (\pi'- \Psi)
%\ee
%As we have checked $\Psi$ is the same in Class and hi-class so we can use $\Psi$ to obtain the $\delta$ of EFTcamb as well.
\subsection{Comparing the  ${\pi'}$ transfer function directly from hi-class and what we get from class indirectly}
Here we use $\delta$ transfer function from the Class output to construct $\pi'$ in Newtonian gauge. We get help of equation \ref{deltaeq} to reconstruct $\pi'$ as following,
\be
\pi'_{newt}= \frac{c_s^2}{1+w} \delta_{newt} + c_s^2 \mathcal{H} \frac{\theta_{newt}}{k^2} +\Psi
\ee
The we compare the result with what we get from the hi-class after gauge transformation as following,
\be
\pi'_{Newt} =\pi'_{Sync}+\alpha'
\ee
\subsection{Comparison of field transfer functions in the hi-class and the EFTcamb}
Now we want to use EFTcamb results for the k-essence which is straightforward to get the output , \\
%{\color{red} If we use pureEFT flag in EFTcamb, what are the related parameters for k-essence case?  since the translation between the standard language with EFTcamb is not trivial according to table 1 of   \url{https://arxiv.org/pdf/1411.3712.pdf} }
%In the beginning we use minimally coupled quintessence flag in the EFTcamb to check the consistency, then we should try the pureEFT flag. We choose the quintessence flag according to \url{http://www.eftcamb.org/images/EFTCAMB_structure.pdf} in the second part.
The result of comparison is shown in the following plot which shows that the two plot do not agree.
\begin{figure}[H]
\begin{center}
\captionsetup{,margin=1cm}
\includegraphics[width=0.60\textwidth]{eft_hiclass.jpg} 
\caption{EFTcamb and hi-class comparison.Here we compare the $\pi$ in Synchronous gauge. $1/\mathcal{H}$ is used as a  normalization factor for EFTcamb and k in hiclass is divided by h to be measures in $h/Mpc$.}
%\label{psicomp}
\end{center}
\end{figure}

\section{Gevolution}
\subsection{Initial condition}
All the transfer functions in the class and hi-class are normalized to one curvature perturbation. Curvature perturbation is obtained from powerspectrum as following,
\be
{\langle \mathcal{R} (k)  \mathcal{R} (k')\rangle} = (2 \pi )^3 \delta_D(k-k') P_{\mathcal{R}} (k)= (2 \pi )^3 \delta_D(k-k')  \frac{2 \pi^2}{k^3} \mathcal{P}_{\mathcal{R}}(k) =  (2 \pi )^3 \delta_D(k-k')   \frac{2 \pi^2}{k^3} A_s (\frac{k}{k_p})^{n_s-1}
\ee
So we can write
\be
\delta_{\mathcal{R}}(k)= \frac{\sqrt{2 \pi^2 A_s(\frac{k}{k_p})^{n_s-1}}}{k^{3/2}}
\ee
So $\delta_{\mathcal{R}}$, curvature perturbation transfer function, is the normalization factor which we should consider. Precisely we should multiply the field value from the class to $\delta_{\mathcal{R}}(k)$ to go in Gevolution units.\\
On the other hand, in the Gevolution after realization of the field the power spectrum is calculated as following,
\be
\langle \pi(k,t)  \pi(k,t) \rangle =(2 \pi )^3  P_{\pi} ^{Gev}=(2 \pi )^3   \delta_{\mathcal{R}}(k) ^2 P_{\pi}^{\text{hiclass}} = \frac{2 \pi^2}{k^3} A_s (\frac{k}{k_p})^{n_s-1} P_{\pi}^{\text{hiclass}}
\ee
In the Gevolution the wavenumber is measured in $\frac{h}{Mpc}$ but for the purpose of working with normal numbers it is multiplied to Boxsize so;
\be
k  \, \left[\frac{h}{Mpc} \right] = {k \times \text{Boxsize }}\, \left [\frac{{h}}{ \text{Boxsize }Mpc} \right]
\ee
As the powerspectrum in Gevolution in in $Mpc^3$ so $P(k[\frac{h}{Mpc}]) [\frac{Mpc^3}{h^3}]=P(k \, h [\frac{1}{Mpc}]) \left [{Mpc^3} \right ] $
%Plus in the gevolution $\frac{\text{Boxsize}}{Mpc}= \text{Number of grid points}$ 
The dimensionless powerspectrum is calculated as following: {(\color{red}{arXiv:0712.3028v2}}) \\
1- For the boxsize $L$, the field in Fourier space is discrete (no matter it is continuos or discrete in real space) and the k-modes are $\vec{k}= (\frac{2 \pi}{L}) (i,j,k)$. \\
2- The discrete Fourier transform is obtained by placing $\delta (x)$ on a lattice of  $N^3$ grid points with spacing $\frac{L}{N}$ 
\be
\delta_k^{DFT}= \frac{1}{N^3} \sum_r e^{-i\vec{k}.{r}} \delta(\vec{r})
\ee
3- Note that,
\be
\delta_k \approx ( \frac{\Delta x}{2 \pi} )^3 N^3 \delta_k^{DFT} \approx \dfrac{1}{\Delta k} \delta_k^{DFT} 
\ee
3- The powerspectrum is;
\be
P(k) \approx \frac{\langle | \delta^{DFT} |^2 \rangle}{(\Delta k)^3}
\ee
 4- The relation between dimensionless powerspectrum with the power with dimension is as following,
 \be
 \mathcal{P} (k)= \frac{k^3}{2 \pi ^2} P (k)
 \ee
 But factor $\frac{1}{N^6}$ should not be considered when we want to compare initial field transfer function with output powerspectrum! It is something internal in the Gevolution for calculating power from the realization. \\
 
-{\color{red}{Question:}} Why we do not get the same field realization (or order of magnitude) after going forward and backward in Fourier space? \\
- Is the below code right for realizing the field in Fourier and real space?
\begin{lstlisting}
generateRealization(*scalarFT_pi, 0., tk_d_kess, (unsigned int) ic.seed,
 ic.flags & ICFLAG_KSPHERE); 
plan_pi_k ->execute(FFT_BACKWARD);
pi_k->updateHalo();	// pi_k now is realized in real space
plan_pi_k->execute(FFT_FORWARD);}
\end{lstlisting}
If we turn off the following part we get completely different solution for the field in real space! As we expect the same order of magnitude of the $\pi (t,\vec{x})$ as other perturbations field $\phi$ so it seems we should turn off the below command! 
\begin{lstlisting}
plan_pi_k->execute(FFT_FORWARD); 
\end{lstlisting}
\subsection{Gevolution initial field factors from initial condition to output power }
When we give the field transfer function as an initial condition in Gevolution, since in class it is normalized to $\mathcal{R}$ curvature perturbation, first we need to multiply to $\sqrt{P_{\mathcal{R}}}$. On the other hand we need to notice to the units of $k$ in input initial condition and what Gevolution work with internally. \\
What happens in the $icbasic.hpp$,
\be
\pi_{gev} = -\sqrt{2} \times \pi_{numb} \; \; \;\frac{ \pi_{class} \sqrt{\mathcal{P}_{\mathcal{R}}(k/ \text{Boxsize})  }} {k[h/Mpc]^{3/2}}
\ee
Note that the factor $\sqrt{2}$ in the above definition is missed in Gevolution! Actually it is recovered in the output power for other variables. So we keep the convention and use the below formula,
\be
\pi_{gev} = - \pi_{numb} \; \; \;\frac{ \pi_{class} \sqrt{\mathcal{P}_{\mathcal{R}}(k/ \text{Boxsize})  }} {k[h/Mpc]^{3/2}}
\ee
{\color{red} What is the negative sign here?}
\\
Where $\mathcal{P}_{\mathcal{R}}=A_s (\frac{k}{k_p})^{n_s-1}$ in the code internally, $k_p$ is in the unit of $1/Mpc$ and $\mathcal{P}_{\mathcal{R}}$ is dimensionless.
and $\pi_{numb}$ $\pi$ number. \\
In order to know why we divide by $k[h/Mpc]^{3/2}$ we need to look at other part of the code to get what happens...\\
Boxsize in gevolution is $Mpc/h$. So its better to give k in class in $h/Mpc$.
\be
k_{gev}= k_{class} [h/Mpc] \text{Boxsize}[Mpc/h]
\ee
So k when it is read out is dimensionless, equivalently it is in comoving box. \\
Moreover the quantities in class and hiclass are normalized to dimensionful curvature perturbation so the coefficient in Gevolution is,
\be
\delta_{\mathcal{R}}(k)= \frac{\sqrt{2 \pi^2 A_s(\frac{k}{k_p})^{n_s-1}}}{k^{3/2}}
\ee
More over it is very important to note that since $\pi$ in class has dimension $Mpc$ is not enough to divide by Boxsize in Gevolution since it is actually in the unit of $1/H$ where in Gevolution it is divided by c light velocity. To make it consistent we do $\pi \mathcal{H}_{class}/\mathcal{H}_{Gev}$ since $\pi \mathcal{H}_{class}$ is dimensionless in class and we make it in time dimension in Gevolution by dividing to $\mathcal{H}_{Gev}$
\\
\subsection{writePowerSpectrum}
Now we want to extract writePowerSpectrum function where the coefficients are very important. \\
First of all  we have k which is in unit of boxsize must divided by boxsize to give $h/Mpc$ unit. Unit conversion for P (power) depends on the quantity but $\frac{1}{N_{points}^6 \times 2 \pi^2}$ is common. $N_{points}^6$ comes from FFT definition, $2 \pi^2$. \\
{\color{red} Here everything is so confusing! I do not know why $\sqrt{2}$ in the definition of $\delta_{\mathcal{R}}$ is missed, I also do not know why at the end it is not multiplied to $\frac{k^3}{2 \pi ^2}$, also negative sign in the definitions!!! it seems that  something is done internally that I cannot track well!!!}. \\
What I'm doing is following the same notation and then check what is going on and do some consistency checks!!


 \subsection{Stress tensor}
{ \color{red}{ Todo: How to add stress tensor?}} \\
Should  
{\begin{lstlisting}
projection_init(&source); 
\end{lstlisting}}
contains the scalar field in it?\\
- what is source and where it is calculated? \\
- Is it $T_\mu^{\nu}$ or $T_{\mu \nu} $in Gevolution?\\
-what is the meaning of below lines before the functions definitions.
{\begin{lstlisting}
-template<typename part, typename part_info, typename part_dataType>
template <class FieldType>
\end{lstlisting}}
 \subsection{Field equation and transfer}
 { \color{red}{ Todo: use the correct field equation and check the transfer function in time?}}
 \subsection{Some tests on Gevolution:}
 We want to test if the initial powerspectrum is the same as realized field powerspectrum in gevolution. \\
 In the figure \ref{comparehi-gev} the powerspectrum of the field from Gevolution's output is compared with the one which is made by hand. We make the dimensionless powerspectrum by hand from the initial transfer function as following,
 \be
 \mathcal{P}_{\pi}^{Gevolution} (k) = \frac{k ^3}{2 \pi ^2} \; \delta_{\mathcal{R}}^2 \,  \delta_{\pi}
^{\text{hiclass}} \times  \delta_{\pi}^{\text{hiclass}} =  \frac{k ^3}{2 \pi ^2} \;    \frac{2 \pi^2 A_s(\frac{k}{k_p})^{n_s-1}}{k^{3}}  \;  \delta_{\pi}
^{\text{hiclass}} \times  \delta_{\pi}^{\text{hiclass}} =    { A_s(\frac{k}{k_p})^{n_s-1}} \;  \delta_{\pi}
^{\text{hiclass}} \times  \delta_{\pi}^{\text{hiclass}} 
 \ee
 \begin{figure}[htbp!]
\begin{center}
\captionsetup{,margin=1cm}
\includegraphics[width=0.60\textwidth]{Gev-hiclass} 
\caption{The powerspectrum which is calculated by hand is compared with output power of Gevolution. Simulation setting is: Boxsize=200 $Mpc/h$,  N$\_$grid=64 .  There is a deviation between two plots in high wavenumbers { \color{Red}{why?}}. While the Nyqvist frequency approximately is $\frac{2 \pi }{\Delta x} \approx  \frac{2 \pi \times \text{num}-{\text{grids}} }{\text{Boxsize}} =2.01 \, \frac{h}{Mpc}$}
\label{comparehi-gev}
\end{center}
\end{figure}
 Note that to get $H0$ in class unit, we have $H0=\frac{100h \times 10^3}{3 \times 10^8} [1/Mpc]$ where the first $10^3$ is because of $km/s$ and is divided by c. \\
 It is also important to note that, to obtain $\pi$ from $\delta $ and $\theta$ in class, because we have $H \pi_{phys}$ combination and since it is dimensionless we have $\mathcal{H} \pi_{conf} = H \pi_{phys}$. \\
 Also $\theta=-\nabla^2 \pi_{conf}=-\nabla^2 \pi_{phy}/a^2$ and negative sign is the convention. \\
 In Gevolution $H_{0}=\sqrt{8 \pi G/3}$ since $\rho_{crit}^0=1$ and also note that we have $4 \pi G= \frac{3 \text{Boxsize}^2}{2 c^2}$, in order for working with normal numbers internally, and $c[100km/s]$ in the code.
 
% \begin{empheq}[box=\tcbhighmath]{equation}

%\end{empheq}
 \section{Numerical solution to the k-essence equation and stress tensors in Gevolution  ({\color{red} Must be checked!})}
In Gevolution we should use $ M^2_{pl}= 1/8 \pi G$.\\
%And it seems the normalization factor is $-3 \mathcal{H}_0^2T_0^0/8\pi G$ \\
So we have:
%\begin{empheq}[box=\mymath]{equation}
\begin{align}
 & T_0^0 (Gev)=-a^3 {T_{0}^{0}}=   {3 a M_{pl}^2   \mathcal{H}^2\Omega} \Bigg[1+ \frac{1+w}{c_s^2} \Big(- 3 \mathcal{H}c_s^2 \pi +\mathcal{H} \pi- \Psi+   {\pi'}  -  \Big(1-2 c_s^2 \Big) 
 \frac{(\vec{\nabla} \pi)^2}{2} \Big )   \Bigg ]
\nonumber \\ &
T^{i}_{0}(Gev)= {3 a M_{pl}^2   \mathcal{H}^2\Omega} \Big[1+ \pi' +(\frac{1}{c_s^2} -1) \Big(-\Psi +\pi' - \frac{(\vec{\nabla} \pi)^2}{2}  \Big ) \Big ] \partial _i \pi 
\nonumber \\ &
T_{j}^{i}(Gev)= a^3 T_j^i = {3 a M_{pl}^2   \mathcal{H}^2\Omega w} \Bigg ( 1+  \frac{1+w}{w}\Big [ -3 \mathcal{H} w \pi- \Psi + \pi' +\mathcal{H} \pi+  \frac{(\vec{\nabla} \pi)^2}{2}   \Big] \delta_{j}^{i}  + \frac{1+w}{w} \delta^{i k} \partial_k \pi \partial_j \pi  \Bigg) 
\end{align}
%\end{empheq}
Note that $\dot{H}$ is determined by all the matter contents of the universe not by k-essence alone, the continuity equation for k-essence or matter gives the dynamics of density. \\
About the unit of $T^0_0$ note that it is $\bar{\rho}_{kessence} [1+\delta \rho/\bar{\rho} ]$ and since in Gevolution it is multiplied to $-a^3$ and since critical density at redshift zero is 1 so we have\\
%\begin{empheq}[box=\mymath]{equation}
\begin{align}
 & T_0^0 (Gev)=  \Omega^0_{kess} a^{-3 w}  \Bigg[1+ \frac{1+w}{c_s^2} \Big(-  \mathcal{H}(3c_s^2-1) \pi- \Psi+   {\pi'}  -  \Big(1-2 c_s^2 \Big) 
 \frac{(\vec{\nabla} \pi)^2}{2} \Big )   \Bigg ]
\nonumber \\ &
T^{i}_{0}(Gev)=  \Omega^0_{kess} a^{-3 w} \Big[1+ \pi' +(\frac{1}{c_s^2} -1) \Big(-\Psi +  \pi' - \frac{(\vec{\nabla} \pi)^2}{2}  \Big ) \Big ] \partial _i \pi 
\nonumber \\ &
T_{j}^{i}(Gev)=  \Omega^0_{kess} a^{-3 w} \Bigg ( 1+  \frac{1+w}{w}\Big [ -3 \mathcal{H} w \pi- \Psi +  \pi' +  \frac{(\vec{\nabla} \pi)^2}{2}   \Big] \delta_{j}^{i}  + \frac{1+w}{w} \delta^{i k} \partial_k \pi \partial_j \pi  \Bigg) 
\end{align}
%\end{empheq}
In Gevolution we extract $\delta T_0^0/ \bar{T}_0^0$ which scales out the coefficient $\Omega^0_{kess} a^{-3 w} $.


The field equation is:
% \begin{align} 
% &\pi'' - \mathcal{H} \Big (1+ 3w \Big)\pi' -3 {a c_s^2 \mathcal{H}}\Big( 1- \frac{w}{c_s^2} \Big )\Psi -a \, {\Psi'}- 3 c_s^2 a \,{\Phi'} 
%  +3  c_s^2 \Big({-\mathcal{H}^2 + \mathcal{H}'} \Big) \pi 
% - c_s^2 {\nabla^2 \pi }
%     + (1-c_s^2)\pi {\nabla^2 \Psi }
%      - 2 c_s^2 \pi {\nabla^2 \Phi }
%            \nonumber
%   \\
%    &
%      + 3 c_s^2  H (1+w)\pi {\nabla^2 \pi }   
%  -  (1-c_s^2)
%   \pi \frac{\nabla^2\pi ' }{a}   
%   - (2 c_s^2-1) {\nabla  \Psi . \nabla \pi }
% + c_s^2 {\nabla  \Phi . \nabla \pi }  
%   +\frac{\mathcal{H}} {2 a } \Big(2+3w+c_s^2  \Big) \,{\nabla  \pi . \nabla \pi }     =0 
%%  -(\frac{1}{c_s^2}-1) \nabla^2 \Psi \pi+ 2 \nabla^2 \Phi \pi - 3 H (1+w) \pi \nabla^2 \pi  + (\frac{1}{c_s^2}-1) \pi \nabla^2 \dot{\pi}   \nonumber \\ &+ (2-\frac{1}{c_s^2})\nabla \Psi \nabla \pi - \nabla \Phi \nabla \pi -\frac{H} {2 c_s^2} \Big(2+3w+c_s^2  \Big) \nabla \pi \nabla \pi =0
%  \end{align} 

we take $d \tau=\tau_{n+1}-\tau_n $ and $x_{i,j,k} $ as lattice point. We solve the differential equation numerically as following;
\be
\pi_v= {\pi}'
\ee
\be
\pi^{n}= \pi ^{n-1}+\pi_v ^{n-\frac{1}{2}} d \tau
\ee
\be \label{eq3}
\pi_v ^{n+\frac{1}{2}}=\pi_v ^{n-\frac{1}{2}} + {\pi''} ^{(n)}  d \tau
\ee

We define the laplacian in code as following,
\begin{align}
& \nabla^2 \pi =-\frac{\pi^{n}_{i-1,j,k}+\pi^{n}_{i+1,j,k} +\pi^{n}_{i,j-1,k} +\pi^{n}_{i,j+1,k}+\pi^{n}_{i,j,k-1}+\pi^{n}_{i,j,k+1} -6 \pi^{n}_{i,j,k}  }{ a^2 dx^2}  
\end{align}
Moreover in order to get scalar in the vertices the derivatives like $\nabla \pi . \nabla \pi $ should be defined symmetric .
So we can rewrite the equation \ref{eq3} as below;
\begin{align} 
 &\pi_v ^{n+\frac{1}{2}}=\pi_v ^{n-\frac{1}{2}} - d \tau \Big [- \mathcal{H}^{(n)} \Big (1+ 3w \Big)\frac{(\pi_{v  \; {i,j,k}}^{n+\frac{1}{2}} +\pi_{v \; {i,j,k}}^{n-\frac{1}{2}} )}{2} -3 {a c_s^2 \mathcal{H}^{(n)}}\Big( 1- \frac{w}{c_s^2} \Big )\Psi^{(n) }-a^{(n)} \, \frac{{\Psi}^{(n)}-{\Psi}^{(n-1)} }{d \tau}- 3 a^{(n)} c_s^2  \, \frac{{\Phi}^{(n)}-{\Phi}^{(n-1)} }{d \tau}    \nonumber
     \\
      &
   +3  c_s^2 \Big({-\mathcal{H}^{2\, (n)}     + \mathcal{H}'}^{(n)} \Big) \pi^{(n)}   - c_s^2 {\nabla^2 \pi ^{(n)}}  + (1-c_s^2)\pi^{(n)} {\nabla^{2} \Psi^{n} }    
    - 2 c_s^2 \pi^{(n)} {\nabla^2 \Phi ^{(n)}}
     + 3 c_s^2  H (1+w)\pi^{(n)} {\nabla^2 \pi^{(n)} }   
     -  (1-c_s^2)
   \pi^{(n)} \frac{\nabla^2 {(\pi_{v  \; {i,j,k}}^{n+\frac{1}{2}} +\pi_{v \; {i,j,k}}^{n-\frac{1}{2}} )}}{2a}
   \nonumber
     \\
       &
    - (2 c_s^2-1) {\nabla  \Psi^{(n)}  . \nabla \pi ^{(n)} }
    + c_s^2 {\nabla  \Phi ^{(n)} . \nabla \pi^{(n)}  }      +\frac{\mathcal{H}^{(n)}} {2 a^{(n)} } \Big(2+3w+c_s^2  \Big) \,{\nabla  \pi^{(n)} . \nabla \pi^{(n)} }  
    \Big]
\end{align}
The last equation cannot be solved in real space using the discrete lattice. So we try to solve it in Fourier space,
\begin{align} 
 &\pi_v ^{n+\frac{1}{2}} \Big (  1-  { (1+ 3w ) \mathcal{H}^{(n)} } \frac{d \tau }{2} +k^2  (1-c_s^2)
   \pi^{(n)}  \frac{d \tau }{2 a^{(n)}}\Big )
   =
   \pi_v ^{n-\frac{1}{2}} \Big ( 1+{ (1+ 3w ) \mathcal{H}^{(n)} \frac{d \tau }{2}- k^2  (1-c_s^2)
   \pi^{(n)}  \frac{d \tau }{2 a^{(n)}}}    \Big) 
   - d \tau \Bigg [-3 {a c_s^2 \mathcal{H}^{(n)}}\Big( 1- \frac{w}{c_s^2} \Big )\Psi^{(n) }
     \nonumber
     \\
     &
     -a^{(n)} \, \frac{{\Psi}^{(n)}-{\Psi}^{(n-1)} }{d \tau}
 - 3 a^{(n)} c_s^2  \, \frac{{\Phi}^{(n)}-{\Phi}^{(n-1)} }{d \tau}         
   +3  c_s^2 \Big({-\mathcal{H}^{2\, (n)}     + \mathcal{H}'}^{(n)} \Big) \pi^{(n)}   - c_s^2 {\nabla^2 \pi ^{(n)}}  + (1-c_s^2)\pi^{(n)} {\nabla^{2} \Psi^{n} }    
    - 2 c_s^2 \pi^{(n)} {\nabla^2 \Phi ^{(n)}}
        \nonumber
     \\
       &
     + 3 c_s^2  H (1+w)\pi^{(n)} {\nabla^2 \pi^{(n)} }   
         - (2 c_s^2-1) {\nabla  \Psi^{(n)}  . \nabla \pi ^{(n)} }
    + c_s^2 {\nabla  \Phi ^{(n)} . \nabla \pi^{(n)}  }      +\frac{\mathcal{H}^{(n)}} {2 a^{(n)} } \Big(2+3w+c_s^2  \Big) \,{\nabla  \pi^{(n)} . \nabla \pi^{(n)} }  
    \Bigg]
\end{align}
Simplifying the expression we get,
\noindent
%\begin{empheq}[box={\mymath [after=\vspace{0.5cm}]}]{equation}
\begin{align} 
  \pi_v ^{n+\frac{1}{2}} (k)
   = &
   \pi_v ^{n-\frac{1}{2}} (k) \Bigg [ 1+{ (1+ 3w ) \mathcal{H}^{(n)} {d \tau }- k^2  (1-c_s^2)
   \pi^{(n)}  \frac{d \tau }{a^{(n)}}}    \Bigg]
   - d \tau \Bigg ( 1+{ (1+ 3w ) \mathcal{H}^{(n)} \frac{d \tau }{2}- k^2  (1-c_s^2)
   \pi^{(n)}  \frac{d \tau }{2 a^{(n)}}}    \Bigg)        \nonumber
     \\
       &
       \Bigg [-3 {a c_s^2 \mathcal{H}^{(n)}}\Big( 1- \frac{w}{c_s^2} \Big )\Psi^{(n) }
     -a^{(n)} \, \frac{{\Psi}^{(n)}-{\Psi}^{(n-1)} }{d \tau}
 - 3 a^{(n)} c_s^2  \, \frac{{\Phi}^{(n)}-{\Phi}^{(n-1)} }{d \tau}         
   +3  c_s^2 \Big({-\mathcal{H}^{2\, (n)}    
    + \mathcal{H}'}^{(n)} \Big) \pi^{(n)}  
          \nonumber
     \\
       &
     - c_s^2 {\nabla^2 \pi ^{(n)}}  + (1-c_s^2)\pi^{(n)} {\nabla^{2} \Psi^{n} }    
    - 2 c_s^2 \pi^{(n)} {\nabla^2 \Phi ^{(n)}}
     + 3 c_s^2  H (1+w)\pi^{(n)} {\nabla^2 \pi^{(n)} }   
         - (2 c_s^2-1) {\nabla  \Psi^{(n)}  . \nabla \pi ^{(n)} }
              \nonumber
     \\
       &
    + c_s^2 {\nabla  \Phi ^{(n)} . \nabla \pi^{(n)}  }      +\frac{\mathcal{H}^{(n)}} {2 a^{(n)} } \Big(2+3w+c_s^2  \Big) \,{\nabla  \pi^{(n)} . \nabla \pi^{(n)} }  
    \Bigg]
\end{align}
\noindent
%\end{empheq}
Note that on discrete lattice $k^2$ is defined as,
\begin{align} 
& k^2=- \Big(\frac{4}{dx^2} \sin^2(\pi k_x/L)+\frac{4}{dx^2} \sin^2(\pi k_y/L)+\frac{4}{dx^2} \sin^2(\pi k_z/L) \Big ) \nonumber \\
&
2\sin^2x=1-\cos(2x)
\end{align}
 We have taken $\pi_{v  \; {i,j,k}}^{n} =\frac{(\pi_{v  \; {i,j,k}}^{n+\frac{1}{2}} +\pi_{v \; {i,j,k}}^{n-\frac{1}{2}} )}{2} $. Then we need to calculate $\mathcal{H}'$, ${\Psi}'$ and  ${\Phi}'$ in each loop, to calculate ${\Psi}'$ we save two $\Psi$ in each loop. \\
 On the other hand we have $\mathcal{H}'$ according to the Friedman equation, where we try to save $\mathcal{H}'$ from $a''$  and $\mathcal{H}$
%Gevolution works with conformal time $\tau$ and light velocity equal to one $c=1$, which imposes 
\section{Programming issues}
One the programming issue is related to the fact that we want to separate background updates of the particles with background updates of the k-essence field. We define another scale factor in the code "a-kess" which takes the value of the scale factor, then it updates the field. In order to match the two background scale factor, we should notice that we update the background after each updating the field and the velocity of the field. Now the only point is that the updated field using the background value in the last half step, so the question is whether this procedure is leap frog or not? {\color{red} To me is not clear enough so to update field by one step we use the value of the back ground in the half step while we must use the value in the last step (n-1)}  \\
The procedure is explained as following,
\begin{lstlisting}
if(cycle==0)
{
	update_pi_k_v( 0.5 * dtau);
	rungekutta4bg(a_kess, fourpiG, cosmo,  0.5 * dtau  );
	pi_k.updateHalo();
	pi_v_k.updateHalo();
}

else
{
	for (i=0;i<sim.nKe_numsteps;i++)
	{
		update_pi_k( dtau  / sim.nKe_numsteps);
		rungekutta4bg(a_kess, fourpiG, cosmo,  0.5*dtau  / sim.nKe_numsteps)
		update_pi_k_v( dtau  / sim.nKe_numsteps);
		rungekutta4bg(a_kess, fourpiG, cosmo,  0.5*dtau  / sim.nKe_numsteps);
		pi_k.updateHalo();
		pi_v_k.updateHalo();
	}
}
\section{Notes_Class}
It is important that we need large bound of wavenumbers. So we set it by $k_scalar_k_per_decade_for_pk$ in the class to get more number of wavenumbers. \\
Moreover notice that the initial file for inputing to Gevolution the wavenumber is in $1/Mpc$ unit, so in Gevolution we must notice this fact and take it into account actually in Gevolution we need to multiply to $h/Sizebox$ which is done already.
%
\end{lstlisting}
 
 
\section{Do we obtain the same results for the $\pi$ and $\pi$ transfer function from class in synchronous gauge? Miguel question!}
\be
\delta_{newt}=\frac{\rho- \bar{\rho}}{\bar{\rho}}=\frac{1+w}{c_s^2} \Big ( -3 c_s^2 \mathcal{H} \pi_{newt}-\Psi_{newt}+\pi' _{newt} + \mathcal{H} \pi_{newt} \Big )
\ee
%Using the equation \ref{deltarho} we can get the density contrast as following,
%\be
%\delta=\frac{\rho- \bar{\rho}}{\bar{\rho}}=\frac{1+w}{c_s^2} \Big ( -\Psi+\pi' -  \frac{\vec{(\nabla} \pi)^2}{2} \Big )
%\ee
where $\bar{\rho}= \frac{\bar{P}}{w}$. The last relation shows that density contrast transfer function is obtained from time derivative of the field and the metric not the field . It is a good test to check that we can recover density contrast transfer function from field derivative and metric $\Psi$ transfer functions. The transfer function is defined as below:
\be
T_{kessence}(k,z_i)=\frac{\delta_{kessence} (k,z=z_i)}{\mathcal{R}(k,z=\infty)}
\ee
The field equations are written in Synchronous gauge in the hi-class while we have calculated density contrast transfer function in the Newtonian gauge. So wee need to calculate density contrast transfer function in Newtonian gauge in terms of Synchronous gauge variable as following,
\begin{align}
\delta_{Newt}&=\frac{1+w}{c_s^2} \Big (  -  (3c_s^2 -1)\mathcal{H} \pi_{newt}-\Psi_{newt}+\pi'_{newt} \Big )= \frac{1+w}{c_s^2} \Big ( -(3c_s^2 -1)\mathcal{H} (\pi_{synch} + \alpha)-\alpha ' - \mathcal{H} \alpha+\pi'_{synch} +\alpha' \Big ) \\ \nonumber
&
= \frac{1+w}{c_s^2}\Big (-(3c_s^2-1) \mathcal{H} \pi_{synch}- 3c_s^2 \mathcal{H} \alpha +\pi'_{synch} \Big )
\nonumber \\ 
&
= \frac{1+w}{c_s^2}\Big (-(3c_s^2-1) \mathcal{H} \pi_{synch} +\pi'_{synch} \Big )- 3 (1+w) \mathcal{H} \alpha
\nonumber \\ 
&
= \frac{1+w}{c_s^2}\Big (-(3c_s^2-1) \mathcal{H} \pi_{synch} +\pi'_{synch} \Big )+ \frac{{\bar{\rho'}}}{\bar{\rho}}\alpha
\nonumber \\ 
&
=\delta_{Syn} + \alpha \frac{{\bar{\rho'}}}{\bar{\rho}}
\label{deltaeq}
\end{align}
So we have,
\be
\delta_{Synch} =\frac{1+w}{c_s^2} \Big (  -  (3c_s^2 -1)\mathcal{H} \pi_{synch}+\pi'_{synch} \Big )
\ee
where $\alpha=({h'} + 6 {\eta'})/2 k^2$, $\Psi=\frac{1}{2 k^2} \Big [ {h^{''}}+ 6 {\eta^{''}} + \frac{a'}{a}  (h'+6 \eta ') \Big]= {\alpha' + \mathcal{H}\alpha.}$  according to equation 18 of  {\color{blue}{https://arxiv.org/pdf/astro-ph/9506072.pdf}} \\
Here we want to compare the calculated $\pi_{Newt}$ from Synchoronous gauge of class with the one we get and validated from Newtonian gauge of class.\\ 
According to the background equations ${\rho'} + 3 \mathcal{H} (\rho + w \rho)=0 $ for constant $w$ we have,
\be
\bar{\rho}_{kessence} (a) =a ^{-3(1+w)}
\ee
so we can write,
\be
 \frac{{\bar{\rho'}}}{\bar{\rho}}= -3 (1+w) \mathcal{H}
 \ee

%\begin{figure}[htbp!]
%\begin{center}
%\captionsetup{,margin=1cm}
%\includegraphics[width=0.60\textwidth]{alphaerror} 
%\caption{The  relative error of $\alpha$ in class and hi-class is shown. It is clear that  the relative error is very large in high wavenumbers.}
%%\label{plt2}
%\end{center}
%\end{figure}
%So we have,
%\be
%\delta_{Newt}= \frac{1+w}{c_s^2} \Big ( -c_s^2 \mathcal{H} (\pi_{synch} + \alpha)- \mathcal{H} \alpha+\pi'_{synch}  \Big )
%\ee
%\be
%\delta_{Synch}= \delta_{newt} + 3 \mathcal{H} (1+w) \alpha
%\ee
%which results,
%\be
%\delta_{synch}= \frac{1+w}{c_s^2} \Big ( -c_s^2 \mathcal{H} (\pi_{synch} + \alpha)- \mathcal{H} \alpha+\pi'_{synch}  \Big )
%\ee

\subsection{Comparing the  ${\pi}$ transfer function directly from hi-class and what we get from class indirectly}
From the equation for $\theta$, we can obtain $\pi$ in Fourier space,
\be
\pi_{newt}=\frac{\theta_k\text{(newt)}}{k^2}  \; \; \; \text{(class)}
\ee
while in the hi-class we can easily obtain $\pi_{newt}$ by gauge transformation as following,
\be
\pi_{newt}= \pi_{synch}+ \alpha
\ee
We can easily observe that according to the gauge transformation in Ma and Bertchinger,
\be
\theta_{synch}= \theta_{newt} -\alpha k^2
\ee
so we can write,
\begin{align}
\theta_{synch} & = \pi_{new} k^2 -\alpha k^2 \nonumber \\&
=(\pi_{newt} - \alpha) k^2
 \nonumber \\&
= \pi_{synch} k^2
\end{align}
so we also have,
\be
\theta_{synch} = \pi_{synch} k^2
\ee
The we compare the result with what we get from the hi-class after gauge transformation as following,
\be
\pi'_{newt} =\pi'_{sync}+\alpha'
\ee
 According to all formulas we have the below relations for both class and hiclass codes to get $\pi$, $\pi'$  in both Newtonian and Synchoronous gauges.
\[ \text{Hiclass(Synchronous)} :
  \begin{cases}
    \pi_{synch} = \text{output of the code} \\
    \pi'_{synch}  = \text{output of the code} \\ 
     \theta_{synch} = \pi_{synch} k^2 \\
      \delta_{synch}  = \frac{1+w}{c_s^2} \Big (  -  (3c_s^2 -1)\mathcal{H} \pi_{synch}+\pi'_{synch} \Big )
  \end{cases}
\]
\\
\[ \text{Hiclass (Newtonian)}: 
  \begin{cases}
     \pi_{newt}=\pi_{synch}+ \alpha \\
   \pi'_{newt}=\pi'_{synch}+ \alpha' \\
    \theta_{newt} = \pi_{synch} k^2 + \alpha k^2 \\
      \delta_{newt}  = \frac{1+w}{c_s^2} \Big (  -  (3c_s^2 -1)\mathcal{H} \pi_{synch}+\pi'_{synch} \Big ) - 3(1+w) \mathcal{H} \alpha
  \end{cases}
\]
\\
\[ \text{Class(Newtonian)}: 
  \begin{cases}
     \pi_{newt}= \theta_{newt}/k^2 \\
   \pi'_{newt}= \frac{c_s^2}{1+w} \delta_{newt} + (3c_s^2 -1) \mathcal{H} \theta_{newt}/k^2 + \Psi_{newt}  \\
    \theta_{newt} = \text{output of the code}  \\
     \delta_{newt} = \text{output of the code} 
  \end{cases}
\]
\\
\[ \text{Class(Synchoronous)}: 
  \begin{cases}
     \pi_{synch}= \theta_{synch}/k^2 \\
   \pi'_{synch}= \frac{c_s^2}{1+w} \delta_{synch} + (3c_s^2 -1) \mathcal{H} \theta_{synch}/k^2 \\
    \theta_{synch} = \text{output of the code}  \\
     \delta_{synch} = \text{output of the code} 
  \end{cases}
\]
Now we are going to compare the results obtained from these codes. Just note that to get $\pi'_{newt}$ from class-Synchronous we need to have $\alpha'$ in terms of other quantities, but we did not do the calculation!
\subsection{Comparison of the results}
Note that the value of $\pi$ is negative, so in the plots we plot $-\pi$. To compare the class results in Newtonian gauge and Synchoronous gauge we need to have $\alpha$, which is accessible  from hi-class or class (Synchoronous gauge) internally. \\
Where $\alpha=({h'} + 6 {\eta'})/2 k^2$, $\Psi=\frac{1}{2 k^2} \Big [ {h^{''}}+ 6 {\eta^{''}} + \frac{a'}{a}  (h'+6 \eta ') \Big]= {\alpha' + \mathcal{H}\alpha.}$  according to equation 18 of  {\color{blue}{https://arxiv.org/pdf/astro-ph/9506072.pdf}} \\
Here we want to compare the calculated $\delta _{Newt}$ from the field and metric perturbations in the hi-class with the class output in both Newtonian and Synchronous gauge.\\ 
According to the background equations ${\rho'} + 3 \mathcal{H} (\rho + w \rho)=0 $ for constant $w$ we have,
\be
\bar{\rho}_{kessence} (a) =a ^{-3(1+w)}
\ee
so we can write,
\be
 \frac{{\bar{\rho'}}}{\bar{\rho}}= -3 (1+w) \mathcal{H}
 \ee
First we compare  $\frac{(\delta_{Newt} - \delta_{Syn})}{-3 (1+w) \mathcal{H}}$ with $\alpha$ that we read from the hi-class. Note that according to the code units $\mathcal{H}(z=0)=H(z=0)=2.25 \times10^{-4} /Mpc $. 
\\
 In the below figure we have compared the $\alpha$ in hi-class with what we get from the calculation in the class. These two quantities agree well for the limit $c_s^2 \to 1$, while for non unity sound speed we get the second plot! \\
 The python part to do the comparison is as following,
\begin{lstlisting}[language=Python]
# H_0 in Gevilution unit.
def Hubble_conf_Mpc(a):
    H0=0.00022593979933110373;w=-0.9;h=0.67556;
    Omega_b=0.022032/h/h; Omega_cdm=0.12038/h/h;
    Omega_m=Omega_b+Omega_cdm; Omega_Lambda=0.0;
    Omega_rad=9.16681e-05; Omega_kessence=1.-Omega_m-Omega_rad;
    return H0*np.sqrt(Omega_m*(a**-3)+Omega_rad*(a**-4)+Omega_Lambda+Omega_kessence*(a**(-3*(1+w))))*a
w=-0.9;
#################################
#z=100
alpha_class_z100=(class_sync_z100[:,4]-class_newt_z100[:,4])/(3.*(1+w)*Hubble_conf_Mpc(1./(1.+100.)));
alpha_hiclass_z100=hiclass_z100[:,3]
#z=0
alpha_class_z0=(class_sync_z0[:,4]-class_newt_z0[:,4])/(3.*(1+w)*Hubble_conf_Mpc(1./(1.+0.)));
alpha_hiclass_z0=hiclass_z0[:,3]
#################################
plt.figure(figsize=(18,12))
ax = plt.gca()
ax.tick_params(axis = 'both', which = 'major', labelsize = 10)
ax.tick_params(axis = 'both', which = 'minor', labelsize = 6)
plt.figure(1)
#################################
plt.loglog(class_newt_z100[:,0], alpha_class_z100[:],color="blue",linestyle='dashed',lw=1.5,label=r"$\alpha = \frac{\delta_{kess}(Synch) -\delta_{kess}(Newt) }{3 \mathcal{H} (1+w)}$, z=100 ")
plt.loglog(hiclass_z100[:,0]/h, alpha_hiclass_z100[:],color="red",linestyle='dashed',lw=1.5,label=r"$\alpha$ hiclass, z=100 ")
plt.loglog(class_newt_z0[:,0], alpha_class_z0[:],color="Green",linestyle='dashed',lw=1.5,label=r"$\alpha = \frac{\delta_{kess}(Synch) -\delta_{kess}(Newt) }{3 \mathcal{H} (1+w)}$, z=0 ")
plt.loglog(hiclass_z0[:,0]/h, alpha_hiclass_z0[:],color="Black",linestyle='dashed',lw=1.5,label=r"$\alpha$ hiclass, z=0 ")

#####################
plt.title(r"$w=-0.9$, $c_s^2=10^{-6}$")
plt.legend(bbox_to_anchor=(0.7, 0.90, 0.3, .102), loc=1,ncol=1,fontsize=13, mode="expand", borderaxespad=0.)
plt.xlabel("k[h/Mpc]",fontsize=14)
plt.ylabel(r"$\alpha$",fontsize=14)
# plt.xlim(0.01,5.e-1)
# plt.ylim(1.e-13,1.e-9)
plt.grid(True)
#################################
#Difference plot!

plt.savefig('class-hiclass_alpha.jpg', format='jpg', dpi=500)
plt.show()
\end{lstlisting}


\begin{figure}[H]
\begin{center}
\captionsetup{,margin=1cm}
\includegraphics[width=0.60\textwidth]{alpha_class} 
\caption{The  $\alpha$ is in hi-class and class is compared. As it is clear they agree well for the limit $c_s^2 \to 1$ . }
%\label{f1}
\end{center}
\end{figure}
\begin{figure}[H]
\begin{center}
\includegraphics[scale=0.4]{class-hiclass_alpha.jpg} 
%\label{f1}
\end{center}
\end{figure}
%\begin{figure}[htbp!]
%\begin{center}
%\captionsetup{,margin=1cm}
%\includegraphics[width=0.60\textwidth]{alphaerror} 
%\caption{The  relative error of $\alpha$ in class and hi-class is shown. It is clear that  the relative error is very large in high wavenumbers.}
%%\label{plt2}
%\end{center}
%\end{figure}
So we have,
\be
\delta_{Newt}= \frac{1+w}{c_s^2} \Big ( -\mathcal{H} (3c_s^2-1)  (\pi_{synch} + \alpha)- \mathcal{H} \alpha+\pi'_{synch}  \Big )
\ee
\be
\delta_{Synch}= \delta_{newt} + 3 \mathcal{H} (1+w) \alpha
\ee
We  also compare the metric fluctuation $\Psi$  in Newtonian gauge with one we obtain from Synchronous gauge parameters in class and hi-class codes to make sure we are dealing with the same functions. The $\Psi$ in Synchronous gauge reads as following,
\be
\Psi=\frac{1}{k^2} \Big [ {h^{''}}+ 6 {\eta^{''}} + \frac{a'}{a}  (h'+6 \eta ') \Big]= \alpha' + \mathcal{H}\alpha
\ee
In the figure \ref{psicomp} we compare the $\Psi$ in Newtonian gauge from what we get in terms of Synchronous gauge quantities.
\begin{figure}[H]
\begin{center}
\captionsetup{,margin=1cm}
\includegraphics[width=0.60\textwidth]{psi_comp.jpg} 
\caption{$\Psi$ in Newtonian gauge and synchronous gauge in class and hi-class are compared. They all agree for $c_s^2  \to 1$. }
\label{psicomp}
\end{center}
\end{figure}
\begin{figure}[H]
\begin{center}
\includegraphics[scale=0.4]{class-hiclass_psi.jpg} 
%\label{f1}
\end{center}
\end{figure}
The python code for $\Psi$ comparison is as following,
\begin{lstlisting}[language=Python]
def Hubble_conf_Mpc(a):
    H0=0.00022593979933110373;w=-0.9;h=0.67556;
    Omega_b=0.022032/h/h; Omega_cdm=0.12038/h/h;
    Omega_m=Omega_b+Omega_cdm; Omega_Lambda=0.0;
    Omega_rad=9.16681e-05; Omega_kessence=1.-Omega_m-Omega_rad;
    return H0*np.sqrt(Omega_m*(a**-3)+Omega_rad*(a**-4)+Omega_Lambda+Omega_kessence*(a**(-3*(1+w))))*a
w=-0.9;
#################################
#z=100
Psi_newt_class_z100=class_newt_z100[:,8];
Psi_synch_class_z100=class_sync_z100[:,8];
Psi_hiclass_z100_direct=Hubble_conf_Mpc(1./(1.+100.))*hiclass_z100[:,3]+hiclass_z100[:,4]
# Psi_hiclass_z100_indirect=hiclass_z100[:,3]
#z=0
Psi_newt_class_z0=class_newt_z0[:,8];
Psi_synch_class_z0=class_sync_z0[:,8];
Psi_hiclass_z0_direct=Hubble_conf_Mpc(1./(1.+0.))*hiclass_z0[:,3]+hiclass_z0[:,4]
#z=100
plt.loglog(class_newt_z100[:,0], Psi_newt_class_z100[:],color="blue",linestyle='dashed',lw=1.5,label=r"$\Psi$, Class, Newtonian, z=100 ")
plt.loglog(class_sync_z100[:,0], Psi_synch_class_z100[:],color="red",linestyle='dashed',lw=1.5,label=r"$\Psi$ Class, Synchronous, z=100 ")
plt.loglog(hiclass_z100[:,0]/h, Psi_hiclass_z100_direct[:],color="Green",linestyle='dashed',lw=1.5,label=r"$\Psi = \alpha' + \mathcal{H} \alpha$, hiclass, z=100 ")
\end{lstlisting}
At the end because of tensions between class and hiclass in $\alpha$ we get different $\pi$ and $\pi'$! Now we want to compare the results of class in Synchronous gauge and Newtonian gauge.
\subsubsection{Class, comparison of two gauges}
Here we compare class for two different gauges according to below python script,
\begin{lstlisting}[language=Python]
# H_0 in Gevilution unit.
def Hubble_conf_Mpc(a):
    H0=0.00022593979933110373;w=-0.9;h=0.67556;
    Omega_b=0.022032/h/h; Omega_cdm=0.12038/h/h;
    Omega_m=Omega_b+Omega_cdm; Omega_Lambda=0.0;
    Omega_rad=9.16681e-05; Omega_kessence=1.-Omega_m-Omega_rad;
    return H0*np.sqrt(Omega_m*(a**-3)+Omega_rad*(a**-4)+Omega_Lambda+Omega_kessence*(a**(-3*(1+w))))*a
#################################
# pi_newt in class from Newtonian gauge according to: \theta_kess/k^2 in correct units which is negative!
pi_classNewt_cs_e3_newt_z100=(class_newt_z100[:,4]/((class_newt_z100[:,0]*h)**2) );
alpha_class_z100=alpha_class_z100=(class_sync_z100[:,4]-class_newt_z100[:,4])/(3.*(1+w)*Hubble_conf_Mpc(1./(1.+100.)));
pi_classSynch_cs_e3_newt_z100=(class_sync_z100[:,4]/((class_sync_z100[:,0]*h)**2) )+alpha_class_z100[:];

#z=0
pi_classNewt_cs_e3_newt_z0=(class_newt_z0[:,4]/((class_newt_z0[:,0]*h)**2) );
alpha_class_z0=alpha_class_z0=(class_sync_z0[:,4]-class_newt_z0[:,4])/(3.*(1+w)*Hubble_conf_Mpc(1./(1.+0.)));
pi_classSynch_cs_e3_newt_z0=(class_sync_z0[:,4]/((class_sync_z0[:,0]*h)**2) )+alpha_class_z0[:];
#pi_newt in hiclass is \pi_synch + \alpha; The columns in the hiclass file are k,\pi_synch,pi'_synch,alpha,alhpha',psi 
# pi_hiclass_cs_e3_newt=hiclass_z100[:,1]-hiclass_z100[:,3];
plt.loglog(class_newt_z100[:,0], np.abs(pi_classNewt_cs_e3_newt_z100[:]),color="blue",linestyle='dashed',lw=1.5,label=r"$\pi_{newt}=\frac{\theta_{newt}}{k^2}$, Class-Newtonian, z=100 ")
plt.loglog(class_sync_z100[:,0], np.abs(pi_classSynch_cs_e3_newt_z100[:]),color="green",linestyle='dashed',lw=1.5,label=r"$\pi_{newt}=\frac{\theta_{synch}}{k^2} + \alpha$, $\alpha = \frac{\delta_{kess}(Synch) -\delta_{kess}(Newt) }{3 \mathcal{H} (1+w)}$ class-Synchronous, z=100 ")
\end{lstlisting}
which results,
\begin{figure}[H]
\begin{center}
\includegraphics[scale=0.5]{comp_field_class.jpg} 
%\label{f1}
\end{center}
\end{figure}
Now the questions are:\\
{\color{red} Why the IC of the two gauges are different?
\\
Why at high k we get different results?}





%%%%%%%%%%%%%%%%%%
%%%%%%%%%%%%%%%%%%
%%%%%%%%%%%%%%%%%%

  
  
  
  
  \section{Report-05April2018}
 - Today I've tried to get $P_{22}$ powerspectrum, for Riess Sciama effect.\\
  I tried, mathematica (Matt mathematica) which did not work because of Nintegration,\\
   I tried old version of class which loop correction to density power is implemented, but it seems the integration is not taken by high precision, so we dont get the right difference! \\
 I also tried FnFast, which is not documented at all, so I could not use it! \\
 Now I want to check Kumatsu's routines which seems good to me, well documented and etc. I can also check CAMB to see if they have implemented it ... \\
 Then I need to calculated powaer of $\Psi'$ according to Riess Sciama effect and compare it with Gevolution! and report the result to all collaborators. \\
 I also could match Class and my mathematica code result, which is intresting, so I need to report it as well! \\
 -After believing the Gevolution and field equation, we need to agree on Matter powerspectrum and stress tensors in class, mathematica code and Gevolution! \\
 - When everything is fixed we must add second order corrections! Compare them with first order terms and .... \\
 - Check the error from predictor -corrector method!... \\
 -Then we need to report about all the results and think for future works and results! \\
 -Also we can think about the question: Is Riess Sciama effect is relevant for scalar field? how much? although it is subpercent in CMB, is it the same in field power?
 -We know how the matter behaves:\\
in $cs^2=w?>0$ the evolution should be the same as GEvolution matter!!!\\
-http://www2.iap.fr/users/pitrou/cmbquick.htm\\
-I check$ \Phi_dot$, if we add to$ \phi_old $we get$ \Phi_new$!\\
-For limit $cs and w?>0$ check if it looks like matter? If the filed is differet, maybe class is written differently! but if$ \delta$ is the smae as matter or non-linear matter then it is ok otherwise there should be a problem.\\
-Our at conversion from $\delta $and $\theta$ are linear, \\
- Do the stupid phi test, get phi at redshift 100 and add $phi_dot$  and check we get at z=50\\
-I must write what tests I did and what are the results! \\ 
-Get the field equation, why at linear order for $w=cs^2=0$ \\
eq.2 and 3 of Domenico paper and Martin, why we get 0 evolution when $\pi$ and $\pi_dot$ is zero but in Domenico it is generated!! \\
- Just check that in matter case for kessence we get sinsible results! \\
Check that also in $\delta T_00$ we get good behaviour in first order perturbation theory! \\
\subsection{Martin meeting}
How much is the $\Phi$ error in GEv and class ? Plot? \\
-The error Poisson equation in class and GEvolution, fix the parameters?
- Use Kumatsu code in two redshift and compute Poisson equation and get $\Phi$ power at two different redshift! and then see $\Phi'$ and cross check by Seljak formula. \\
Use their model in the paper (8) $\\url{0809.4488.pdf} $ to do the same analysis!
\\
-If we assume $\Phi'$ in Gevolution correct! Solve the Euler and Continuity equation  in mathematica with and without $\Phi'$ from Gevolution and see if it is sensitive to it or not  and compare with $\delta_m$ from class at the same redsift and then put it into the Gevolution to check versus our result.


% \begin{figure}[H]
% \includegraphics[scale=0.3]{IMG_3589.jpg} 
% \end{figure}
-I've tried to get non linear $\dot{\Psi}$ from CMBquick but it seeems the integration is not provided. It only gives, the transfer function for the configuration ($k_1$, $k_2$ and $\mu$). \\
As Cyril suggested I'm gonna try SONG (contact Christian Fidler), but if I could not get a good result, I'm gonna integrate myself or I'll use Joyce code! If nothing has worked I'm gonna try to compare $\dot{\Phi}$ in Gadget! and compare with Gevolution!

\section{Check the equations  for $c_s^2 \to 0$ and $ w \to 0$  to get the matter behaviour and cross check with class!}
Todo:  \\
-Writing down the equaton in this limit, and compare with fluid approximated equations, if we get the same thing? \\
-Plot $\delta_m$ in class versus $\delta_{kess}$ to check if we get the same behaviour! \\
-Plot $\delta_m$ in the Gev versus $\delta_{kee}$ to check again! Run Gev with the relevant IC!\\
- Add $\dot{\Phi}$ to $\Phi$ in redshift $z=100$ in class and check if we get the same in the redshift $z=50$ and the same thing in Gevolution as a check! \\
-Get the field equation, why at linear order for w = $cs_2$ = 0 eq.2 and 3 of Domenico paper and Martin, why we get 0 evolution when $\pi$ and $\dot{\pi}$ is zero but in Domenico it is generated!! \\
- Just check that in matter case for kessence we get sinsible results! Check that also in $\delta T_{0}^0$ we get good behaviour in first order perturbation theory!\\
-Use Riess Sciama formula to get $P_{\dot{\Phi}}$ and compare it with class and Class! \\
In the Gevolution for IC, we use $\pi_{class} \frac{H_{class}}{H_{gev}}$, so as an input we need to give $\pi$ and $\pi'$ from the class!  which is obtained by the output file!
\subsection{Class and Gevolution. results, $w \longrightarrow0$, $c_s^2 \longrightarrow0$: in theory,  both equations and $T_{\mu \nu}$ } 
%\end{empheq}
 \begin{align} 
 &{ \pi''+\mathcal{H}(1- 3w) \pi' } +3 {  \mathcal{H}}\Big( -c_s^2+ {w} \Big )\Psi - \, {\Psi'}- 3 c_s^2  \,{\Phi'} + {
 \Big( 3\mathcal{H}^2 (c_s^2 -w) + \mathcal{H}' (1-3c_s^2)\Big) \pi }
           \nonumber
   \\
    &
 - c_s^2 {\nabla^2 \pi} =0
    % Second order terms==0
  \end{align} 
\begin{align}
 & T_0^0 (Gev)=  \Omega^0_{kess} a^{-3 w}  \Bigg[1+ \frac{1+w}{c_s^2} \Big(- 3 \mathcal{H}c_s^2 \pi- \Psi+   {({\pi'}+ \mathcal{H} \pi) }    \Big )   \Bigg ]
\nonumber \\ &
T^{i}_{0}(Gev)= - \Omega^0_{kess} a^{-3 w} (1+w) \partial _i \pi 
\nonumber \\ &
T_{j}^{i}(Gev)= w  \, \Omega^0_{kess} a^{-3 w} \Bigg ( 1+  \frac{1+w}{w}\Big [ -3 \mathcal{H} w \pi- \Psi +   {({\pi'}+ \mathcal{H} \pi) }\Big] \delta_{j}^{i}   \Bigg) 
\end{align}
Just to observe, its interesting to notice the relation between $c_s^2$ with $\delta P/\delta \rho$ for scalar field, {\color{red} derive it? Is there any inconsistency? why dont get simply $c_s^2$?}
\be
\frac{\delta P}{\delta \rho}= \frac{1+w) \Big [ -3 \mathcal{H} w \pi- \Psi +   {({\pi'}+ \mathcal{H} \pi) }\Big] } { \frac{1+w}{c_s^2} \Big(- 3 \mathcal{H}c_s^2 \pi- \Psi+   {({\pi'}+ \mathcal{H} \pi) }    \Big ) } = 
\ee
In the limit  $w = c_s^2 \rightarrow 0$, we end up with,
\be
\pi''+\mathcal{H}\pi'     -  {\Psi'} 
 + \mathcal{H}'  \pi  = 0 \label{fieldeq}
\ee
Comparing with other results: like eq. B.22 of {\url{arXiv:1611.07966v2}} we see that in the limit of $w \rightarrow 0$ we have (according to the equation in the paper):
\be
\ddot{\pi}_{phys} - \dot{\Phi} =0
\ee
we know that in their notation $\Phi$ is our $\Psi$, moreover our equation is written for $\pi$ in constant conformal time hypersurfaces and the derivatives are taken based on conformal time! \\Applying the relation for $\pi_{phys}$ we have:
\be
\ddot{\pi}_{phys} = \frac{\mathcal{H} \pi_{c}' +\mathcal{H}' \pi_{c} + \pi_{c}'' }{a}
\ee
index "c" refers to conformal time! More over we have:
\be
\dot{\Phi}=\Phi'/a
\ee
So we recover our equation for this limit! \\
It is noteworthy that we have checked the complete linear equation versus the result obtained by Filippo's paper and Iggi et. al paper!
\begin{align}
 & T_0^0 (Gev)=  \Omega^0_{kess}  \Bigg[1+ \frac{1}{c_s^2} \Big( -\Psi+   { ({\pi'}+ \mathcal{H} \pi) }\Big )   \Bigg ]
\nonumber \\ &
T^{i}_{0}(Gev)= - \Omega^0_{kess} \partial _i \pi 
\nonumber \\ &
T_{j}^{i}(Gev)=   \, \Omega^0_{kess}  \Bigg ( - \Psi +    ({\pi'}+ \mathcal{H} \pi)   \Bigg) 
\end{align}
According to the stress tensor in this limit we have,
\be
\delta= \frac{1}{c_s^2} \Big( -\Psi+   { ({\pi'}+ \mathcal{H} \pi) }\Big ) 
\ee
\be
u_i=\partial _i \pi 
\ee
Which is basically the same as eq.3.12 of  {\url{arXiv:1611.07966v2}}! \\
\subsection{Solving the equation:}
Before solving the equation we can observe that:
\be
\pi'+\mathcal{H} \pi -\Psi \sim c_s^2 \partial^2 \Psi/ \mathcal{H}^2 
\label{eqsenatore}
\ee
or equivalently,
\be
\dot{\pi}_{phys} - \Psi  \sim c_s^2 \partial^2 \Psi/ {H}
\ee
To observe this relation it is better to look at the equation in terms of physical time which according to eq.3.9 of  {\url{arXiv:1611.07966v2}} is as following,
\be
\frac{1}{a^3 M_2^3} \frac{d}{dt} \Big[a^3 M_2^4 (\dot{\pi} -\Psi)\Big] = c_s^2 a^{-2} \partial^2 \pi 
\ee
Since we are in the limit $c_s^2 \rightarrow 0$ we expand the scalar field in terms of sound speed $\pi= \pi_0 + \pi_{,c_s^2} c_s^2$. Plugging into the equation we get,
\be
\dot{\pi}_0 = \Psi
\ee
and 
\be
\frac{1}{a^3 M_2^3} \frac{d}{dt} \Big[a^3 M_2^4 (\dot{\pi_{,c_s^2}} )\Big]   a^{-2} \partial^2 \pi_0 \sim a^{-2} H ^{-1} \partial^2 \Psi
\ee
Where we have taken that time derivatives to be of order $H$ . So we have used $\pi_0 \sim H^{-1} \Psi$ and finally we have $ \pi_{,c_s^2} \sim  H ^{-1} \partial^2 \Psi $ which result in:
\be
\dot{\pi} -\Psi \sim c_s^2 \partial^2 \Psi /H^2
\ee
 \begin{figure}[H]
 \includegraphics[scale=0.5]{cancellation_stress_tensor.jpg} 
 \end{figure}

\subsection{How is it related to fluid language?}
First of all according to our observation we saw that $\dot{\pi} -\Psi \sim c_s^2 \partial^2 \Psi /H^2$, so actually the $\dot{\pi}$ plays the role of gravitational potential for us! But to confirm the relation we look at the fluid equations according to eq. 2 or eq. 9 of {\url{https://arxiv.org/abs/0909.0007v2}}. We start off continuity equation from Ma and Bertschinger paper {\url{https://arxiv.org/pdf/astro-ph/9506072.pdf}}
\be
\delta' = -(1+w) (\theta - 3 \Phi') - 3 \mathcal{H} \Big( \frac{\delta P}{\delta \rho} -w \Big ) \delta
\ee
which $'$ denotes the derivative with respect to conformal time! In the limit $w=0$ and also $c_s^2 \rightarrow 0$ we can rewrite the equation as following,
\be
\delta' = - (\theta - 3 \Phi')
\ee
Just using the the values of quantities for kessence case $\delta= \frac{1}{c_s^2} \Big( -\Psi+   { ({\pi'}+ \mathcal{H} \pi) } \Big ) $ and 
$\theta = - \partial^2 \pi $. It is easy to see that we get $c_s^2 (\theta - 3 \Phi')$ in righthand side which goes away for small sound speeds and left hand side is actually what we are looking for,
\be
\delta'=\pi''+\mathcal{H}' \pi + \mathcal{H} \pi' - \Psi'=0
\ee
\subsubsection{Euler equation:}
According to the equation 30 of  {\url{https://arxiv.org/pdf/astro-ph/9506072.pdf}}, the Euler equation is (for the limit we are interested in):
\be
\theta'= -\mathcal{H} \theta  - \nabla^2 \Psi
\ee
Using the relation for $\theta= - \partial^2 \pi  $ we get,
\be
 - \partial^2 \pi' - \mathcal{H}  \partial^2 \pi  + \nabla^2 \Psi = 0
\ee
which easily gives just the derivative of continuity equation 
\be
 - \partial^2 \Big( \pi' + \mathcal{H} \pi - \Psi  \Big )=0
\ee
\subsection{Solution of the equation in the matter dominated universe }
According to the FRW equation in matter dominated universe we know that,
\be
H^2=8 \pi G \rho_m/3 \longrightarrow \rho_m \sim 1/a^3  \longrightarrow H \sim a^{-3/2}  \longrightarrow \mathcal{H} = a {H}\sim a^{-1/2}
\ee
and 
\be
a(\tau) = (\tau/\tau_0)^2 \; \; \; \; a(t) = (t/t_0)^{2/3}
\ee
First of all for $\Psi$ we use the Poisson equation which is (Poisson equation for physical length derivative, while for conformal length derivative we do not have $a^2$ contribution which is absorbed in $\nabla^2$!) 
\be
- k^2 \Psi = 4 \pi G \rho \delta a^2
\ee
On the other hand according to Einsteins equations  we have,
\be
k^2 (\Phi -\Psi) = -32 \pi G a^2 \rho_r \Theta_{r,2} \; \;  \text{and}\; \;  k^2 \Phi + 3 H (\dot{\Phi} - \Psi H ) = 4 \pi G a^2 (\rho_m \delta_m + 4 \rho_r \Theta_{r,0})
\ee
which $\Theta_{r,2}$ is the quadrupole moment of radiation which in our case is zero, but in general  the l'th moment is defined as,
\be
\Theta_l=\frac{1}{(-i)^l} \int _{-1}^1 \frac{d \mu}{2} \mathcal{P}_l(\mu) \Theta(\mu)
\ee
At the end for matter dominating universe we have,
\be
k^2 (\Phi -\Psi) = 0 \; \;  \text{and}\; \;  k^2 \Phi + 3 H (\dot{\Phi} - \Psi H ) = 4 \pi G a^2 \rho_m \delta_m 
\ee
which end-up with,
\be
k^2 \Phi + 3 H (\dot{\Phi} - \Phi H ) = 4 \pi G a^2 \rho_m \delta_m 
\ee
which is equivalent to Poisson equation!  $\Theta(\mu)$ is temperature and $ \mathcal{P}_l$ Legendre polynomial of order l,
And form the Euler and Continuity equation we can write,
\be
\delta' = - (\theta - 3 \Phi') \;\;\;\; \theta'= -\mathcal{H} \theta  + k^2 \Psi
\ee
Combining the two equation we get,
\be
\delta'' + \theta '+ 3 \Phi''=0  \longrightarrow  \delta'' -\mathcal{H}  (-\delta' +3 \Phi' )  -4 \pi G \rho \delta a^2+ 3 \Phi''=0 \;   
\ee
Just using the observation that in Matter dominated universe $\Phi$ is constant and putting $a(\tau) = (\tau/\tau_0)^2$, $ \mathcal{H} = \frac{a'}{a} = 2/\tau $ and $4 \pi G  \rho_m a^2 =( 3/2) \mathcal{H}^2 =6/\tau ^2$
\be
 \delta'' + 2  \delta' /\tau -6 \delta/\tau^2=0 \;   
\ee
The solution to this second order equation is,
\begin{figure}[H]
 \includegraphics[scale=0.6]{Soveld2} 
 \end{figure}
 which one is growing $\delta \sim a$ and the other is decaying!
\subsection{Checking the consistency!}
To check the consistency we need to prove that the solution of field equation (\ref{fieldeq}) gives the same answer! we use $\delta_{kess}$ and show that the time solution for each mode is like matter! \\
The filed equation for $\Psi' \approx 0$, $\mathcal{H}= 2/\tau$  and $\mathcal{H}'= -2/\tau^2$ is written,
\be
\pi''+ 2\pi'  /\tau      -2  \pi /\tau^2 = 0 
\ee
The solution is,
\begin{figure}[H]
 \includegraphics[scale=0.6]{eq3} 
 \end{figure}
 We don't get the consistent result if we just naively assume that for the behaviour of $\delta$ with $\mathcal{H}\pi \sim const$ and other terms are also constants so $\delta$ is constant in time which is contradiction!\\
 The contradiction comes from the fact that we have $c_s^2$ in the denominator of expression for $\delta$ which becomes so large in our naive calculation. If we use the result which we have obtained according to \ref{eqsenatore}, we observe that 
\be
\delta_{kess} \sim (\pi'+\mathcal{H} \pi -\Psi)/c_s^2 \sim  (c_s^2 \partial^2 \Psi/ \mathcal{H}^2)/c_s^2 \sim -k^2 \Psi/ \mathcal{H}^2 \sim 1/\mathcal{H}^2 \sim \tau^2 \sim a
\ee
which we have assumed that $\Psi$ is constant in time. So we get consistent results!

\subsection{Class and Gevolution. results, $w=10^{-8}$, $c_s^2=10^{-16}$: results from the code} 
We cannot set $w=0$ in the class, it should be negative! so we put it $10^{-16}$ and $c_s=10^{-8}$. \\
The linear equation  is as following;
\begin{figure}[H]
 \includegraphics[scale=0.5]{class} 
 \end{figure}
 For the Gevolution, when we set the small parameters, the particles speedup and go outside of CPUs. So it seems that the relation we need to be satisfied $ (\pi'+\mathcal{H} \pi -\Psi)/c_s^2 \sim  (c_s^2 \partial^2 \Psi/ \mathcal{H}^2)/c_s^2 $ is not satisfied so we get a large contribution from $1/c_s^2$. \\
 Now lets turn off the $T_{\mu \nu}$ of kessence and just using the obtained value of $\pi$ and $\pi'$ and $\Psi$ from class to make $\delta_{kess}$ ourselves! Or we can increase the value of sound speed to $c_s^2=10^{-6}$ and keep the small value of $w=10^{-8}$, since we only want to need if $\delta_{matt}$ and $\delta_{kess}$ match?!
  \begin{figure}[H]
 \includegraphics[scale=0.5]{class_10_e6.jpg} 
 \end{figure}
 \begin{figure}[H]
 \includegraphics[scale=0.5]{class_10_e8.jpg} 
 \end{figure}
\newpage
\section{$\Psi'$ comparison in Gevolution and other codes}
\subsection{Equation for $\Phi'$} In Class 
we want to add a new equation for $\Phi'$ since Gevolution does not give the right solution in high k, which is maybe because of noises ?!..\\
We add the below equation which is $0i$ Einstein equation in conformal Newtonian gauge, eq 23.b of \url{https://arxiv.org/pdf/astro-ph/9506072.pdf}
\be
\Phi'=-\mathcal{H} \Psi+ \sum _i\frac{4 \pi G a^2 (\bar{\rho_i} + \bar{P_i}) \theta_i}{k^2}
\ee
How to implement it in Gevolution?!\\
We have,
\be
 (\bar{\rho} + \bar{P}) \theta =  \nabla_i T_0^i = i k^j T_j^0
\ee
But again we need the derivative of $T_0^i$ which is the same issue as $\Phi'$ instead we use $(00)$ Einstein equation, eq. 23a of \url{https://arxiv.org/pdf/astro-ph/9506072.pdf}
\be
\Phi'=-\mathcal{H} \Psi - \sum _i\frac{3 \delta_i} {2}
\ee
Actually in matter dominated universe the first order term vanishes, so class and Gevolution mismatch even in high redshift! so we need to check $\dot{\Psi}$ in Gevolution compared with second order perturbation theory.
-
\subsection{Comparing the $\Psi'$ in Gevolution with Riess Sciama effect at redshift 50.}
In matter dominated universe ${\Phi'}$ vanishes in linear order. Next order contribution would be,
\be
{\Phi'} = -\frac{3 H_0^2 }{2 k^2} {a'} \delta_2
\ee
where $\delta= a \delta_1 +a^2 \delta_2 $.
\be
\delta_2 (\vec{k}) = \int d^3 {q_1} \int d^3{q_2 } \;  \delta_D(\vec{k}-\vec{q_1}-\vec{q_2}) \;  F_2(\vec{q}_1 , \vec{q}_2)  \; \delta_1 (|\vec{q}_1|) \, \delta_1(|\vec{q}_2|)
\ee
\be
F_2(\vec{q}_1,\vec{q}_2)= \frac{5}{7} + \frac{1}{2} \frac{\vec{q}_1 . \vec{q}_2}{q_1 q_2} \Big ( \frac{q_1}{q_2} + \frac{q_2}{q_1} \Big) + \frac{2}{7} \frac{(\vec{q}_1 .\vec{q}_2)}{q_1^2 q_2^2}
\ee
So we obtain,
\be
P_{{\Phi'} }=  \frac{9}{4} (\frac{H_0}{k} )^4 {a'}^2 \; P_{22}
\ee
\be
P_{22} (k) = \int d^3 {q} P_{\delta} ({q}) P_{\delta} (|\vec{k}-\vec{q}|) \; F_2^2(\vec{q} , \vec{k} - \vec{q})
\ee
We  use the Growth factor,
\be
D^{+}= H(a) \frac{5 \Omega_m}{2} \int \frac{d \,a}{a^3 H(a)}
\ee
and physical Hubble ,
\be
H(a)=\sqrt{\Omega_m a^{-3} + (1-\Omega_m- \Omega_{\Lambda}-\Omega_{kess}) a^{-2}+ \Omega_{kess}^{-3(1+w)}+\Omega_{\Lambda}}
\ee
To make dimensionless quantity we have:
\be
P_{{\frac{\Phi'}{\mathcal{H}}} }=  \frac{9}{4} (\frac{H_0}{k} )^4 {a}^2 \; P_{22}
\ee
where $[  P_{{\frac{\Phi'}{\mathcal{H}}} }]=[P_{22}]= L^3$ and to make dimensionless powerspectrum we have $\mathcal{P}_{{\frac{\Phi'}{\mathcal{H}}} } =  k^3 P_{{\frac{\Phi'}{\mathcal{H}}}}/2 \pi^2 $. \\
What we are going to compare:
\be
\mathcal{P}_{\frac{\Phi'}{\mathcal{H}(a)}} (class)= \mathcal{R} ^2 \Big( \frac{\Phi'_{class}}{\mathcal{H}(a) \mathcal{R}} \Big)^2= A_s  \Big( \frac{k}{k_p} \Big)^{n_s-1} \; (\frac{\Phi'}{\mathcal{H} \mathcal{R}})^2
\ee
\be
\mathcal{P}_{\Phi'/\mathcal{H}} (Gevolution) = \text{output}
\ee
\be
P_{{\frac{\Phi'}{\mathcal{H}}} } (\text{Riess-Sciama})=  \frac{9}{4} (\frac{H_0}{k h} )^4 {a}^2 \; P_{22}
\ee
Since the unit of $P_{22}$ is $L^3/h^3$ to get dimensionless powerspectrum we have,
\be
\mathcal{P}_{{\frac{\Phi'}{\mathcal{H}}} } (\text{Riess-Sciama}) = k^3 P_{{\frac{\Phi'}{\mathcal{H}}} } (\text{Riess-Sciama}) /2 \pi^2
\ee
and $k=h/Mpc$ \\
The pyhton code for the way we calculate it:
\begin{lstlisting}[language=Python]
# Parameters for converting to dimensionless power.
As=2.19*10**-9;
h=0.67556
kp=0.05/h; 
ns=0.96;
cs2=1.e-6;
def Hubble_conf_Mpc(a):
    H0=0.00022593979933110373;w=-1;h=0.67556;
    Omega_b=0.022032/h/h; Omega_cdm=0.12038/h/h;
    Omega_m=Omega_b+Omega_cdm; Omega_Lambda=0.0;
    Omega_rad=9.16681e-05; Omega_kessence=1.-Omega_m-Omega_rad;
    return H0*np.sqrt(Omega_m*(a**-3)+Omega_rad*(a**-4)+Omega_Lambda+Omega_kessence*(a**(-3*(1+w))))*a
a50=1./(1.+50.);
pow_phi_prime_Riess_Sciama= a50**2 * ((P_22_power[:,0])**3/(2*np.pi**2))  * P_22_power[:,2]*(9./4.)*(Hubble_conf_Mpc(1.)/P_22_power[:,0]*h)**4

# Class power!
classpower_phi_prime_z50=As*((class_phi_z50[:,2])**2)*((class_phi_z50[:,0]/(kp*h))**(ns-1.));

classpower_phi_z50=As*((class_phi_z50[:,3])**2)*((class_phi_z50[:,0]/(kp*h))**(ns-1.));
plt.plot(Gev_power_phi_prime[:,0]*h,Gev_power_phi_prime[:,1] ,color="blue",linestyle='dashed',lw=1.5,label=r"$\Phi'$/H  Gev z=50")
##############################
plt.plot(class_phi_z50[:,0],classpower_phi_prime_z50[:] ,color="red",linestyle='dashed',lw=1.5,label=r"$\Phi'$/H class, z=50 ")

\end{lstlisting}
\begin{figure}[H]
 \includegraphics[scale=0.5]{compI.jpg} 
 \end{figure}
\subsection{Comparing $P_{22}$ from different codes!}
We are going to compare $P_{22}$ from Joyce code and Eichiro Kumatsu's. It seems they more or less agree! \\

\begin{figure}[H]
 \includegraphics[scale=0.5]{one_loop_comp} 
 \end{figure}
\subsection{Comparing $\dot{\Phi}$ in Gevolution with Gadjet code!}
Gadget does not give nor $\Phi$ neither $\delta_m$ power as the output!
\subsection{Test of Gevolution when we add $\Psi' d\tau$ to $\Psi_{ini}$ to get the final value in class and Gevolution  }
What we are going to do is comparing $\Phi(z=50)$ with $\Phi(z=90) + \Phi'(z=90) \times d \tau = \Phi(z=90) + \Phi'(z=90) \times \frac{1}{da /d\tau} \frac{da}{dz} dz = \Phi(z=90) - \Phi'(z=90)/\mathcal{H} \times a \Delta z $ and $\Phi'/\mathcal{H}$ is defined in the Gevolution!
\begin{figure}[H]
 \includegraphics[scale=0.5]{compII.jpg} 
 \end{figure}
 \begin{figure}[H]
 \includegraphics[scale=0.4]{comp_class_gev.jpg} 
 \end{figure}
 -arXiv:0809.4488v3 [astro-ph] Try to compare Gadjet and Gevolutiion and the plot in the paper! \\
 -Check thet if we give $\Phi'$ from Gev we get the same solution in Gevolution and Mathematica!
 \\
  add the plots in mathematica which we get the same result in class and GEv and from solving myself when we turn off $\Phi'$ . \\
- Use the small quantity $\pi'+ \mathcal{H} \pi -\Psi$ as a  new variable with $\pi$ or $\pi'$ which we know that works well, \\
Maybe we need to switch bot $\pi$ and $\pi'$ to $\delta$ and $\theta$. Now we have accurate Stress tensor, and we can compare,
\subsection{Trying to get $\Phi'$ from second order codes}
Up to now I have tried to understand what is going on the Ramses and Gadget2 code, but although the documentations are good but is not so useful for cosmological applications like how to get powerspectrum of potential or even matter?! Even after searching the word power, nothing is mentioned in the documentation how to get power as an output!! \\ 
Now I'm going to try other cosmological codes like L-PCola to see if I can obtain what I want? L-PCola was the same. Now I'm gonna try FastPM which seems better! Or I can try also Pcola not L-Pcola! Lets see!
\subsubsection{FastPM}
FastPM is a N-body code. From the docs: FastPM solves the gravity Possion equation with a boosted particle mesh. Arbitrary
time steps can be used.  
The code is indented to study the formation of large scale structure. \\In addition to the snapshots, FastPM calculates and writes
the power-spectrum at each time step. \\
I get some library errors when I make the makefile which after some tries I could not solve! Maybe it is better to try another code and if I could not use them come back to this code!

\subsubsection{L-PICOLA}
I could run it but it does not give the potential powerspectrum directly! So it is not useful for us! We can use it for fast N-body simulation...
\subsubsection{Going to second order Boltzmann codes}
As we realized that the radiation is not implemented in the N-body codes like RAMSES or Gadget-2 and also ISW effect is really relavant at high redshift ($z \sim 50$) as we see the effect in class, because $\Phi'$ is not zero and it is because of radiation! Moreover these N-body codes do not give matter power or $\Phi$ power to compare with Gevolution. \\
It turn out that the best way is to catch all the ISW and Riess Sciama effect is that we obtain matter power in these codes in two near redshifts and compute $\Phi$ power from Poisson equation and then by subtracting them get $\Phi'$ and then compare with $\Phi'$ in class and the same way in  Gevolution. to see what happens! \\
Moreover we need to change equation to two other variables which catch the smallness of $\pi'-\Psi$ well from the begining.
\subsection{Can we trust Poisson equation to get $\Phi$ power from $\delta$ power in other codes like RAMSES or Gadge? {\color{red} Todo? check the parameters!}!}
\subsection{Matter power and $\Phi$ power in Gevolution and class via Poisson equation: ({\color{red} Class results must be checked!})}
Now we want to first show that if we use Poisson equation we can relate $\delta$ power to $\Phi$ power in Gevolution and class,
\be
- k^2 \Psi = 4 \pi G \rho \delta a^2
\ee
which results,
\be
\mathcal{P}_{\Psi}=\frac{ 16 \pi^2 G^2  \rho_m^2 \mathcal{P}_{\delta}}{k^4 (1+z)^2}
\ee
Moreover we have $\rho_m=\Omega_m \rho_{cr}/a^3$ and $4 \pi G \rho_{cr} =(3/2) {H_0}^2 $, so we have  $4 \pi G  \rho_m a^2 = 4 \pi G  \frac{\rho_m (a)}{\rho_{cr} (0)} \rho_{cr} (0) a^2 = 4 \pi G  \frac{\Omega_m (0) a^{-3} \rho_{cr}}{\rho_{cr} (0)} \rho_{cr} (0) a^2= (3/2) {H_0}^2   {\Omega_m (0) a^{-3} }  a^2 = ( 3/2) \mathcal{H}_0^2 \Omega_m (0) /a $ which gives,
\be
\mathcal{P}_{\Psi}=\frac{ 9  \mathcal{H}_0 ^4  \Omega_{m,0}^2  \mathcal{P}_{\delta}}{ 4 a^2 k^4}
\ee
So we compare $\mathcal{P}_{\Psi}$ from Poisson equation obtained from $\mathcal{P}_{\delta}$ with $\mathcal{P}_{\Psi}$ directly from Gevolution. Note that $\mathcal{H}$ and k must be in the same unit.
  
Again we see the tension in the class as well.  but why?! {\color{red}{Strange?!}}
So we compare $\mathcal{P}_{\Psi}$ from Poisson equation obtained from $\mathcal{P}_{\delta}$ with $\mathcal{P}_{\delta}$ directly from Gevolution. Note that $\mathcal{H}$ and k must be in the same unit.
\\
Comparing the class $\Phi$ directly from transfer function and what we get from Poisson equation  according to the below Python script,
\begin{lstlisting}[language=Python]
def Hubble_conf_Mpc(a):
    H0=0.00022593979933110373;w=-1.;h=0.7;
    Omega_b=0.022032/h/h; Omega_cdm=0.12038/h/h;
    Omega_m=0.30; Omega_Lambda=0.0;
    Omega_rad=9.16681e-05; Omega_kessence=1.-Omega_m-Omega_rad;
    return H0*np.sqrt(Omega_m*(a**-3)+Omega_rad*(a**-4)+Omega_Lambda+Omega_kessence*(a**(-3.*(1.+w))))*a
# Class power!
h=0.7;
Omega_b=0.022032/h/h; Omega_cdm=0.12038/h/h;
Omega_m=0.3;
a100=1./(1.+100.)
a10=1./(1.+10.)
a1=1./(1.+1.)
a0=1./(1.+0.)
#phi from Poisson equation: 
As=2.3e-9;
w=-1.
# h=0.67556
kp=0.05/h; 
ns=1.;
# cs2_e3=1.e-6;
#################################
#Plotting
plt.figure(figsize=(20,12))
ax = plt.gca()
ax.tick_params(axis = 'both', which = 'major', labelsize = 24)
ax.tick_params(axis = 'both', which = 'minor', labelsize = 16)
#Making power of class field to compare with Gev. dimensionless power of phi from matter power!
phi_class_poisson_z100= (9./4.)* (Omega_m/a100)**2 * (Hubble_conf_Mpc(1./(1.+0.0))**4)* ((class_pow_z100[:,0])**3/(2.*np.pi**2)) * class_pow_z100[:,1]/((class_pow_z100[:,0]*h)**4)
phi_class_poisson_z10= (9./4.)* (Omega_m*(a10**-1))**2 * (Hubble_conf_Mpc(1./(1.+0.0))**4)* ((class_pow_z10[:,0])**3/(2.*np.pi**2)) * class_pow_z10[:,1]/((class_pow_z10[:,0]*h)**4)
phi_class_poisson_z1= (9./4.)* (Omega_m*(a1**-1))**2 * (Hubble_conf_Mpc(1./(1.+0.0))**4)* ((class_pow_z1[:,0])**3/(2.*np.pi**2)) * class_pow_z1[:,1]/((class_pow_z1[:,0]*h)**4)
phi_class_poisson_z0= (9./4.)* (Omega_m*(a0**-1))**2 * (Hubble_conf_Mpc(1./(1.+0.0))**4)* ((class_pow_z0[:,0])**3/(2.*3.1415**2)) * class_pow_z0[:,1]/((class_pow_z0[:,0]*h)**4)

power_class_z100_direct= As*(class_all_z100[:,6])**2*((class_all_z100[:,0]/kp)**(ns-1.));
power_class_z10_direct= As*(class_all_z10[:,6])**2*((class_all_z10[:,0]/kp)**(ns-1.));
power_class_z1_direct= As*(class_all_z1[:,6])**2*((class_all_z1[:,0]/kp)**(ns-1.));
power_class_z0_direct= As*(class_all_z0[:,6])**2*((class_all_z0[:,0]/kp)**(ns-1.));

power_class_z100_poisson= As*phi_class_poisson_z100*((class_pow_z100[:,0]/kp)**(ns-1.));
power_class_z10_poisson= As*phi_class_poisson_z10*((class_all_z10[:,0]/kp)**(ns-1.));
power_class_z1_poisson= As*phi_class_poisson_z1*((class_all_z1[:,0]/kp)**(ns-1.));
power_class_z0_poisson= As*phi_class_poisson_z0*((class_all_z0[:,0]/kp)**(ns-1.));
plt.loglog(class_all_z100[:,0],np.abs(phi_class_poisson_z100[:]-power_class_z100_direct[:])/(power_class_z100_direct[:]) ,color="red",linestyle='dashed',lw=1.5,label=r"$\Delta \mathcal{P}_\Phi/\mathcal{P}_\Phi$ class, z=100")
plt.loglog(class_all_z100[:,0],np.abs(phi_class_poisson_z10[:]-power_class_z10_direct[:])/(power_class_z10_direct[:]) ,color="blue",linestyle='dashed',lw=1.5,label=r"$\Delta \mathcal{P}_\Phi/\mathcal{P}_\Phi$ class, z=10")
plt.loglog(class_all_z100[:,0],np.abs(phi_class_poisson_z1[:]-power_class_z1_direct[:])/(power_class_z1_direct[:]) ,color="black",linestyle='dashed',lw=1.5,label=r"$\Delta \mathcal{P}_\Phi/\mathcal{P}_\Phi$ class, z=1")
plt.loglog(class_all_z0[:,0],np.abs(phi_class_poisson_z0[:]-power_class_z0_direct[:]*1.2)/(power_class_z0_direct[:]*1.2) ,color="purple",linestyle='dashed',lw=1.5,label=r"$\Delta \mathcal{P}_\Phi/\mathcal{P}_\Phi$ class, z=0"
\end{lstlisting}

We get the below results which shows that something goes wrong.
 \begin{figure}[H]
 \includegraphics[scale=0.4]{Class_Poisson.jpg} 
 \end{figure}
  \begin{figure}[H]
 \includegraphics[scale=0.2]{Class_Gev_Poisson.jpg} 
 \end{figure}
As it is clear in the figure in low redshifts in class and z=0 in Gevolution we have about $10\%$ error (?) and in $z=0$ {\color{red}{Strange?!}}
In the Gevolution also according to below python script, we get the related plot,
\begin{lstlisting}[language=Python]
def Hubble_conf_Mpc(a):
#     H0=0.00022593979933110373
    w=-1;h=0.67556;c=2.9992458*10**5;
    H0=100.*h/c
    Omega_b=0.022032/h/h; Omega_cdm=0.12038/h/h;
    Omega_m=Omega_b+Omega_cdm; Omega_Lambda=0.0;
    Omega_rad=9.16681e-05; Omega_kessence=1.-Omega_m-Omega_rad;
    return H0*np.sqrt(Omega_m*(a**-3)+Omega_rad*(a**-4)+Omega_Lambda+Omega_kessence*(a**(-3.*(1.+w))))*a
# Class power!
# Formula= P_phi= 9 H0^4 Omega_0^2 P_delta /4a^2 k^4
a50=1./(1.+49.333442);
a0=1./(1-0.000995);
Omega_b=0.022032/h/h; Omega_cdm=0.12038/h/h;
Omega_m=Omega_b+Omega_cdm;
#phi from Poisson equation: 
phi_poisson_z50= (9./4.) * (Omega_m*(a50**-1))**2 * (Hubble_conf_Mpc(1./(1.+0.0))**4)*deltam_gev_pow_z50[:,1]/((deltam_gev_pow_z50[:,0]*h)**4)
phi_poisson_z0= (9./4.) * (Omega_m*(a0**-1))**2 * (Hubble_conf_Mpc(1./(1.+0.0))**4)*deltam_gev_pow_z0[:,1]/((deltam_gev_pow_z0[:,0]*h)**4)
plt.loglog(phi_gev_pow_z50[:,0],np.abs(phi_poisson_z50[:]-phi_gev_pow_z50[:,1])/(phi_gev_pow_z50[:,1]) ,color="blue",linestyle='dashed',lw=1.5,label=r"$\Delta \mathcal{P}_\Phi/\mathcal{P}_\Phi (gev)  $ Gev, z=50")
plt.loglog(phi_gev_pow_z0[:,0],np.abs(phi_poisson_z0[:]-phi_gev_pow_z0[:,1])/(phi_gev_pow_z0[:,1]) ,color="red",linestyle='dashed',lw=1.5,label=r"$\Delta \mathcal{P}_\Phi/\mathcal{P}_\Phi (gev)  $ Gev, z=0")
#################################
plt.legend(bbox_to_anchor=(0.0, 0.15, 0.3, .102), loc=1,ncol=1,fontsize=13, mode="expand", borderaxespad=0.)
# plt.title('tiling factor=512, N_grid=2048, boxsize =1400.0, time step limit=0.04')
plt.xlabel("k[1/Mpc]",fontsize=14)
plt.ylabel(r"$|\Delta \Phi|/\Phi$",fontsize=23)
plt.grid(True)
plt.savefig('Gev_Poisson.jpg', format='jpg', dpi=500)
plt.show()
\end{lstlisting}
\begin{figure}[H]
 \includegraphics[scale=0.4]{Gev_Poisson_newrun.jpg} 
 \end{figure}
 \subsection{Results for the relatively big run and comparison between class and Gevolution vs plot in the arXiv:0809.4488v3 paper! }
\begin{figure}[H]
 \includegraphics[scale=0.9]{Comp_paper.jpg} 
 \end{figure}
 
 
 %%%%%%%%%%%%%%%%
%%%%%%%%%%%%%%%%
%%%%%%%%%%%%%%%%
\subsection{Comparing obtained $\Phi'$ from full second order matter and Poisson equation at two different redshifts with the one we have obtained from Riess Sciama effect {\color{red} To do!}}
{\color{red}We still have problems in Poisson equation!}
%%%%%%%%%%%%%%%%
%%%%%%%%%%%%%%%%
%%%%%%%%%%%%%%%%
\subsection{Use Kumatsu code in two redshift and compute Poisson equation and get ? power at two different redshift! and cross check by Seljak formula.}
%%%%%%%%%%%%%%%%
%%%%%%%%%%%%%%%%
%%%%%%%%%%%%%%%%
\subsection{Comparing $\Phi'$ in class and Gevolution in much larger scales!}
Since the plot which shows tension between class and Gevolution is sketched for high wavenumbers which does not show if class and Gevolution agree on larger scales (more linear part), now we want to compare then in much larger scales (large boxsize)!
As you can see from the three below plots, there is a big tension between class and Gevolution!! Although we cannot see the effect on the potential power, it is very clear in $\Phi'/H$ power!
We have also checked that $\Phi'$ power is stable under change of time stepping, moreover in some previous chapters we also checked that it is really what it is by adding $\Phi'$ to $\Phi_{ini}$ and getting $\Phi_{f}$! {\color{red} Is it early ISW effect?!}
So here we need to prove that Gevolution result is correct!?


\begin{figure}[H]
 \includegraphics[scale=0.35]{./phi_prime_results/comp_new.jpg} 
 \end{figure}

 \begin{figure}[H]
 \includegraphics[scale=0.2]{./phi_prime_results/comp1.jpg} 
 \end{figure}
 \begin{figure}[H]
 \includegraphics[scale=0.2]{./phi_prime_results/comp2.jpg} 
 \end{figure}
 \section{Solving the linear field equation in mathematica and comparing with Gevolution and checking that if we use $\Phi'$ of Gevolution we get the same result as Gevolution, just consistency check!}
 Firstly we show that if we neglect $\Phi'$ in mathematica, which is a good approximation for matter dominated universe, we get the same result as class (how much error?) and Gevolution if we just add linear field equation and turn off $\Phi'$ by hand. According to the field equation and Stress energy tensor as following,
\begin{align} 
 &{ \pi''+\mathcal{H}(1- 3w) \pi' } +3 {  \mathcal{H}}\Big( -c_s^2+ {w} \Big )\Psi - \, {\Psi'}- 3 c_s^2  \,{\Phi'} + {
 \Big( 3\mathcal{H}^2 (c_s^2 -w) + \mathcal{H}' (1-3c_s^2)\Big) \pi }
           \nonumber
   \\
    &
 - c_s^2 {\nabla^2 \pi} =0
    % Second order terms==0
  \end{align} 
\begin{align}
 & T_0^0 (Gev)=  \Omega^0_{kess} a^{-3 w}  \Bigg[1+ \frac{1+w}{c_s^2} \Big(- 3 \mathcal{H}c_s^2 \pi- \Psi+   {({\pi'}+ \mathcal{H} \pi) }    \Big )   \Bigg ]
\nonumber \\ &
T^{i}_{0}(Gev)= - \Omega^0_{kess} a^{-3 w} (1+w) \partial _i \pi 
\nonumber \\ &
T_{j}^{i}(Gev)= w  \, \Omega^0_{kess} a^{-3 w} \Bigg ( 1+  \frac{1+w}{w}\Big [ -3 \mathcal{H} w \pi- \Psi +   {({\pi'}+ \mathcal{H} \pi) }\Big] \delta_{j}^{i}   \Bigg) 
\end{align}
In mathematica (Equation$\_$TestSolve.nb), the background part is defined:
 \begin{figure}[H]
 \includegraphics[scale=0.7]{math1} 
 \end{figure}
 The full equation where $\Psi'$ neglected is written and then we have changed the variable to $a(\tau)$ and all the functions of $\tau$ as well.
  \begin{figure}[H]
 \includegraphics[scale=0.7]{math2} 
 \end{figure}
 Importing the file made in the class for the initial condition of the field and its derivative and also $\Psi$ value as a function of k which is assumed remains constant in time (in CDM universe),
   \begin{figure}[H]
 \includegraphics[scale=0.7]{math3} 
 \end{figure}
Now for each k we need to solve the differential equation and obtain the solution in time, in the below mathematica code, we have use constan $\Psi$ from class and it is important to note that since in the new ODE which is in terms of $a$ and not $\tau$ we need to rescale the initial condition from the class (which is appropriate for $\pi'$) to get the IC for $d\pi/da$. \\At the to be able to compare with Class and Gevolution results we need to rescale the obtained solution and also make dimensionless quantity to compare simply (which is done in the below code),
   \begin{figure}[H]
 \includegraphics[scale=0.7]{math4} 
 \end{figure}
 Now the way that we can have access to the information  of the solved equation is shown in the mathematica code below, moreover the way which class  and Gevolution output should be scaled to be comparable with mathematica solution is shown,\begin{figure} [h]
 \includegraphics [scale=0.6]{math5}
 \end{figure}
 The comparison result of Gevolution, class and our solution in mathematica for the case which $\Phi'$ is turned off is as following. \\
 Neglecting the $\Phi'$ and $\Psi'$ in Gevolution and out direct calculation we get the below result! After starting to solve at redshift $z=100$ from initial condition in class and getting result at $z=10$, the relative error of $\pi'$ between class and our solution is not very good because we assumed $\Phi'$ and $\Psi'$ exactly zero. The relative error in terms of $k$ in $1/Mpc$ is as following. As it is shown up to $k=0.5 [1/Mpc]$ we have less than 7 $\%$ error in $\pi'$.
 \begin{figure} [H]
 \includegraphics [scale=0.6]{math_relerror_piv}
 \end{figure}
While the relative error on $\pi$ is negligible as it is clear in the below plot,
 \begin{figure} [H]
 \includegraphics [scale=0.6]{math_relerror_pi}
 \end{figure}
 At the end if we compare the $\pi$ result from Mathematica and class we get,
  \begin{figure} [H]
 \includegraphics [scale=0.6]{math_relerror_pi}
 \end{figure}
 It is important to note that neglecting the potential decay $\Phi'$ and $\Psi'$ causes about 7$\%$ error on $\pi'$ while it does not affect $\pi$ solution. \\
 For a reasonable run we get a realatively good match between Gevolution (when $\Phi'$ and $\Psi'$ term is not included in Gevolution) and class and our solution. It is clear from the plot that in Class the potential decay causes that $\pi'$ decays in Class relative to out mathematica solution. It is also important to note that the result in Gevolution is sensitive to number of grid and number of kessence field update, it is sensitive to number of grid because otherwise we get wrong potential from the particles which causes large error in the results and number of kessence field update is important to insure that we get the right solution from Leap frog method.
  \begin{figure} [H]
 \includegraphics [scale=0.4]{pi_gev_class_1}
 \end{figure}
   \begin{figure} [H]
 \includegraphics [scale=0.4]{pi_v_gev_class_1}
 \end{figure}
 the difference between Gevolution and our solution at high wavenumbers absolutely comes from potential $\Psi$ term!
\subsection{How much the result in Gevolution is sensitive to precision parameters?}
To check how much other parameters are important in solution to scalar field, we plot the same figure but with different precisions! \\
To see the effect of the number of kessence field update, we dearcease the number of updates to . It is clear that we get less match in comparison with 10 number of updates.
 \begin{figure} [H]
 \includegraphics [scale=0.4]{pi_1_1}
 \end{figure}
  \begin{figure} [H]
 \includegraphics [scale=0.4]{pi_v_1_1}
 \end{figure}
To see the effect of Number of grids we decrease the number of grids to 128. It is clear that we do not get the same behaviour as we got before.
 \begin{figure} [H]
 \includegraphics [scale=0.4]{pi_gev_class_1_128grid}
 \end{figure}
 \begin{figure} [H]
 \includegraphics [scale=0.4]{pi_v_gev_class_1_128grid}
 \end{figure}
 For the 64 number of grids,
  \begin{figure} [H]
 \includegraphics [scale=0.4]{pi_gev_class_1_64grid}
 \end{figure}
 \begin{figure} [H]
 \includegraphics [scale=0.4]{pi_v_gev_class_1_64grid}
 \end{figure}
 To measure the effect of number of kessence field update and number of grid for part of the differential equations we also have,
   \begin{figure} [H]
 \includegraphics [scale=0.4]{wave_pi_1}
 \end{figure}
   \begin{figure} [H]
 \includegraphics [scale=0.4]{wave_pi_v_1}
 \end{figure}
   \begin{figure} [H]
 \includegraphics [scale=0.4]{wave_pi_2}
 \end{figure}
   \begin{figure} [H]
 \includegraphics [scale=0.4]{wave_pi_v_2}
 \end{figure}
 And for the highest possible number of grid on the local computer,
    \begin{figure} [H]
 \includegraphics [scale=0.4]{wave_pi_3}
 \end{figure}
    \begin{figure} [H]
 \includegraphics [scale=0.4]{wave_pi_v_3}
 \end{figure}
 \subsection{What happens if we turn on $\Phi'$ term in Gevolution? }
 As we have shown that the behaviour of $\Phi'$ and $\Psi'$ is different in class and Gevolution and since these potentials directly source $\pi'$ and $\pi$ we get different result when we turn on these terms in the Gevolutoin specially in high wavenumbers.
  \begin{figure} [H]
 \includegraphics [scale=0.4]{pi_phi_prime_comp}
 \end{figure}
   \begin{figure} [H]
 \includegraphics [scale=0.4]{pi_v_phi_prime_comp}
 \end{figure}
 The difference between the linear theory in Gevolution and our solution is completely because of $\Phi'$ and $\Psi'$ terms and we have checked for different precisions $\Phi'$ is different than class solution. We have checked the results for lower number of steps, higher number of grids and higher number of kessence field updates and the result is the same! We also have checked in previous sections that $\Phi'$ is what it should be in Gevolution...
 \subsubsection{Consistency check: If we get $\Phi$ and $\Phi'$ from Gevolution as an initial condition and solve it in mathematica}
 There is technical point here about how we use the initial condition from Gevolution to solve the ODE. Since the output of Gevolution is $\mathcal{P}_{\mathcal{H} \pi}$ we need to first get $\mathcal{H} \pi$ from the below and then dividing by $\mathcal{H}$ to put as an initial condition and then multiplying to each $\mathcal{H}$ to compare with Gevolution results in other redshifts! Moreover $\Phi'$ in Gevolution output is multiplied t $1/\mathcal{H}$ to make it dimensionless, so we need to multiply it to $\mathcal {H}$ to input as an initial condition.
    \begin{figure} [H]
 \includegraphics [scale=0.6]{mathematica_23april_1}
 \end{figure}
 Providing the initial condition at z=100 and putting $\Phi$ and $\Phi'$ values from $z=100$ or $z=10$ we get bad results, which shows that the redshift difference between initial condition and final state is so much. 
 \\ So setting the initial condition from Gevolution as following. From the figure we see that the initial condition of ODE and Gevolution result exactly overlap!
    \begin{figure} [H]
 \includegraphics [scale=0.4]{IC_newplot_1}
 \end{figure}
The result  for setting $\Phi$ and $\Phi'$ constant from z=10, which is not satisfying is as following,
     \begin{figure} [H]
 \includegraphics [scale=0.4]{pi_solve_23april}
 \end{figure}
     \begin{figure} [H]
 \includegraphics [scale=0.4]{pi_v_solve_23april}
 \end{figure}
 It is clear why our solution here is higher than Gevolution output, since we put the value of $\Phi$ and $\Phi'$ from redshift z=10 as the initial condition, so we are overstimating the value! \\
 To get a good result we use IC and final redshift near to each other...  \\
 Even providing the IC at z=20 and solving for z=10 is not good enough! (why?!) maybe $\Phi'$ changes so much at these redshift? 
      \begin{figure} [H]
 \includegraphics [scale=0.4]{z20z10_pi_1}
 \end{figure}
      \begin{figure} [H]
 \includegraphics [scale=0.4]{z20z10_pi_v_1}
 \end{figure}
 Providing the initial condition at z=21 and getting output at z=19 and comparing with Gevolution output we get the following plots. We see that we get better match than before but still we don't expect mismatch for these two  near redshifts?!
     \begin{figure} [H]
 \includegraphics [scale=0.4]{pi_com_z1921}
 \end{figure}
      \begin{figure} [H]
 \includegraphics [scale=0.4]{pi_v_com_z1921}
 \end{figure}
 If we provide the IC at z=50 and getting result at z=48 we get the below results. Is it consistent?!
      \begin{figure} [H]
 \includegraphics [scale=0.4]{pi_com_z5048}
 \end{figure}
      \begin{figure} [H]
 \includegraphics [scale=0.4]{pi_v_com_z5048}
 \end{figure}
 The results show that we have non negligible effect of $\Phi''$  which is very interesting on its own.


\section{Rewriting the field equations in terms of two new variables}
 Assuming that Gevolution calculates $\Phi$ and $\Phi'$ correctly, we rewrite the equations in terms of two new variables and implement in Gevolution to see what happens? Moreover we need to provide the appropriate initial conditions for the new variables! To rewriting the field equation we get help of Mathematica! \\
 If we use $\delta$, $\theta$ instead of $\pi$ and $\pi'$ for linear terms we will have problem which means we need more from Boltzmann hierarchy $\Pi$ anisotropic pressure and for non-linear scalar field equation we need even more... \\
 But just for linear theory! and just for the stress tensor part which causes problem in Gevolution for $c_s^2 \to 0$, so we take  $\delta$ when $c_s^2 \to 0$ as a new variable which is the term that the cancelation happen and we set the cancelation as an initial condition from class and let the other variable be $\pi$ or $\pi'$ according to scalar field equation.  \\
 \subsection{Linear calculation}
 The linear part of ODE and stress tensor is written,
 Just to observe, its interesting to notice the relation between $c_s^2$ with $\delta P/\delta \rho$ for scalar field, {\color{red} derive it? Is there any inconsistency? why dont get simply $c_s^2$?}
  \begin{align} 
 &{ \pi''+\mathcal{H}(1- 3w) \pi' } +3 {  \mathcal{H}}\Big( -c_s^2+ {w} \Big )\Psi - \, {\Psi'}- 3 c_s^2  \,{\Phi'} + {
 \Big( 3\mathcal{H}^2 (c_s^2 -w) + \mathcal{H}' (1-3c_s^2)\Big) \pi }
           \nonumber
   \\
    &
 - c_s^2 {\nabla^2 \pi} =0
    % Second order terms==0
  \end{align} 
  

\begin{align}
 & T_0^0 (Gev)=  \Omega^0_{kess} a^{-3 w}  \Bigg[1+ \frac{1+w}{c_s^2} \Big(- 3 \mathcal{H}c_s^2 \pi- \Psi+   {({\pi'}+ \mathcal{H} \pi) }    \Big )   \Bigg ]
\nonumber \\ &
T^{i}_{0}(Gev)= - \Omega^0_{kess} a^{-3 w} (1+w) \partial _i \pi 
\nonumber \\ &
T_{j}^{i}(Gev)= w  \, \Omega^0_{kess} a^{-3 w} \Bigg ( 1+  \frac{1+w}{w}\Big [ -3 \mathcal{H} w \pi- \Psi +   {({\pi'}+ \mathcal{H} \pi) }\Big] \delta_{j}^{i}   \Bigg) 
\end{align}
we take $\pi$ and $\zeta$ defined as following  as new set of variables,
\be
\zeta \doteq	 -\Psi + \pi' + \mathcal{H} \pi,
\ee
After substitution in mathematica we get,
\begin{figure} [H]
 \includegraphics [scale=0.6]{mathem_270418}
 \end{figure}
so we get,
%\begin{align} 
%\text{cs}^2 (-\text{piC}(\tau )) \left(3 H'(\tau )-3 H(\tau )^2+\nabla 2\right)-3 H(\tau ) \left(\text{cs}^2 \Psi (\tau )+w \zeta (\tau )\right)-3 \text{cs}^2 \Phi '(\tau )+\zeta '(\tau )
%  \end{align} 
%  

%  \begin{align} 
% &{ \zeta' - (\mathcal{H} \pi)' +\Psi'+\mathcal{H}(1- 3w) ( \zeta- \mathcal{H} \pi+\Psi) } +3 {  \mathcal{H}}\Big( -c_s^2+ {w} \Big )\Psi - \, {\Psi'}- 3 c_s^2  \,{\Phi'} + {
% \Big( 3\mathcal{H}^2 (c_s^2 -w) + \mathcal{H}' (1-3c_s^2)\Big) \pi }
%           \nonumber
%   \\
%    &
% - c_s^2 {\nabla^2 \pi }
%    % Second order terms
%     -2 c_s^2  \Phi  {\nabla^2 \pi }  
%  %//////////////// 
%  +   (1-c_s^2)  \Psi {\nabla^2 \pi}
%  %////////////////
%  +3 c_s^2 \mathcal{H} (1+w)\pi {\nabla^2 \pi }
%      %////////////////
%        -   (1-c_s^2)  { (\zeta + \Psi) } \nabla^2 {\pi }
%                                       \nonumber
%   \\
%    &
%        %//////////////// 
%             +c_s^2 {\nabla  \Phi . \nabla \pi}
%   %//////////////// 
%        -(2 c_s^2-1) {\nabla  \Psi . \nabla \pi }  
%   %//////////////// 
% +\frac{\mathcal{H}} {2 } \Big(2+3w+c_s^2  \Big){\nabla  \pi . \nabla \pi} 
%    %//////////////// 
%     -2   (1-c_s^2){\nabla  \pi . {  \nabla {  (\zeta + \Psi)   }}}     =0
%    % Second order terms==0
%  \end{align} 
  which after simplifying in Fourier space,
    \begin{align} 
 &{ \zeta'  -3 \mathcal{H}(w \zeta +c_s^2 \Psi) +c_s^2  ({ 3 \mathcal{H}^2 - 3 \mathcal{H}' }) \pi-3 c_s^2 \Phi'  -c_s^2 \nabla^2 \pi }=0
  \end{align} 
  and equation for updating $\pi$ is as following,
  \be
  \pi'-\zeta+\mathcal{H} \pi -\Psi=0:
  \ee
  Lets look at the two equation for the limit $c_s^2, w \to 0$
 \begin{align} 
  \zeta' =0
  \end{align} 
  which shows that $\zeta =C$ is a constant,
  and equation for updating $\pi$ is as following,
  \be
  \pi'-\zeta+\mathcal{H} \pi -\Psi=0
  \ee
  which is updated by $\Psi$ and $\mathcal{H} \pi$.\\
  If we only assume $c_s^2 \to 0$ and not $w$ we have,
   \begin{align} 
  \zeta' -3\mathcal{H}w \zeta =0
  \end{align} 
The solution in matter dominated universe $\mathcal{H}=2/\tau$ is,
\begin{figure} [H]
 \includegraphics [scale=0.6]{mathem_270418_2}
 \end{figure}
 where for $w<-0$ which is the case for scalar field equation of state, we have decaying solution which decays more than matter! It is interesting since in the level of the equations of motion we know that the cancelation between $\pi'+ \mathcal{H} \pi$ and $\Psi$ decreases. So if we provide the right initial condition, we guarantee that at any redshift the stress tensor does not blow up!
 
\section{Numerical solver}
 We choose Leap frog method to solve the two first order linear differential equations as following. The two equations for updating $\zeta$ and $\pi$ are,
  \begin{align} 
 &{ \zeta'=  3 \mathcal{H}(w \zeta +c_s^2 \Psi) -c_s^2  ({ 3 \mathcal{H}^2 - 3 \mathcal{H}' }) \pi + 3 c_s^2 \Phi'  +c_s^2 \nabla^2 \pi }
  \end{align} 
  \be
  \pi'=\zeta-\mathcal{H} \pi +\Psi
  \ee
  So for the leapfrog method we can write,
  \be
  \zeta_{n+1}=\zeta_{n} + \zeta'_{n+\frac{1}{2}} \Delta t
  \ee
  where $ \zeta'_{n+1/2}$ reads from the differential equation as following,
  \be
  \zeta'_{n+\frac{1}{2}}=3 \mathcal{H}_{n+\frac{1}{2}}(w \zeta_{n+\frac{1}{2}} +c_s^2 \Psi_{n+\frac{1}{2}}) -c_s^2(   3 \mathcal{H}^2_{n+\frac{1}{2}} 
  - 3 \mathcal{H}' _{n+\frac{1}{2}}) \pi_{n+\frac{1}{2}}+ 3 c_s^2 \Phi'_{n+\frac{1}{2}}  +c_s^2 \nabla^2 \pi_{n+\frac{1}{2}}
    \ee
  and for the $\pi$,
    \be
      \pi_{n+1}=\pi_{n} + \pi'_{n+\frac{1}{2}} \Delta t
    \ee
     \be
  \pi'_{n+\frac{1}{2}}=\zeta_{n+\frac{1}{2}}  -\mathcal{H}_{n+\frac{1}{2}}  \pi_{n+\frac{1}{2}}  +\Psi_{n+\frac{1}{2}} 
      \ee
     Since the scalar field Stress energy tensor must be synchronized with particles stress tensor, we need to have all the variables at the same step which is $n$, so we need to write all the terms at step ${n+\frac{1}{2}} $ in terms of the values at step $n$ and $n+1$ as following. The easiest model to calculate $F_{{n+\frac{1}{2}} }$ is by taking average of the next and last step, so
     \be
     \zeta_{n+\frac{1}{2}} = \frac{ \zeta_{n+1} + \zeta_{n} }{2 }
     \ee
     and the same for all the other variables at step ${n+\frac{1}{2}}$. For $\pi$ we have,
     \be
      \pi_{n+1}= \pi_{n} + \Delta t\Big [ \frac{ \zeta_{n+1} + \zeta_{n} }{2 }   -\mathcal{H}_{n+\frac{1}{2}}  ( \frac{ \pi_{n+1} + \pi_{n} }{2 })  +\frac{ \Psi_{n+1} + \Psi_{n} }{2 } \Big ]
     \ee
         \be
      \pi_{n+1}=  \frac{1}{1- \mathcal{H}_{n+\frac{1}{2}} \Delta t/2}\Bigg[ \pi_{n} + \Delta t\Big [ \frac{ \zeta_{n+1} + \zeta_{n} }{2 }   -\mathcal{H}_{n+\frac{1}{2}}   \frac{  \pi_{n} }{2 } +\frac{ \Psi_{n+1} + \Psi_{n} }{2 } \Big ] \Bigg]
     \ee
     For the part $\nabla^2 \pi_{n+\frac{1}{2}}$ we use predictor corrector method in the control "if" and also for $\Psi_{n+\frac{1}{2}} $ we use corrector predictor method but for the first guess we use $\Psi_{n+\frac{1}{2}}= \Psi_{n} + \Psi'_{n} \Delta t/2$


  %%%%%%%%%%%%%%%%%
   %%%%%%%%%%%%%%%%%
 %%%%%%%%%%%%%%%%%
   \subsection{Non-linear calculation and numerical solver}
    \begin{align} 
 &{ \pi''+\mathcal{H}(1- 3w) \pi' } +3 {  \mathcal{H}}\Big( -c_s^2+ {w} \Big )\Psi - \, {\Psi'}- 3 c_s^2  \,{\Phi'} + {
 \Big( 3\mathcal{H}^2 (c_s^2 -w) + \mathcal{H}' (1-3c_s^2)\Big) \pi }
           \nonumber
   \\
    &
 - c_s^2 {\nabla^2 \pi }
    % Second order terms
     -2 c_s^2  \Phi  {\nabla^2 \pi }  
  %//////////////// 
  +   (1-c_s^2)  \Psi {\nabla^2 \pi}
  %////////////////
  +3 c_s^2 \mathcal{H} (1+w)\pi {\nabla^2 \pi }
      %////////////////
        -   (1-c_s^2)  { (\mathcal{H} \pi+ \pi') } \nabla^2 {\pi }
                                       \nonumber
   \\
    &
        %//////////////// 
             +c_s^2 {\nabla  \Phi . \nabla \pi}
   %//////////////// 
        -(2 c_s^2-1) {\nabla  \Psi . \nabla \pi }  
   %//////////////// 
 +\frac{\mathcal{H}} {2 } \Big(2+3w+c_s^2  \Big){\nabla  \pi . \nabla \pi} 
    %//////////////// 
     -2   (1-c_s^2){\nabla  \pi . {  \nabla {  (\mathcal{H} \pi+ \pi')   }}}     =0
  \end{align} 
\begin{align}
 & T_0^0 (Gev)=  \Omega^0_{kess} a^{-3 w}  \Bigg[1+ \frac{1+w}{c_s^2} \Big(- 3 \mathcal{H}c_s^2 \pi- \Psi+   {({\pi'}+ \mathcal{H} \pi) }    \Big )   \Bigg ]
\nonumber \\ &
T^{i}_{0}(Gev)= - \Omega^0_{kess} a^{-3 w} (1+w) \partial _i \pi 
\nonumber \\ &
T_{j}^{i}(Gev)= w  \, \Omega^0_{kess} a^{-3 w} \Bigg ( 1+  \frac{1+w}{w}\Big [ -3 \mathcal{H} w \pi- \Psi +   {({\pi'}+ \mathcal{H} \pi) }\Big] \delta_{j}^{i}   \Bigg) 
\end{align}
we take $\pi$ and $\zeta$ defined as following  as new set of variables,
\be
\zeta \doteq	 -\Psi + \pi' + \mathcal{H} \pi,
\ee
After substitution in mathematica we get,
so we get,
  \begin{align} 
 &{ \zeta' - (\mathcal{H} \pi)' +\Psi'+\mathcal{H}(1- 3w) ( \zeta- \mathcal{H} \pi+\Psi) } +3 {  \mathcal{H}}\Big( -c_s^2+ {w} \Big )\Psi - \, {\Psi'}- 3 c_s^2  \,{\Phi'} + {
 \Big( 3\mathcal{H}^2 (c_s^2 -w) + \mathcal{H}' (1-3c_s^2)\Big) \pi }
           \nonumber
   \\
    &
 - c_s^2 {\nabla^2 \pi }
    % Second order terms
     -2 c_s^2  \Phi  {\nabla^2 \pi }  
  %//////////////// 
  +   (1-c_s^2)  \Psi {\nabla^2 \pi}
  %////////////////
  +3 c_s^2 \mathcal{H} (1+w)\pi {\nabla^2 \pi }
      %////////////////
        -   (1-c_s^2)  { (\zeta + \Psi) } \nabla^2 {\pi }
                                       \nonumber
   \\
    &
        %//////////////// 
             +c_s^2 {\nabla  \Phi . \nabla \pi}
   %//////////////// 
        -(2 c_s^2-1) {\nabla  \Psi . \nabla \pi }  
   %//////////////// 
 +\frac{\mathcal{H}} {2 } \Big(2+3w+c_s^2  \Big){\nabla  \pi . \nabla \pi} 
    %//////////////// 
     -2   (1-c_s^2){\nabla  \pi . {  \nabla {  (\zeta + \Psi)   }}}     =0
    % Second order terms==0
  \end{align} 

 %%%%%%%%%%%%%%%%%
   %%%%%%%%%%%%%%%%%
 %%%%%%%%%%%%%%%%%

 \subsection{Initial condition making from Class to be implemented in Gevolution!}
 \subsection{Discuss about the output of Gevolution and dimensions }
 \subsection{Compare the Gevolution results for new variables with Class and our solution for different redshifts and also the stress tensor!}
 



\subsection{To check non-linear part of field equation, use Maldacena's relation for long modes or look at some other limits? {\color{red}To do! }! }
 %%%%%%%%%%%%%%%%%
   %%%%%%%%%%%%%%%%%
 %%%%%%%%%%%%%%%%%
\subsection{Check the equations  in class and Gevolution for $c_s^2 \to 0$ and$ w \to 0$ for new set of variables  to get the matter behaviour and cross check with class! {\color{red}To do! }}
  \subsection{To check non-linear part of field equation, use Maldacena's relation for long modes or look at some other limits? {\color{red}To do! }! }

 %%%%%%%%%%%%%%%%%
   %%%%%%%%%%%%%%%%%
 %%%%%%%%%%%%%%%%%


\section{Checking the effect of discrete lattice which Julian suggested? {\color{red}  Try to derive the result!}}
According to the email of Julian on 26April2018, he suggested 
      \begin{figure} [H]
 \includegraphics [scale=0.8]{Julian_discussion}
 \end{figure}
 Now I'm trying to do what he suggested! By using our $\Phi$ and $\Phi'$ obtained from big simulations. The result for a relatively good run is as following,
      \begin{figure} [H]
 \includegraphics [scale=0.5]{BigRun_Juliantest}
 \end{figure}
      \begin{figure} [H]
 \includegraphics [scale=0.8]{Julian_email}
 \end{figure}
       \begin{figure} [H]
 \includegraphics [scale=0.8]{Julian_discuss_01}
 \end{figure}
        \begin{figure} [H]
 \includegraphics [scale=0.8]{Julian_discuss_02}
 \end{figure}
        \begin{figure} [H]
 \includegraphics [scale=0.8]{Julian_discuss_03}
 \end{figure}

 
%{\color{red} If we use pureEFT flag in EFTcamb, what are the related parameters for k-essence case?  since the translation between the standard language with EFTcamb is not trivial according to table 1 of   \url{https://arxiv.org/pdf/1411.3712.pdf} }
%In the beginning we use minimally coupled quintessence flag in the EFTcamb to check the consistency, then we should try the pureEFT flag. We choose the quintessence flag according to \url{http://www.eftcamb.org/images/EFTCAMB_structure.pdf} in the second part.

   
\section{Comparing some important quantities in class and Gevolution for the same initial condition, like Hubble factor, matter power, potential power and ...}
Using the following python script for comparing Hubble factor in Gevolution and Class,
\begin{lstlisting}[language=Python]
# H in unit Mpc!
h=0.67556;
c=2.99*1.e5;
H0=100*h/c;
print("H0[1/Mpc]=100h/c: ",H0);
def Hubble_conf_Mpc(a):
    H0=0.00022593979933110373;w=-0.9;h=0.7;
    Omega_b=0.022032/h/h; Omega_cdm=0.12038/h/h;
    Omega_m=0.3; Omega_Lambda=0;
    Omega_rad=9.16681e-05; Omega_kessence=1.-Omega_m-Omega_rad;
    return H0*np.sqrt(Omega_m*(a**-3)+Omega_rad*(a**-4)+Omega_kessence*(a**(-3*(1+w))))*a
Hclass_cs_e3=Bg_class_cs_e3[:,3]
aclass_cs_e3=1./(1+Bg_class_cs_e3[:,0])
Hclass_cs_e0=Bg_class_cs_e0[:,3]
aclass_cs_e0=1./(1+Bg_class_cs_e0[:,0])
a=np.arange(0.001,2,0.0001)
# plt.loglog(aclass_cs_e3,Hclass_cs_e3[:]*aclass_cs_e3,color="red",label="Class_cs3")
# plt.loglog(aclass_cs_e0,Hclass_cs_e0[:]*aclass_cs_e0,color="purple",label="Class_cs0")
# plt.loglog(aclass_cs_e3,Hubble_conf_Mpc(aclass_cs_e3),color="blue",label="Formlua")
plt.loglog(aclass_cs_e3,np.abs(Hclass_cs_e3[:]*aclass_cs_e3-Hubble_conf_Mpc(aclass_cs_e3)),color="green",label="Difference")
plt.loglog(aclass_cs_e0,np.abs(Hclass_cs_e0[:]*aclass_cs_e0-Hubble_conf_Mpc(aclass_cs_e0)),color="khaki",label="Difference2")

# plt.loglog(a,H0*myHa,color="green",label="FormluaII")

Hubble_conf_Mpc(a)
plt.xlim(1e-3,2)
# plt.ylim(1e-4,2e1)
plt.legend()
plt.show()
\end{lstlisting}
    \begin{figure} [H]
 \includegraphics [scale=0.5]{Hubble_comp.jpg}
 \end{figure}
 {\color{red} Why the error increases in time? what is neglected?}

\section{Python script to make initial conditions from hi-class and hi-class internal code}
 \section{Miguel}
 Compare Synch class with hiclass on $\alpha$ \\
 Look at the equations in hiclass, about factor "3". \\
 
\end{document}
 